
\subsection{Axiom of infinity}


The axiom of infinity states that:

$\exists I (\varnothing \in I \land \forall x \in I((x\lor \{x\})\in I))$

There exists a set, called the infinite set. This contains the empty set, and for all elements in \(I\) the set also contains the successor to it.

\subsubsection{Sequential function}

Let's define the sequential function:

$s(n):=\{n\lor \{n\}\}$

We can now rewrite the axiom of infinity as:

$\exists \mathbb{N} (\varnothing \in \mathbb{N} \land \forall x \in \mathbb{N}(s(x)\in \mathbb{N}))$

\subsubsection{Zero}

This set contains the null set: \(\varnothing \in \mathbb{N} \).

Zero is defined as the empty set.

$0:=\{\}$

\subsubsection{Natural numbers}

For all elements in the infinite set, there also exists another element in the infinite set: \(\forall x \in \mathbb{N}((x\lor \{x\})\in \mathbb{N}) \).

We then define all sequential numbers as the set of all preceding numbers. So:

$1:=\{0\}=\{\{\}\}$

$2:=\{0,1\}=\{\{\},\{\{\}\}\}$

$3:=\{0,1,2\}=\{\{\},\{\{\}\},\{\{\},\{\{\}\}\}\}$

\subsubsection{Existence of natural numbers}

Does each natural number exist? We know the infinite set exists, and we also know the axiom schema of specification:

Point is: For each set, all finite subsets exist. PROVE ELSEWHERE

\subsubsection{From infinite set to natural set}

We don’t know I is limited to natural numbers. Could contain urelements etc.

\subsubsection{More}

Infinite set axiom written using N. should be I


I could be superset of N, for example set of all natural numbers, and also the set containing the set containing 2.

Can extract N using axiom of specification

We need a way to define the set of natural numbers:

\(\forall n (n\in \mathbb{N}\leftrightarrow ([n=\emptyset \lor \exists k (n=k\lor \{k\})]\land ))\)

If we can can define N, we can show it exists from specicication

\(\forall x \exists s [P(x)\leftrightarrow (x\in s)]\)

\(\forall n \exists s [n\in N \leftrightarrow (n\in s)]\)

