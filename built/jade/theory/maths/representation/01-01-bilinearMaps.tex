
\subsection{Bilinear maps}

A bilinear map (or function) is a map from two inputs to an output which preserves addition and scalar multiplication. This is in contrast to a linear map, which only has one input.

In addition, the function is linear in both arguments.

That is if function \(f\) is bilinear then:

$X=aM+bN$

$Y=cO+dP$

$f(X,Y)=f(aM+bN,cO+dP)$

$f(X,Y)=f(aM,cO+dP)+f(bN,cO+dP)$

$f(X,Y)=f(aM,cO)+f(aM,dP)+f(bN,cO)+f(bN,dP)$

$f(X,Y)=acf(M,O)+adf(M,P)+bcf(N,O)+bdf(N,P)$

Note that:

$f(X,Y)=f(X+0,Y)$

$f(X,Y)=f(X,Y)+f(0,Y)$

$(0,Y)=0$

That is, if any input is \(0\) in an additative sense, the value of the map must be zero.

