\subsection{Substitution}

If we have a tautology, then we can substitute the formula of any propositional variable with any formula to arrive at any other tautology.

For example, we know that \(\theta \lor \neg \theta \) is a tautology. This means that an arbitrary formula for \(\theta \) is also a tautology.

An example is \((\gamma \land \alpha )\lor \neg (\gamma \land \alpha )\), which we know is a tautology, without having to examine each variable.

\subsection{Syntactic consequence}

Let us call the first formula \(A\) and the second \(B\). We can then say:

$A\vdash B$

This says that: if \(A\) is true, then we can deduce that \(B\) is true using steps such as substitution.

\subsection{Modus Ponens}

Modus Ponens is a deduction rule. This allows us to use stpes other than substitution to derive new tautologies.

If \(A\) implies \(B\), and \(A\) is true, then \(B\) is also true.

$(\theta \rightarrow \gamma )\land \theta \Rightarrow \gamma $

That is, if we can show that the following are true:

$\theta \rightarrow \gamma $

$\theta $

We can infer that the following is also true:

$\gamma $

\subsection{Inference with horn clauses}

If the horn clause is true, and so is the normal form part, then \(X\) is also true.

As all inference with horn clauses uses Modus Ponens, it is sound.

Inference with horn clauses is also complete.

\subsection{Theory}

Results derived from substitution or induction are called theorems. Theorems often divided into:

\begin{itemize}
\item Theorems - important results
\item Lemmas - results used for later theorems
\item Corollaries - readily deduced from a theorem
\end{itemize}

We take a set of axioms, as true, and a deduction rule which enables us to derive additional formulae, or theorems. The collection of axioms and theorems is known as the theory.
