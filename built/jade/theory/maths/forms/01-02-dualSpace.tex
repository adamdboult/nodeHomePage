
\subsection{Dual space}

The dual space \(V^*\) of vector space \(V\) is the set of all linear forms, \(\hom(V,F)\).
\subsubsection{The dual space is itself a vector space}

\(v\in V\)

\(f\in F\)

\(av = f\)

\(bv = g\)

\((a\oplus b)v=f+g\)

\((a\oplus b)v=av + bv\)

So there is some operation we can do on two members of dual space

Linear in addition. That is, if we have two dual "things", we can define the addition of functions as the operation which results int he outputs being added.

what about linear in scalar? same approach.

Well we define 

\(c\odot a)=cav\)

\subsubsection{The dual space has the same dimension as the underlying vector space}

