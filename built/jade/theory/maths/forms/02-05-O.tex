
\subsection{Orthogonal groups \(O(n, F)\)}

\subsubsection{Recap: Metric-preserving transformations}

The bilinear form is:

\(u^TMv\)

The transformations which preserve this are:

\(P^TMP=M\)

\subsubsection{The orthogonal group}

If the metic is \(M=I\) then the condition is:

\(P^TP=I\)

\(P^T=P^{-1}\)

These form the orthogonal group.

We use \(O\) instead of \(P\):

\(O^T=O^{-1}\)

\subsubsection{Rotations and reflections}

The orthogonal group is the rotations and reflections.

\subsubsection{Parameters of the orthogonal group}

The orthogonal group depends on the dimension of the vector space, and the underlying field. So we can have:
\begin{itemize}
\item \(O(n, R)\); and
\item \(O(n, C)\).
\end{itemize}

\subsubsection{We generally refer only to the reals}

\(O(n)\) means \(O(n,R)\).

The generally refer to the reals only.

