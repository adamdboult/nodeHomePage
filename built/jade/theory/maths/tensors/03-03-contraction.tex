
\subsection{Tensor contraction}

We have a vector \(v\in V\) and \(w\in V^*\).

\(\mathbf v=\sum_i v^i \mathbf e_i\)

\(\mathbf w=\sum_i w_i \mathbf f^i\)

\(\mathbf w\mathbf v=[\sum_i v^i \mathbf e_i][\sum_i w_i \mathbf f^i]\)

\(\mathbf w\mathbf v=\sum_i \sum_j [v^i \mathbf e_i][w_j \mathbf f^j]\)

\(\mathbf w\mathbf v=\sum_i \sum_j v^i w_j \mathbf e_i\mathbf f^j\)

We use the dual basis so:

\(\mathbf w\mathbf v=\sum_i \sum_j v^i w_j \mathbf e_i\mathbf e^j\)

\(\mathbf w\mathbf v=\sum_i \sum_j v^i w_j \delta_i^j\)

We can see that this value is unchanged when there is a change in basis.

What if these were both from \(V\)?

\(\mathbf v=\sum_i v^i \mathbf e_i\)

\(\mathbf w=\sum_i w^i \mathbf e_i\)

\(\mathbf w\mathbf v=[\sum_i v^i \mathbf e_i][\sum_i w^i \mathbf e_i]\)

\(\mathbf w\mathbf v=\sum_i \sum_j v^i w^j \mathbf e_i\mathbf e_i\)

This term is dependent on the basis, and so we do not contract.

So if we have \(v_iw^i\), we can contract, because the result (calculated from the components) does not depend on the basis.

But if we have \(v_iw_i\), the result (calcualted from the components) will change depending on the choice of basis.

We define a new object

\(c=\sum_i w^iv_i\)

This new term, c, does not depend on \(i\), and so we have contracted the index.

