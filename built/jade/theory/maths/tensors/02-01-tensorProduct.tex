
\subsection{Tensor product}

We have spaces \(V\) and \(W\) over field \(F\). If we have a linear operation which takes a vector from each space and returns a scalar from the underlying field, it is an element of the tensor product of the two spaces.

For example if we have two vectors:

\(v=e_iv^i\)

\(w=e_jw^j\)

A tensor product would take these and return a scalar.

There are three types of tensor products:

\begin{itemize}
\item Both are from the vector space
\item \(T_{ij}v^iw^j\)
\item \(T_{ij} \in V\otimes W\)
\item Both are from the dual space
\item \(T^{ij}v_iw_j\)
\item \(T_{ij} \in V^*\otimes W^*\)
\item One is from each space
\item \(T_i^jv^iw_j\)
\item \(T_{ij} \in V\otimes W^*\)
\end{itemize}


As a vector space, we can add together tensor products, and do scalar multiplication.

\subsubsection{Homomorphisms}

We can define homomorphisms in terms of tensor products.

\(Hom (V) =V \otimes V^*\)

\(T_j^i\)

We use the dual space for the second argument. This is because it ensures that changes to the bases do not affect the maps.

\(w^j=T_i^j v^i\)

