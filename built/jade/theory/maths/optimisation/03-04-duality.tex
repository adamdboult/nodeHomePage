
\subsection{Primal and dual problems}

\subsubsection{The primal problem}

We already have this.

\subsubsection{The dual problem}

We can define the Lagrangian dual function:

\(g(\lambda, \nu ) = \inf_{x\in X} \mathcal{L}(x, \lambda ,\nu )\)

That is, we have a function which chooses the returns the value of the optimised Lagrangian, given the values of \(\lambda \) and \(\nu\).

This is an unconstrained function.

We can prove this function is concave (how?).

The infimum of a set of concave (and therefore also affine) functions is concave.

The supremum of a set of convex (and therefore also affine) functions is convex.

Given a function with inputs \(x\), what values of \(x\) maximise the function?

We explore constrained and unconstrained optimisation. The former is where restrictions are placed on vector \(x\), such as a budget constraint in economics.

\subsubsection{The dual problem is concave}

\subsubsection{The duality gap}

We refer to the optimal solution for the primary problem as \(p^*\), and the optimal solution for the dual problem as \(d^*\).

The duality gap is \(p^*-d^*\).

