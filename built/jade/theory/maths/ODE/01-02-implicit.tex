
\subsection{Implicit and explit differential equations}

An ordinary differential equation is one with only one independent variable. For example:

$\frac{dy}{dx}=f(x)$

The order of a differential equation is the number of differentials of \(y\) included. For example one with the second derivative of \(y\) is of order \(2\).

Ordinary equations can can either implicit or explicit. An explicit function shows the highest order derivative as a function of other terms.

An implicit function is one which is not explicit.

A linear ODE is an explicit ODE where the derivative terms of \(y\) do not multiply together, that is, in the form:

$y^{(n)}=\sum_ia_i(x)y^{(i)}+r(x)$

\subsubsection{First-order ODEs}

We have an evolution:

\(\frac{dy}{dt}=f(t,y)\)

And a starting condition:

\(y_0=f(t_0)\)

We now discuss various ways to solve these.

