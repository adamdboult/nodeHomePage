\subsection{Rational numbers}


\subsubsection{Defining rational numbers}

We previously defined integers in terms of natural numbers. Similarly we can define rational numbers in terms of integers.

$\forall ab \in \mathbb{I} (¬(b=0)\rightarrow \exists c (b.c=a))$

A rational is an ordered pair of integers.

$\{\{a\},\{a,b\}\}$

So that:

$\{\{a\},\{a,b\}\}=\frac{a}{b}$

\subsubsection{Converting integers to rational numbers}

Integers can be shown as rational numbers using:

$(i,1)$

Integers can then be turned into rational numbers:

$\mathbb{Q}=\frac{a}{1}$

$a=\frac{a_1}{a_2}$

$b=\frac{b_1}{b_2}$

$c=\frac{c_1}{c_2}$

\subsubsection{Equivalence classes of rationals}

There are an infinite number of ways to write any rational number, as with integers. \(\frac{1}{2}\) can be written as \(\frac{1}{2}\), \(\frac{-2}{-4}\) etc.

The class of these terms form an equivalence class.

We can show these are equal:

$\frac{a}{b}=\{\{a\},\{a,b\}\}$

$\frac{ca}{cb}=\{\{a\},\{a,b\}\}$

$\frac{ca}{cb}=\{\{ca\},\{ca,cb\}\}$

$\{\{a\},\{a,b\}\}=\{\{ca\},\{ca,cb\}\}$

