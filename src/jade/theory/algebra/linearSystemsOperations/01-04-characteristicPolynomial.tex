
\subsection{Calculating eigenvalues and eigenvectors using the characteristic polynomial}

The characteristic polynomial of a matrix is a polynomial whose roots are the eigenvalues of the matrix.

We know from the definition of eigenvalues and eigenvectors that:

\(Av=\lambda v\)

Note that

\(Av-\lambda v=0\)

\(Av-\lambda Iv=0\)

\((A-\lambda I)v=0\)

Trivially we see that \(v=0\) is a solution.

Otherwise matrix \(A-\lambda I\) must be non-invertible. That is:

\(Det(A-\lambda I)=0\)

\subsection{Calculating eigenvalues}

For example

\(A=\begin{bmatrix}2&1\\1 & 2\end{bmatrix}\)

\(A-\lambda I=\begin{bmatrix}2-\lambda &1\\1 & 2-\lambda \end{bmatrix}\)

\(Det(A-\lambda I)=(2-\lambda )(2-\lambda )-1\)

When this is \(0\).

\((2-\lambda )(2-\lambda )-1=0\)

\(\lambda =1,3\)

\subsection{Calculating eigenvectors}

You can plug this into the original problem.

For example

\(Av=3v\)

\(\begin{bmatrix}2&1\\1 & 2\end{bmatrix}\begin{bmatrix}x_1\\x_2\end{bmatrix}=3\begin{bmatrix}x_1\\x_2\end{bmatrix}\)

As vectors can be defined at any point on the line, we normalise \(x_1=1\).

\(\begin{bmatrix}2&1\\1 & 2\end{bmatrix}\begin{bmatrix}1\\x_2\end{bmatrix}=\begin{bmatrix}3\\3x_2\end{bmatrix}\)

Here \(x_2=1\) and so the eigenvector corresponding to eigenvalue \(3\) is:

\(\begin{bmatrix}1\\1\end{bmatrix}\)

