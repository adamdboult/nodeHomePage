\subsection{Recap}

Previously we defined addition and multiplication in terms of successive use of the sucessor function. That is, the definition of addition was:

\(\forall a \in \mathbb{N} (a+0=a)\)

\(\forall a b \in \mathbb{N} (a+s(b)=s(a+b))\)

And similarly for multiplication:

\(\forall a \in \mathbb{N} (a.0=0)\)

\(\forall a b \in \mathbb{N} (a.s(b)=a.b+a)\)

Additional functions could also be defined, following the same pattern:

\(\forall a \in \mathbb{N} (a\oplus _n 0=c)\)

\(\forall a b \in \mathbb{N} (a\oplus _{n} s(b)=(a\oplus_{n} b)\oplus_{n-1}a)\)

\subsection{Powers}

Powers can also be defined:

\(\forall a \in \mathbb{N} a^0=1\)

\(\forall a b \in \mathbb{N} a^{s(b)}=a^b.a\)

\subsection{Example}

So \(2^2\) can be calculated like:

\(2^2=2^{s(1)}\)

\(2^{s(1)}=2.2^1\)

\(2.2^1=2.2.2^0\)

\(2.2.2^0=2.2.1\)

\(2.2.1=4\)

Unlike addition and multiplication, exponention is not commutative. That is

\(a^b\ne b^a\)

\subsection{Exponential rules}

\(a^ba^c=a^{b+c}\)

\((a^b)^c=a^{bc}\)

\((ab)^c=a^cb^c\)

