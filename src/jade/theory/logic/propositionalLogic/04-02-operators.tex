\subsection{How many unique operators are there?}

An arbitrary operator takes \(n\) inputs are returns \(T\) or \(F\).

With \(0\) inputs there is one posible permutation. For every additional input the number of possible permutations doubles. Therefore there are \(2^n\) possible permutations.

For the operator with one permutation there are two operators. For every additional permutation the number of operator doubles. Therefore there are \(2^{(2^n)}\) possible operations.

With \(0\) inputs, we need \(2\) different operators to cover all outputs. For \(1\) input we need \(4\) and for \(2\) inputs we need \(16\).

\subsection{We don't need \(0\)-ary operators}

There are two unique \(0\)-ary operators. One always returns \(T\) and the other always returns \(F\). These are already described.

\subsection{We need one unary operator}

For the operators with \(1\) input we have:

\begin{itemize}
\item one which always returns \(T\)
\item one which always returns \(F\)
\item one which always returns the same as the input
\item one which returns the opposite of the input
\end{itemize}

It is this last one, negation, shown as \(\neg \) and is of most interest.

\subsection{We can use a subset of binary operators}

The full list of binary operators are included below.

Of these, the first two are \(0\)-ary operators, and so are not needed. The next four are unary operators, and so are not needed.

The non-implications can be rewritten using negation.

\subsection{Brackets replace the need for n-ary operators}

N-ary operators contain \(3\) or more inputs.

N-ary operators can be defined in terms of binary operators.

As an example if we want an operator to return positive if all inputs are true, we can use:

\((\theta \land \gamma )\land \beta \)

