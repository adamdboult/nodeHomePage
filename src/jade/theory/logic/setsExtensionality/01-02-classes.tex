
\subsection{Equivalence classes}

We have already ready defined the relationship equality, between terms.

\(a=b\).

Sometimes we may wish to talk about a collection of terms which are all equal to each other. This is an equivalence class.

Though we have not yet defined them, integers are example of this. For example \(-1\) can be written as \(0-1\), \(1-2\) and so on.

\(\forall y \forall x x=y\rightarrow x\in z\)

For all sets, we can call the class of all sets equal to the set an equivalence class.

This does not necessarily exist.

