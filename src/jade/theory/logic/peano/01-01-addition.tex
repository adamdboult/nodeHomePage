
\subsection{Definition}
Let’s add another function: addition. Defined by:

\(\forall a \in \mathbb{N} (a+0=a)\)

\(\forall a b \in \mathbb{N} (a+s(b)=s(a+b))\)

That is, adding zero to a number doesn’t change it, and \((a+b)+1=a+(b+1)\).

\subsection{Example}

Let’s use this to solve \(1+2\):

\(1+2=1+s(1)\)

\(1+s(1)=s(1+1)\)

\(s(1+1)=s(1+s(0))\)

\(s(1+s(0))=s(s(1+0))\)

\(s(s(1+0))=s(s(1))\)

\(s(s(1))=s(2)\)

\(s(2)=3\)

\(1+2=3\)

All addition can be done iteratively like this.

\subsection{Commutative property of addition}

Addition is commutative:

\(x+y=y+x\)

\subsection{Associative property of addition}

Addition is associative:

\(x+(y+z)=(x+y)+z\)

