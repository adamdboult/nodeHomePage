
\subsection{Random utility functions}

With non-discrete choice a consumer chooses how much of product \(x\) to consume.

The utility function is of the form:

\(U_i(x_1,...,x_m;d)\)

And the consumer chooses how much of \(x_i\) to consume to maximise this, subject to the budget constraint.

We include features relating to the individual, \(d\).

\subsubsection{Discrete choice}

The utility customer \(i\) gets from product \(j\) is:

\(U_{ij}=f_i(p,d)+\epsilon_{ij}\)

The customer chooses the product with the highest utility.

\subsubsection{Price preferences}

We start with a simple model, where the customer has price preference.

\(U_{ij}=-\beta_i p_{ij} +\epsilon_{ij}\)

\subsubsection{Product characteristics}

\(U_{ij}=\alpha_i x_j -\beta_i p_{ij} +\epsilon_{ij}\)

\subsubsection{Individual characteristics}

\(U_{ij}=\alpha_i x_j -\beta_i p_{ij} + \theta_j d_i +\epsilon_{ij}\)

\subsubsection{The general form}

We can convert this to the form:

\(U_{ij}=\Theta z_{ij} + \epsilon_{ij}\)

\subsubsection{The outside option}

We need to know the market share of the outside option. Do thsi theoretically. Eg number of customers in area, and 1 per day.

