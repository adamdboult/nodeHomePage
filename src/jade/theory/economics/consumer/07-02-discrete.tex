
\subsection{Modelling homogeneous preferences with multinomial logit}

\subsubsection{Recap}

Our model is:

\(U_{ij}=\Theta z_{ij} + \epsilon_{ij}\)

\(U_{ij}=\alpha_i x_j -\beta_i p_{ij} + \theta_j d_i +\epsilon_{ij}\)

\subsubsection{Homogeneous model}

We model all customers as having the same preferences.

\(U_{ij}=\alpha x_j -\beta p_{ij} + \theta_j d_i +\epsilon_{ij}\)

\subsubsection{The multinomial logit assumption}

If errors are IID and extreme we get:

\(P_{ij}=\dfrac{e^{\Theta z_j}}{\sum_k e^{\Theta z_k }}\)

\subsubsection{The outside option}

A user has the option of not buying anything.

\(U_0=0\)

This gives us the following shares:

\(P_{ij}=\dfrac{e^{\Theta z_j}}{e^0+\sum_{k=1} e^{\Theta z_k }}\)

\(P_{ij}=\dfrac{e^{\Theta z_j}}{1+\sum_{k=1} e^{\Theta z_k }}\)

\subsubsection{Own-price elasticity of demand}

\(P_{ij}=\dfrac{e^{\Theta z_j}}{1+\sum_{k=1}} e^{\Theta z_k }\)

\(P_{ij}=\dfrac{e^{\alpha x_j -\beta p_j + \theta_j d_i}}{1+\sum_{k=1} e^{ \alpha x_k -\beta p_k +\theta_k d_i}}\)

\(\dfrac{\delta P_{ij} }{\delta p_j}\dfrac{p_j}{P_{ij}}=-\beta p_j(1-P_{ij})\)

\(\dfrac{\delta P_{ij} }{\delta p_k}\dfrac{p_j}{P_{ij}}=\beta p_kP_{ij}\)

This means that the lower the price, the lower the own price elasticity of demand.

This means that mark ups are higher for cheaper goods, which doesn't always match reality.

This can be adjusted by changing the form. For example we could use \(\ln p\) or \(p^2\).

However, we are still getting the shape by assumption.

\subsubsection{Cross-price elasticity of demand}

\subsubsection{Getting aggregate market shares}

\(s_j = \dfrac{1}{n}\sum_i P_{ij}\)

