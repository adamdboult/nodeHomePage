
\subsection{Cournot competition and the Lerner index}


\subsubsection{Monopoly recap}

In the monopoly model we have:

\(\pi = pq-c\)

\(p[1+\dfrac{q}{p}\dfrac{\delta p}{\delta q}]=MC\)

The price elasticity of demand is: \(\epsilon = \dfrac{p}{q}\dfrac{\delta q}{\delta p}\)

\subsubsection{Cournot competition}

With competition, the elasticity of demand refers to the whole market, not just a single producer. Instead we have:

\(\epsilon = \dfrac{p}{Q}\dfrac{\delta Q}{\delta p}\)

\(Q=\sum_j q_j\)

We now get:

\(p[1+\dfrac{q}{Q}\dfrac{\delta Q}{\delta q}\dfrac{Q}{p}\dfrac{\delta p}{\delta Q}]=MC\)

\(p[1+\dfrac{\mu }{\epsilon }]=MC\)

Using the firm's size elasticity: \(\mu = \dfrac{q}{Q}\dfrac{\delta Q}{\delta q}\)

With monopoly this is:

\(\mu = 1\)

\subsubsection{The Lerner index}

The Lerner index is:

\(\dfrac{p-MC}{p}\)

In this model this is:

\(\dfrac{p-MC}{p}=-\dfrac{\mu}{\epsilon }\)

