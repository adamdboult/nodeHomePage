
\subsection{Profit}

The profit of a firm is the difference between revenue and costs.

\(\pi = pq-c\)

Where \(q\) is the amount producted, and \(p\) is the price, and \(c\) is a function of production.

\subsection{Maximising profit}

\(\pi = pq-c\)

The firm's production \(q\) affects the market price \(p\).

\(\dfrac{\delta \pi }{\delta q}= \dfrac{\delta }{\delta q} [pq-c]\)

\(\dfrac{\delta \pi }{\delta q}= p+q\dfrac{\delta p}{\delta q}-\dfrac{\delta c}{\delta q}\)

The firm chooses \(Q\) to maximise profits.

\(p+q\dfrac{\delta p}{\delta q}=\dfrac{\delta c}{\delta q}\)

The right side is marginal costs (MC), the left is marginal revenue.

\(p[1+\dfrac{q}{p}\dfrac{\delta p}{\delta q}]=MC\)

We know that the price elasticity of demand is: \(\epsilon = \dfrac{p}{q}\dfrac{\delta q}{\delta p}\)

So we have:

\(p[1+\dfrac{1 }{\epsilon }]=MC\)

\(p=\dfrac{\epsilon }{1+\epsilon }MC\)

\subsection{Intensive and extensive margins}

\(revenue = pq\)

\(MR=p +q\dfrac{\delta p}{\delta q}\)

\(p\) is the extensive margin.

\(q\dfrac{\delta p}{\delta q}\) is the (negative) intensive margin.

monopoly pricing. when lower prices, gain money on extensive margin. lose money on intensive margin.

