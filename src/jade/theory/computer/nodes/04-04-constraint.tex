
\subsection{Constraint Satisfaction Problem}

A CSP problem is one where we don't care about the path, we just want to identify the goal state.

For example, solving a sudoku

\subsubsection{Defining a CSP}

A CSP has:

\begin{itemize}
\item Variables \(X_i\).
\item Domain for each variable \(D_i\).
\item Constraints \(C_j\).
\end{itemize}

In a CSP there are a range of variables each with a domain. There are on top constraints on combinations of values.

A solution does not violate any constraint.

To solve, start with no allocations of variables. successor function assigns a value to an unassigned variable. goal test

Use heuristic minimum remaining value MRV: choose variable with fewest remaining legal values

Least constraining value: choose item in domain which constrains the least other moves

Forward checking. keep track of remaining legal moves for each variable. terminate if none left

After each move, update legal moves for each

Implement all this with recursive backtrack function, which returns a solution or failure. This is a depth first search

\subsubsection{Arc-consistency}

X-Y is arc consistent if all of domain of X is consistnet with some value of Y.

\subsubsection{Node-consistency}

X is node consistent if all of domain satisfies all unary constraint.

\subsubsection{Path-consistency}

Arc consistency for additional variables.

\subsubsection{Constraint propagation}

Constraint propagation can be used to prevent bad choices

We can check for:

\begin{itemize}
\item node-consistency
\item arc-consistency
\item path-consistency
\end{itemize}

\subsubsection{AC-3}

AC-3 algorihm makes a CSP arc-consistent

Take all arcs.

It may be possible to break the problem down into sub problems, making the problems much easier to solve.

Can do this before/after other algorithm.

