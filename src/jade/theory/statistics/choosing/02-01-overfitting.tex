
\subsection{Overfitting}

Role of lambda: high makes impact of more variables lower => high bias

Low makes impacts of more variables strong => high variance

Can trade off using cut off. only make positive if above \(0.7\)

How to use? difficult, as lambda within cost!

Can do similarly to d:

Run for a range of lambda (eg 0, 0.01, 0.02, 0.04, 0.08:~10), then pick from cross validation set

Low lambda always has low cost for training set, but not for cv set..

Regularisation: add to error term the size of the term. penalised large parameters

May not fit outside sample

High bias: eg house prices and size. linear would have high bias for out of scope sample (underfitting)

High variance: making polynomial passing through all data (overfitting)

Can reduce overfitting by reducing features either manaually or using models

OR regularisation: keep all features, but reduce magnitude of theta

\subsection{Regularisation}

Make cost function include size of \(\theta^2\) values

\(\min \dfrac{1}{2m} [\sum (h(x)-y)^2 + 1000 \theta 3 ^2 + 1000 \theta 4 ^2]\)

or more broadly:

\(\min \dfrac{1}{2m}[\sum ..... + \lambda \sum \theta j^2]\)

Tend to not include theta 0 as convention, no regularisation

Update for linear regression is

\(\theta j = \theta j -\alpha{(\dfrac{1}{m})* sum(h(x)-y)xj + (\lambda/m \theta j)}\)

\(\theta j = \theta j (1- \alpha \lambda / m) -alpha {(1/m)*\sum(h(x)-y)xj}\)

This is the same as before, but theta \(j\) updates from a smaller \(\theta \) \(j\) each time.

Normal equation needs a change

\((X'X)^-1X'y=\theta''\)

Now is

\((X'X+\lambda I)^-1X y'\)

although for theta 0, lambda zero, so indentiy matrix, but first element 0

REGULARISATION FOR REGULARISATION

add to end of \(J(\theta)\):

\(\dfrac{\lambda }{2m} .\sum \thetaj^21\)

update for \(\theta \)\(j\) \(j>0\):
is a as linear regression, but \(h(x)\) is a different function

