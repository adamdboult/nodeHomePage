
\subsection{Markov's inequality and Chebyshev's inequality}
\subsubsection{Lemma 1}

\(E[I_{X\ge a}]=P(X\ge a)\)

Consider the indicator function.

\(I_{X\ge a}\)

This is equal to \(0\) if \(X\) is below \(a\) and \(1\) otherwise.

We can take expectations of this.

\(E[I_{X\ge a}]=P(X\ge a).1+P(X<a).0=P(X\ge a)\)

\(E[I_{X\ge a}]=P(X\ge a)\)

\subsubsection{Lemma 2}

\(aI_{X\ge a}\le X\)

While \(X\) is below \(a\) the left side is equal to \(0\), which holds.

While \(X\) is equal to \(a\) the left side is equal to \(X\), which holds.

While \(X\) is above \(a\) the left side is equal to \(a\), which holds.

\subsubsection{Markov’s inequality}

\(P(X\ge a)\le \frac{\mu  }{a}\)

From above:

\(aI_{X\ge a}\le X\)

We can take expectations of both sides:

\(E[aI_{X\ge a}]\le E[X]\)

\(aP(X\ge a)\le E[X]\)

\(P(X\ge a)\le \frac{\mu  }{a}\)

\subsubsection{Chebyshev’s inequality}

We know from Markov’s inequality that:

\(P(X\ge a)\le \frac{\mu }{a}\)

Let’s take the variable \(X\) to be \((X-\mu )^2\)

\(P((X-\mu )^2\ge a)\le \frac{E[(X-\mu )^2]}{a}\)

\(P((X-\mu )^2\ge a)\le \frac{\sigma^2}{a}\)

\(P(|X-\mu | \ge \sqrt{a})\le \frac{\sigma^2}{a}\)

Take \(a\) to be a multiple \(k^2\) of the variance \(\sigma^2\).

\(a=k^2\sigma^2\)

\(P(|X-\mu | \ge k\sigma )\le \frac{\sigma^2}{k^2\sigma^2}\)

\(P(|X-\mu | \ge k\sigma )\le \frac{1}{k^2}\)

