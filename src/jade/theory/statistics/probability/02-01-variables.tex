
\subsection{Random variables}

\subsubsection{Defining variables}

We have a sample space, \(\Omega \). A random variable \(X\) is a mapping from the sample space to the real numbers:

$X: \Omega \rightarrow \mathbb{R}$

We can then define the set of elements in \(\Omega \). As an example, take a coin toss and a die roll. The sample space is:

$\{H1,H2,H3,H4,H5,H6,T1,T2,T3,T4,T5,T6\}$

A random variable could give us just the die value, such that:

$X(H1)=X(T1)=1$

We can define this more precisely using set-builder notation, by saying the following is defined for all	 \(c\in \mathbb{R}\):

$\{\omega |X(\omega )\le c\}$

That is, for any number random variable map \(X\), there is a corresponding subset of \(\Omega \) containing the \(\omega \)s in \(\Omega \) which map to less than \(c\).

\subsubsection{Multiple variables}

Multiple variables can be defined on the sample space. If we rolled a die we could define variables for

\begin{itemize}
\item Whether it was odd/even
\item Number on the die
\item Whether it was less than 3
\end{itemize}

With more die we could add even more variables

\subsubsection{Derivative variables}

If we define a variable \(X\), we can also define another variable \(Y=X^2\).

