
\subsection{Measure space}

\subsubsection{Existence of measure space}

We have a sample space, \(\Omega \) consisting of elementary events.

We have a \(\sigma\)-algebra over \(\Omega \) called \(F\). A \(\sigma\)-algebra takes a set a provides another set containing subsets closed under complement. The power set is an example.

For all events \(E\) in \(F\), the probability function \(P\) is defined.

This gives us the following measure space:

$(\Omega, F, P)$

\subsubsection{Subsets}

All events \(E\) are subsets of \(\Omega\)

\(\forall E\in F E\subseteq \Omega\)

\subsubsection{Union and intersection}

As events are sets, we can define the probability of algebra on sets. For example for two events \(E_i\) and \(E_j\) we can define:

\begin{itemize}
\item \(P(E_i\land E_j)\)
\item \(P(E_i\lor E_j)\)
\end{itemize}

\subsubsection{Mutually exclusive events}

Events are mutually exclusive if they are disjoint sets.

\subsubsection{Complements}

For each event \(E\), there is a complementary event \(E^C\) such that:

$E\lor E^C=\Omega$

$E\land E^C=\varnothing$

This exists by construction in the measure space.

