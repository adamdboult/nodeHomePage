
\subsection{Independence and conditional independence}

\subsubsection{Independence}

\(x\) is independent of \(y\) if:

$\forall x_i \in x,\forall y_j \in y (P(x_i|y_j)=P(x_i)$

If \(P(x_i|y_j)=P(x_i)\) then:

$P(x_i\land y_j)=P(x_i).P(y_j)$

This logic extends beyond just two events. If the events are independent then:

$P(x_i\land y_j \land z_j)=P(x_i).P(y_j \land z_k)=P(x_i).P(y_j).P(z_k)$

Note that because:

$P(x_i|y_j)=\frac{P(x_i\land y_j)}{P(y_j)}$

If two variables are independent

$P(x_i|y_j)=\frac{P(x_i)P(y_j)}{P(y_j)}$

$P(x_i|y_j)=P(x_i)$

\subsubsection{Conditional independence}

\(P(A\land B|X)=P(A|X)P(B|X)\)

This is the same as:

\(P(A|B\land X)=P(A|X)\)

