
\subsection{Expected value}
\subsubsection{Definition}

For a random variable (or vector of random variables), \(x\), we define the expected value of \(f(x)\) as :

\(E[f(x)]:=\sum f(x_i) P(x_i)\)

The expected value of random variable \(x\) is therefore this where \(f(x)=x\).

\(E(x)=\sum_i x_i P(x_i)\)

\subsubsection{Linearity of expectation}

We can show that \(E(x+y)=E(x)+E(y)\):

\(E[x+y]=\sum_i \sum_j (x_i+y_j) P(x_i \land y_j)\)

\(E[x+y]=\sum_i \sum_j x_i [P(x_i \land y_j)]+\sum_i \sum_j [y_j P(x_i \land y_j)]\)

\(E[x+y]=\sum_i x_i \sum_j [P(x_i \land y_j)]+\sum_j y_j \sum_i [P(x_i \land y_j)]\)

\(E[x+y]=\sum_i x_i P(x_i)+\sum_j y_j P(y_j)\)

\(E[x+y]=E[x]+E[y]\)

\subsubsection{Expectations of multiples}

Expectations

\(E(cx)=\sum_i cx P(x_i)\)

\(E(cx)=c\sum_i x P(x_i)\)

\(E(cx)=cE(x)\)

\subsubsection{Expectations of constants}

\(E(c)=\sum_i c_i P(c_i)\)

\(E(c)= cP(c)\)

\(E(c)= c\)

\subsubsection{Conditional expectation}

If \(Y\) is a variable we are interested in understanding, and \(X\) is a vector of other variables, we can create a model for \(Y\) given \(X\).

This is the conditional expectation.

\(E[Y|X]\)

\(E[P(Y|X)Y]\)

In the continuous case this is

\(E(Y|X)=\int_{-\infty }^{\infty }yP(y|X)dy\)

We can then identify an error vector.

\(\epsilon :=Y-E(Y|X)\)

So:

\(Y=E(Y|X)+\epsilon \)

Here \(Y\) is called the dependent variable, and \(X\) is called the dependent variable.

\subsubsection{Iterated expectation}

\(E[E[Y]]=E[Y]\)

\(E[E[Y|X]=E[Y]\)

