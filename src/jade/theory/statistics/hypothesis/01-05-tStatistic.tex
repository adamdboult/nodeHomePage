
\subsection{T-test for variable significance}

\subsubsection{T-statistic}

In practice we don't know the population standard deviation and so must estimate it instead.

We use the standard deviation on the sample.

\(z=\dfrac{\bar x-x_0}{s) }\)

\subsubsection{Student's t-distribution}

As we have used the sample standard deviation we have lost a degree of freedom, and can no longer model the variable as a normal distribution, as we did for the z-statistic.

We now have a distribution with an addition parameter, the number of degrees of freedom.

The number of degrees of freedom is \(n-1\).

As the sample size tends towards infinity, the distribution tends towards the normal distribution.

\subsubsection{Student's t-test}

\subsubsection{Confidence interval}


