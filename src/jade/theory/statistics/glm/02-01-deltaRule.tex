
\subsection{Delta rule}

\subsubsection{Introduction}

We want to train the parameters \(\theta \).

We can do this with gradient descent, by working out how much the loss function falls as we change each parameter.

The delta rule tells us how to do this.

\subsubsection{The loss function}

The error of the network is:

\(E=\sum_j\dfrac{1}{2}(y_j-a_j)^2\)

We know that \(a_j=a(\theta x_j)\) and so:

\(E=\sum_j\dfrac{1}{2}(y_j-a(\theta x_j))^2\)

\subsubsection{Minimising loss}

We can see the change in error as we change the parameter:

\(\dfrac{\delta E}{\delta \theta_i }=\sum_j \dfrac{\delta E}{\delta a_j}\dfrac{\delta a_j}{\delta z_j}\dfrac{\delta z_j}{\delta \theta_i}\)

\(\dfrac{\delta E}{\delta \theta_i }=-\sum_j(y_j-a_j)a'(z_j)x_{ij}\)

\subsubsection{Delta}

We define delta as:

\(\delta_i=-\dfrac{\delta E}{\delta z_j}=\sum_j(y_j-a_j)a'(z_j)\)

So:

\(\dfrac{\delta E}{\delta \theta_i }=\delta_i x_{ij}\)

\subsubsection{The delta rule}

We update the parameters using gradient descent:

\(\Delta \theta_i=\alpha \delta_i x_{ij}\)

