
\subsection{The binomial data generating process}

\subsubsection{Introduction}

For linear regression our data generating process is:

\(y=\alpha + \beta x +\epsilon \)

For linear classification our data generating process is:

\(z=\alpha + \beta x +\epsilon \)

And set \(y\) to \(1\) if \(z>0\)

Or:

\(y=\mathbf I[\alpha+\beta x+\epsilon >0]\)

\subsubsection{Probability of each class}

The probability that an invididual with characteristics \(x\) is classified as \(1\) is:

\(P_1=P(y=1|x)\)

\(P_1=P(\alpha + \beta x+\epsilon >0)\)

\(P_1=\int \mathbf I [\alpha + \beta x+\epsilon >0]f(\epsilon )d\epsilon \)

\(P_1=\int \mathbf I [\epsilon >-\alpha-\beta x ]f(\epsilon )d\epsilon \)

\(P_1=\int_{\epsilon=-\alpha-\beta x}^\infty f(\epsilon )d\epsilon \)

\(P_1=1-F(-\alpha-\beta x) \)

\subsubsection{Example: The logistic function}

Depending on the probability distribution of \(epsilon \) we have different classifiers.

For the logistic case we then have

\(P(y=1|x)=\dfrac{e^{\alpha + \beta x}{1+e^{\alpha + \beta x}}\)

