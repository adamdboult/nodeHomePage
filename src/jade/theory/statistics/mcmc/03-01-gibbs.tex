
\subsection{Gibb's sampling}

\subsubsection{Introduction}

As with Metropolis-Hastings, we want to generate samples for \(P(X)\) and use this to approximate its form.

We do this by using the conditional distribution. If \(X\) is a vector then we also have:

\(P(x_j|x_0,...,x_{j-1},x_{j+1},...,x_n)\)

We use our knowledge of this distribution.

Start with vector \(x_0\).

This has components \(x_{0,j}\)

To form the next vector \(x_1\) we loop through each component.

\(P(x_{1,0}|x_{0,0},x_{0,1},...,x_{0,n})\)

We use this to form \(x_{1,0}\)

However after th the first component we update this so it uses the updated variables.

\(P(x_{1,k}|x_{1,0},...,x_{1,k-1},x_{0,k},...,x_{0,n}\)

This means we only need to know the conditional distributions.

