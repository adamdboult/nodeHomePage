
\subsection{The Metropolis-Hastings algorithm}

\subsubsection{The Metropolis-Hastings algorithm}

The Metropolis-Hastings algorithm creates a set of samples \(x\) such that the distribution of the samples approaches the goal distribution.

\subsubsection{Initialisation}

The algorithm takes an arbitrary starting sample \(x_0\). It then must decide which sample to consider next.

\subsubsection{Generation}

It does this using a Markov chain. That is, there is a map \(g(x_j, x_i)\).

This distribution is generally a normal distribution around \(x_i\), making the process a random walk.

\subsubsection{Acceptance}

Now we have a considered sample, we can either accept or reject it. It is this step that makes the end distribution approximage the function.

We accept if \(\frac{f(x_j)}{f(x_i)}>u\), where \(u\) is a random variable between \(0\) and \(1\), generated each time.

We can calculate this because we know this function.

\subsubsection{Properties}


