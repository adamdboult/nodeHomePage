
\subsection{Normal equation}

\subsubsection{Least squares}

The square error is \(\sum_i (\hat{y_i}-y_i)^2\).

The differential of this with respect to \(\hat{\theta_j }\) is:

\(2\sum_i \dfrac{\delta \hat{y_i}}{\delta \hat{\theta_j}}(\hat{y_i}-y_i)\)

The stationary point is where this is zero:

\(\sum_i \dfrac{\delta \hat{y_i}}{\delta \hat{\theta_j}}(\hat{y_i}-y_i)=0\)

\subsubsection{Linear least squares}

Here, \(\hat{y_i}= \sum_j x_{ij}\hat{\theta_j}\)

Therefore: \(\dfrac{\delta \hat{y_i}}{\delta \hat{\theta_j}}=x_{ij}\)

And so the stationary point is where

\(\sum_i x_{ij}( \sum_j x_{ij}\hat{\theta_j }-y_i)=0\)

\(\sum_i x_{ij}( \sum_j x_{ij}\hat{\theta_j)}= \sum_i x_{ij}y_i\)

\subsubsection{Normal equation}

We can write this in matrix form.

\(X^TX\hat{\theta }=X^Ty\)

We can solve this as:

\(\hat{\theta }=(X^TX)^{-1}X^Ty\)

\subsubsection{Perfectly correlated variables}

If variables are perfectly correlated then we cannot solve the normal equation.

Intuitively, this is because for perfectly correlated variables there is no single best parameter, as changes to one parameter can be counteracted by changes to another.

