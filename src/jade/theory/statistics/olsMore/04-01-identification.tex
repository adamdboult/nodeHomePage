
\subsection{Identification through exogeneous variables}

Previously our structural model was:

\(Q=\alpha_1 + \beta_1 P + \epsilon_1 \)

\(Q=\alpha_2 + \beta_2 P + \epsilon_2 \)

And our reduced form:

\(P =\dfrac{\alpha_2-\alpha_1 }{\beta_1-\beta_2}+\dfrac{\epsilon_2-\epsilon_1 }{\beta_1-\beta_2}\)

\(Q =\dfrac{\alpha_2\beta_1-\alpha_1\beta_2 }{\beta_1-\beta_2}+\dfrac{\beta_1\epsilon_2 -\beta_2\epsilon_1}{\beta_1-\beta_2}\)

Or:

\(P=\pi_1 + \tau_1 \)

\(Q= \pi_2 + \tau_2\)

\subsubsection{Adding another variable}

This time we add another measured variable, \(I\).

\(Q=\alpha_1 + \beta_1 P + \theta_1 I + \epsilon_1 \)

\(Q=\alpha_2 + \beta_2 P + \theta_2 I + \epsilon_2 \)

The reduced form is now:

\(P =\dfrac{\alpha_2 -alpha_1 }{\beta_1-\beta_2}+\dfrac{\theta_2-\theta_1 }{\beta_1-\beta_2}I+\dfrac{\epsilon_2-\epsilon_1}{\beta_1-\beta_2}\)

\(Q =\dfrac{\alpha_2\beta_1-\alpha_1\beta_2 }{\beta_1-\beta_2}+\dfrac{\theta_2\beta_1-\theta_1\beta_2}{\beta_1-\beta_2}I+\dfrac{\beta_1\epsilon_2 -\beta_2\epsilon_1}{\beta_1-\beta_2}\)

Or:

\(P =\pi_{11} +\pi_{12}I + \tau_1 \)

\(Q= \pi_{21} +\pi_{22}I + \tau_2 \)

We can estimate \(\pi_1 \) and \(\pi_2 \) as \(\hat \pi_1\) and \(\hat \pi_2\) respectively.

We can now create estimators \(\hat \pi_{11}\), \(\hat \pi_{12}\), \(\hat \pi_{21}\) and \(\hat \pi_{22}\).

\subsubsection{Identification with an exogeneous variable}

We now have \(4\) estimators and \(6\) parameters, meaning that we still cannot identify the model.

\subsubsection{Partial identification}

Can we use \(\hat \pi \) to identify any of the structural parameters?

We know that:

\begin{itemize}
\item \(\pi_{11} =\dfrac{\alpha_2 -\alpha_1 }{\beta_1-\beta_2}\)
\item \(\pi_{12} =\dfrac{\theta_2-\theta_1}{\beta_1-\beta_2}\)
\item \(\pi_{21} =\dfrac{\alpha_2\beta_1-\alpha_1\beta_2}{\beta_1-\beta_2}\)
\item \(\pi_{22} =\dfrac{\theta_2\beta_1-\theta_1\beta_2}{\beta_1-\beta_2} \)
\end{itemize}

If the exogenous variable only affects one side of the equation, so \(\theta_1=0\), we have:

\begin{itemize}
\item \(\pi_{11} =\dfrac{\alpha_2 -\alpha_1 }{\beta_1-\beta_2}\)
\item \(\pi_{12} =\dfrac{\theta_2}{\beta_1-\beta_2}\)
\item \(\pi_{21} =\dfrac{\alpha_2\beta_1-\alpha_1\beta_2}{\beta_1-\beta_2}\)
\item \(\pi_{22} =\dfrac{\theta_2\beta_1}{\beta_1-\beta_2} \)
\end{itemize}

So we can see that:

\(\hat \beta_1 = \dfrac{\hat \pi_{22}}{\hat \pi_{12}}\)

This means we now have:

\begin{itemize}
\item \(\pi_{11} =\dfrac{\pi_{12}(\alpha_2 -\alpha_1 )}{\pi_{22}-\pi_{12}\beta_2}\)
\item \(\pi_{12} =\dfrac{\pi_{12}\theta_2}{\pi_{22}-\pi_{12}\beta_2}\)
\item \(\pi_{21} =\dfrac{\pi_{12}(\alpha_2\beta_1-\alpha_1\beta_2)}{\pi_{22}-\pi_{12}\beta_2}\)
\item \(\pi_{22} =\dfrac{\pi_{12}\theta_2\beta_1}{\pi_{22}-\pi_{12}\beta_2}\)
\end{itemize}

We can use this to also identify \(\alpha_1\).

\subsubsection{Complete identification}

If we have independent variables for each of the two equations, we can fully identify the model.

We will have \(6\) estimators and \(6\) parameters.

We are estimating:

\(Q=\alpha_1 + \beta_1 P + \theta_1 I + \epsilon_1 \)

\(Q=\alpha_2 + \beta_2 P + \theta_2 J + \epsilon_2 \)

\(I\) and \(J\) are essentially instrumental variables for the model.

\(I\) is an instrumental variable for demand shocks, and \(J\) is an instrumental variable for supply shocks.

