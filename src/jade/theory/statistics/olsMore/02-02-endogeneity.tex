
\subsection{Parameter identification problem with simultaneous equations}

\subsubsection{Identification terminology}

A system is under-identified if there are not enough estimators for all structural parameters.

A system is exactly identified if there are the same number of estimators as structural parameters.

A system is over-identified if there are more estimators than structural parameters.

In general we have in our structural form:

\(\sum^n_i\beta_{ij}y_i=\sum^m_i\gamma_{ij}x_i+\epsilon_j\)

This is a system with \(n\) endogeneous variables and \(m\) exogeneous variables.

We can write this in matrix form.

\(B\mathbf y =\Gamma \mathbf{x} + \mathbf{\epsilon}\)

We can use this to get:

\(\mathbf{y} =B^{-1}\Gamma \mathbf{x} + B^{-1}\mathbf{ \epsilon}\)

We estimate by placing restrictions on \(\Gamma\).

\subsubsection{Strucutral models}

If our data generating process is:

\(Q=\alpha + \beta P +\epsilon \)

We can estimate \(\alpha \)and \(\beta \) through measuring \(P\) and \(Q\).

If, however the data generating process involves simulataneous equations, we can have:

\(Q=\alpha_1 + \beta_1 P + \epsilon_1 \)

\(Q=\alpha_2 + \beta_2 P + \epsilon_2 \)

\subsubsection{Reduced form}

We can reduce this:

\(\alpha_1 + \beta_1 P + \epsilon_1 =\alpha_2 + \beta_2 P + \epsilon_2 \)

\((\alpha_1 -\alpha_2 )+ (\beta_1 -\beta_2 )P + (\epsilon_1 -\epsilon_2 )=0\)

\(P =\dfrac{\alpha_2-\alpha_1 }{\beta_1-\beta_2}+\dfrac{\epsilon_2-\epsilon_1 }{\beta_1-\beta_2}\)

We can rewrite this as:

\(P=\pi_1 + \tau_1 \)

Similarly we can reduce for \(Q\):

\(Q =\frac{\alpha_2\beta_1-\alpha_1\beta_2 }{\beta_1-\beta_2}+\frac{\beta_1\epsilon_2 -\beta_2\epsilon_1}{\beta_1-\beta_2}\)

\(Q= \pi_2 + \tau_2\)

\subsubsection{We can't directly estimate structural models}

If \(P\) is correlated with \(epsilon_1\) or \(\epsilon_2\) then our estimates for \(\beta_1\) and \(\beta_2\) will be biased.

This also affects \(Q\).

From the reduced forms we can see that \(P\) will be correlated, due to simultaneity.

\subsubsection{The identification problem}

We can estimate \(\pi_1 \) and \(\pi_2\), but this does not allow us to identify any of the structural parameters.

We have \(2\) estimators, but \(4\) parameters.

This is the identification problem.

