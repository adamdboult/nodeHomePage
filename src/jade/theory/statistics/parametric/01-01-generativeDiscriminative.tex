
\subsection{Generative and discriminative models}

\subsubsection{Recap}

For parametric models without dependent variables we have a form:

\(P(y| \theta )\)

And we have various ways of estimating \(\theta \).

We can write this as a likelihood function:

\(L(\theta ;y )=P(y|\theta)\)

\subsubsection{Discriminative models}

In discriminative models we learn:

\(P(y|X, \theta )\)

Which we can write as a likelihood function:

\(L(\theta ;y, X )=P(y| X, \theta)\)

\subsubsection{Generative models}

In generative models we learn:

\(P(y, X| \theta )\)

Which we can write as a likelihood function:

\(L(\theta ;y, X )=P(y, X|\theta)\)

We can use the generative model to calculate dependent probabilities.

\(P(y| X, \theta )=\dfrac{P(y, X| \theta )P(\theta )}{P(X, \theta )}\)

\(P(y| X, \theta )=\dfrac{P(y, X| \theta )}{P(X| \theta )}\)

