
\subsection{Order statistics}

\subsubsection{Defining order statistics}

The \(k\)th order statistic is the \(k\)th smallest value in a sample.

\(x_{(1)}\) is the smallest value in a sample, the minimum.

\(x_{(n)}\) is the largest value in a sample, the maximum.

\subsubsection{Probability distributions of order statistics}

The probability distribution of order statistics depends on the underlying probability distribution.

\subsubsection{Probability distribution of sample maximum}

If we have:

\(Y=\max \mathbf X\)

The probability distribution is:

\(P(Y\le y)=P(X_1\le y, X_2\le y,...,X_n\le y)\)

If these are iid we have:

\(P(Y\le y)=\prod_i P(X_i\le y)\)

\(F_y(y)=F_X(y)^n\)

The density function is:

\(f_y(y)=nF_X(y)^{n-1}f_x(y)\)

\subsubsection{Probability distribution of the sample minimum}

If we have:

\(Y=\min \mathbf X\)

The probability distribution is:

\(P(Y\le y)=P(X_1\ge y, X_2\ge y,...,X_n\ge y)\)

If these are iid we have:

\(P(Y\le y)=\prod_i P(X_i\ge y)\)

\(F_y(y)=[1-F_X(y)]^n\)

The density function is:

\(f_y(y)=-n[1-F_X(y)]^{n-1}f_x(y)\)

