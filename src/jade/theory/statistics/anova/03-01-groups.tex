
\subsection{The cross-sectional model}

\subsubsection{Hierarchical data}

Our standard linear model is:

\(y_i=\alpha + X_i\theta +\epsilon_i\)

If we had two sets of data we could view these as:

\(y_{i,0}=\alpha_0 + X_{i,0}\theta_0 +\epsilon_{i,0}\)

\(y_{i,1}=\alpha_1 + X_{i,1}\theta_1 +\epsilon_{i,1}\)

Here, the data data from \(1\) does not affect the parameters in \(2\).

\subsubsection{Pooled data}

If we think the data generating process is similar between models, then by restricting the freedom of parameters between models we can get more data for each estimate.

For example if we think that all parameters are the same between the models we can estimate:

\(y_{i,0}=\alpha + X_{i,0}\theta +\epsilon_{i,0}\)

\(y_{i,1}=\alpha + X_{i,1}\theta +\epsilon_{i,1}\)

Or:

\(y_{ij}=\alpha + X_{ij}\theta + \epsilon_{ij}\)

\subsubsection{Fixed slopes}

Intercepts may be different between the groups. In this case we can instead use the model:

\(y_{ij}=\alpha + X_{ij}\theta + \xi_j + \epsilon_{ij}\)

There are different ways of estimating this model:

\begin{itemize}
\item Pooled OLS
\item Fixed effects
\item Random effects
\end{itemize}

