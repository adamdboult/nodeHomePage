
\subsection{Parametric models}

\subsubsection{Introduction}

We have some data \(X\). We model it as a function of parameters, \(\theta \).

\(P(X|\theta )=f(\theta )\).

There is a data generating process. We use models to say things about the data generating process.

\subsubsection{Parameters}

Parameters could be arguments to some function, or moments.

\subsubsection{Distributions of \(\theta \)}

We can create a distribution of \(\theta \):

\(P(\theta | X)\).

So to create a model for the data we need:

\begin{itemize}
\item A model - \(P\)
\item Parameters for the model - \(\theta \)
\end{itemize}

There are therefore two questions: how to select a model and, given a model, how to select parameters.

