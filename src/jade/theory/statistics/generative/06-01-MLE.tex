
\subsection{Maximum likelihood estimation}

\subsubsection{Maximising the likelihood function}

We have a likelihood function of the data.

\(L(\theta ; X)=P(X|\theta )\)

We choose values for \(\theta \) which maximise the likelihood function.

\(argmax_\theta P(X|\theta )\)

That is, for which values  of \(\theta \) was the observation we saw most likely?

This is a mode estimate.

\subsubsection{IID}

\(L(\theta ; X)=\prod_i P(x_i|\theta )\)

\subsubsection{Logarithms}

We can take logarithms, which preserve stationary points. As logarithms are defined on all values above \(0\), and all probabilities are also above zero (or zero), this preserves solutions.

The non-zero stationary points of:

\(\ln L(\theta ; X)=\ln \prod_i P(x_i|\theta )\)

\(\ln L(\theta ; X)=\sum_i \ln P(x_i|\theta )\)

\subsubsection{Example: Coin flip}

Let’s take our simple example about coins. Heads and tails are the only options, so \(P(H)+P(T)=1\). 

\(P(H|\theta )=\theta \)

\(P(T|\theta )=1-\theta \)

\(\ln L(\theta ; X)=\sum_i \ln P(x_i|\theta )\)

If we had \(5\) heads and \(5\) tails we would have:

\(\ln L(\theta ; X)=5\ln (\theta )+ 5\ln (1-\theta )\)

So \(P(H)=\dfrac{1}{2}\) is the value which makes our observation most likely.

\subsubsection{Relative likelihood ratios}

A bit like confidence intervals
	
