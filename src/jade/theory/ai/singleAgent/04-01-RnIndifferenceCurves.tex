
\subsection{Indifference curves}

\subsubsection{Marginal rate of substitution}

We can consider how much of one good a consumer is willing to give up to get one of another.

Consider the utility function:

\(U=f(x,y)\)

We can examine the change in utility following changes in inputs by taking the total differential.

\(dU=\dfrac{\delta f}{\delta x}dx+\dfrac{\delta f}{\delta y}dy\)

We want to examine changes where \(dU\) is \(0\), so:

\( \dfrac{\delta f}{\delta x}dx+\dfrac{\delta f}{\delta y}dy=0\)

\( \dfrac{\delta f}{\delta x}dx+\dfrac{\delta f}{\delta y}dy=0\)

\( MU_xdx+MU_ydy=0\)

\( \dfrac{dy}{dx}=-\dfrac{MU_x}{MU_y}\)

This is the form of the indifference curve.  We refer to \(\dfrac{MU_x}{MU_y}\) as the marginal rate of substitution.

ADD GRAPH TO SHOW

\subsubsection{Multiple choices}

If the set of choices is more complex, say there are now apples and bananas, we have to be more careful with a representative function.

The agent still prefers more apples and more bananas, but the following imply different choices:

\(f=x^2.y^2\)

\(f=ln(x)+ln(y)\)

In the first example, the agent would always swap an apple for a banana, if they had more apples, whereas the opposite is true in the second case. Functional form is important with multiple goods.

\subsubsection{Solving}

