
\subsection{Choice set}

Economic agents face options from some set. This could be consumption choices, numbers of hours to work, or how much capital to invest in at a factory.

\subsubsection{Utility functions}

Calculating choices: Unrestricted choices

The choice of an agent is the selection which corresponds to the highest value of the utility function. Consider:

\(f=2(x-1)^2-10\)

We can easily calculate that even if the agent can choose any real number \(x\), they will chose \(1\).

This approach can be used if there are not meaningful constraints, or those constraints are implicit in the utility function. For example a firm can be modelled as profit maximising, where profit is a function of revenue and costs, with a single maximising value.

Other agents, such as a consumers, may instead have utility over consumption and leisure, and a separate constraint over this. This could be solved using simultaneous equations, but such an approach is not always desirable.

