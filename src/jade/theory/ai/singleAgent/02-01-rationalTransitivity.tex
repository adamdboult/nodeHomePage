
\subsection{Transitivity}

\subsubsection{Axiom 3: Transitivity}

If \(a\preceq b\), and \(b\preceq c\), then \(a\preceq c\).

For example an agent is offered apples or bananas, and prefers apples, and then bananas or carrots, and chooses bananas. If the agent is then offered apples or carrots, they axiom of transitivity says they chould choose apples.

This is not always observed. For example if a firm offers two products, a cheap \(a\) option and an expensive \(b\) option, then a preference may be:

\(a\prec b\)

The firm may add another product \(c\) slightly less expensive than \(b\), with fewer features, such if there were only \(2\) of the goods on sale:

\(a\prec b\)

\(a\prec c\)

\(b\prec c\)

In particular, the last two imply the first, through the axiom of transitivity.

However with all three, the marketing effectively makes the consumer choose \(b\), by making it look like a better deal, so we observe:

\(a\prec s\)

\(e\prec c\)

\(c\prec s\)

Even though the elements have not changed.

Axioms of revealed preference

Add weak axiom of revealed preference

