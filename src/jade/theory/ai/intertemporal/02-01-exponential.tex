
\subsection{Exponential discounting}

\subsubsection{Introduction}

We have:

\(U_T=\sum_[t=T]^{\infty }d_tU(x_t)\)

\subsubsection{Exponential discounting}

\(d_t=(1+\delta )^t\)

\(U_T=\sum_[t=T]^{\infty }(1+\delta )^tU(x_t)\)

\(\delta \) is the discount rate.

