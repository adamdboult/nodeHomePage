
\subsection{Alpha-beta pruning}

\subsubsection{Pruning}

Alpha-beta pruning

We can use pruning to reduce the number of nodes we search.

Alpha-beta pruning is a method for pruning. For each node we maintain two values:

\begin{itemize}
\item Alpha. The lower bound on the maximum value of the children.
\item Beta. The upper bound on the minimum value of the children.
\end{itemize}

Consider a player node with a score of \(2\), which has a neighbour. This neighbour has a 

We can know for sure we don't need to explore some paths. for example consider the mini player. there is a node we know the minimax value of a node is 2, and there is a neighbour with one child with a value of 3, then 
	

like DFS, but keep track of:

\begin{itemize}
\item alpha (largest value for max across viewed children)
\item initialise to  \(+/-\infty \)
\item initialise to \(\infty \)
\end{itemize}

Propagate \(\alpha \) and \(\beta \) down during search. prune where \(\alpha \ge \beta \)

Update \(\alpha \) and \(\beta \) by propagating upwards.

Only update alpha on max nodes, beta on min node


Ordering matters. if it's worst, then no pruning. want an ordering with lots of pruning
	p can:
	p + do shallow nodes
	p + order node so best are done first
	p + domain knowledge heuristics (chess: capture, threaten, forward, backwards)

In practice can get time down to \(O(b^\frac{m}{2})\).

Use heuristic. deep blue uses \(6000\)

