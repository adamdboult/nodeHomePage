
\subsection{Injective, bijective and surjective functions}

\subsubsection{Injective functions}

$f(a)=f(b)\rightarrow a=b$

\subsubsection{Surjective functions}

All points in codomain have at least one matching point in domain

Mapping info, details

\subsubsection{Bijective}

Both injective and surjective

\subsubsection{Other}

__Identity function__

The identity function maps a term to itself.

__Idempotent__

An idempotent function is a function which does not change the term if the function is used more than once. An example is multiplying by \(0\).

\subsubsection{Inverse functions}

An inverse function of a function is one which maps back onto the original value.

\(g(x)\) is an inverse function of \(f(x)\) if	

$g(f(x))=x$

\subsubsection{Properties of binary functions}

Binary functions can be written as:

$f(a,b)=a\oplus b$

A function is commutative if:

$x\oplus y = y\oplus x$

A function is associative if:

$(x\oplus y)\oplus z = x\oplus (y\oplus z)$

A function \(\otimes \) is left distributive over \(\oplus \) if:

$x\otimes (y\oplus z)=(x\otimes y) \oplus (x\otimes z)$

Alternatively, function \(\otimes \) is right distributive over \(\oplus \) if:

$(x\oplus y)\otimes z=(x\otimes z) \oplus (y\oplus z)$

A function is distributive over another function if it both left and right distributive over it.

