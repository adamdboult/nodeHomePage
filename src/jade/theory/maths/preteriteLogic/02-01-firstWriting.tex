
\subsection{Writing first-order logic}

\subsubsection{Existential quantifier}

We introduce a shorthand for “at least one term satisfies a predicate”, that is:

\(P(x_0)\lor P(x_1)\lor P(x_2)\lor P(x_2)\lor P(x_3)...\)

The short hand is:

\(\exists x P(x)\)

\subsubsection{niversal quantifier}

We introduce another shorthand, this time for:

\)P(x_0)\land P(x_1)\land P(x_2)\land P(x_2)\land P(x_3)...\)

The shorthand is

\(\forall x P(x)\)

\subsubsection{Free and bound variables}

A bound variable is one which is quantified in the formula. A free variable is one which is not. Consider:

\(\forall x P(x,y)\)

In this, \(x\) is bound while \(y\) is free.

Free variables can be interpreted differently, while bound variables cannot.

We can also bind a specific variable to a value. For example \(0\) can be defined to be bound.

\subsubsection{Ground terms}

A ground term does not contain any free variables. A ground formula is one which only includes ground terms.

\(\forall x\ x\) is a ground term.

\(\forall x P(x)\) is a ground formula.

