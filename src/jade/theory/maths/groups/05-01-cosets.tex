
\subsection{Cosets and normal subgroups}


A coset is defined between a group and a subgroup of the group.

For a group \(G\), and its subgroup \(H\):

\begin{itemize}
\item The left coset is \(\{gH\}\)
\item The right coset is \(\{Hg\}\)
\end{itemize}

For \(\forall g\in G\).

For abellian groups, the left and right cosets are the same.

The left and right cosets can also be the same, even if the group \(G\) is not abelian.

\subsubsection{Normal subgroups}

If the left and right cosets are the same then \(H\) is a normal subgroup.

\subsubsection{Cosets divide a group.}

Consider two left cosets, \(aH\) and \(bH\), with a common element.

This means that \(ah_i=bh_j\).

We can use this to get:

\(a=bh_jh_i^{-1}\)

\(b=ah_ih_j^{-1}\)

We know that:

\(ah\in aH\)

\(bh\in bH\)

So:

\(bh_jh_i^{-1}h\in aH\)

\(ah_ih_j^{-1}h\in bH\)

And so:

\(bH\subset aH\)

\(aH\subset bH\)

Therefore:

\(aH=bH\)

\subsubsection{Example 1}

Consider the group \(\{-1,1\},\times \)

For the subgroup \(\{1\},\times \), the left coset is \(\{gH\}=\{1,-1\}\).

The right coset is the same.

\subsubsection{Example 2}

Consider the group of integers and addition: \((Z,+)\)

For subgroup \((mZ,+)\), the left and right cosets are the same because the group is abelian.

The coset of the subgroup is the subgroup multiplied by each element in \(G\).

This is \(mZ\), \(mZ+1\), \(mZ+2\) and so on.

Once we reach \(mZ+m\) this has looped, and is already a coset, so we only need the sets upto \(mZ+m-1\).

