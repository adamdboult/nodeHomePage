
\subsection{Inverse matrices}

An invertible matrix implies that if the matrix is multiplied by another matrix, the original matrix can be recovered.

That is, if we have matrix \(A\), there exists matrix \(A^{-1}\) such that \(AA^{-1}=I\).

Consider a linear map on a vector space.

\(Ax=y\)

If \(A\) is invertible we can have:

\(A^{-1}Ax=A^{-1}y\)

\(x=A^{-1}y\)

If we set \(y=\mathbf 0\) then:

\(x=\mathbf 0\)

So if there is a non-zero vector \(x\) such that:

\(Ax=\mathbf 0\) then \(A\) is not invertible.

\subsection{Left and right inverses}

That is, for all matrices \(A\), the left and right inverses of \(B\), \(B_L^{-1}\) and \(B_R^{-1}\), are defined such that:

\(A(BB_R^{-1})=A\)

\(A(B_L^{-1}B)=A\)

Left and right inversions are equal

Note that if the left inverse exists then:

\(B_L^{-1}B=I\)

And if the right inverse exists:

\(BB_R^{-1}=I\)

Let’s take the first:

\(B_L^{-1}B=I\)

\(B_L^{-1}BB_L^{-1}=B_L^{-1}\)

\(B_L^{-1}BB_L^{-1}-B_L^{-1}=0\)

\(B_L^{-1}(BB_L^{-1}-I)=0\)

\subsection{Inversion of products}

\((AB)(AB)^{-1}=I\)

\(A^{-1}AB(AB)^{-1}=A^{-1}\)

\(B^{-1}B(AB)^{-1}=B^{-1}A^{-1}\)

\((AB)^{-1}=B^{-1}A^{-1}\)

\subsection{Inversion of a diagonal matrix}

\(DD^{-1}=I\)

\(D_{ii}D_{ii}^{-1}=1\)

\(D_{ii}^{-1}=\dfrac{1}{D_{ii}}\)

