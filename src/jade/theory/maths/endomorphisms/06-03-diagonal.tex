
\subsection{Diagonalisable matrices and eigendecomposition}

If matrix \(M\) is diagonalisable if there exists matrix \(P\) and diagonal matrix \(A\) such that:

\(M=P^{-1}AP\)

\subsubsection{Diagonalisiable matrices and powers}

If these exist then we can more easily work out matrix powers.

\(M^n=(P^{-1}AP)^n=P^{-1}A^nP\)

\(A^n\) is easy to calculate, as each entry in the diagonal taken to the power of \(n\).

\subsubsection{Defective matrices}

Defective matrices are those which cannot be diagonalised.

Non-singular matries can be defective or not defective, for example the identiy matrix.

Singular matrices can also be defective or not defective, for example the empty matrix.

\subsubsection{Eigen-decomposition}

Consider an eigenvector \(v\) and eigenvalue \(\lambda \) of matrix \(M\).

We known that \(Mv=\lambda v\).

If \(M\) is full rank then we can generalise for all eigenvectors and eigenvalues:

$MQ=Q\Lambda$

Where \(Q\) is the eigenvectors as columns, and \(\Lambda \) is a diagonal matrix with the corresponding eigenvalues. We can then show that:

\(M=Q\Lambda Q^{-1}\)

This is only possible to calculate if the matrix of eigenvectors is non-singular. Otherwise the matrix is defective.

If there are linearly dependent eigenvectors then we cannot use eigen-decomposition.

\subsection{Using the eigen-decomposition to invert a matrix}

This can be used to invert \(M\).

We know that:

$M^{-1}=(Q\Lambda Q^{-1})^{-1}$

$M^{-1}=Q^{-1}\Lambda^{-1}Q$

We know \(\Lambda \) can be easily inverted by taking the reciprocal of each diagonal element. We already know both \(Q\) and its inverse from the decomposition.

If any eigenvalues are \(0\) then \(\Lambda \) cannot be inverted. These are singular matrices.

