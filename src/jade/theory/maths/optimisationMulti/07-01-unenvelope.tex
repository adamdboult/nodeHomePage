
\subsection{Unconstrained envelope theorem}

Consider a function which takes two parameters:

\(f(x,\alpha \)

We want to choose \(x\) to maximise \(f\), given \(\alpha \).

\(V(\alpha )=\sup_{x\in X}f(x,\alpha )\)

There is a subset of \(X\) where \(f(x,\alpha )=V(\alpha )\).

\(X^*(\alpha )=\{x\in X|f(x, \alpha )=V(\alpha )\}\)

This means that \(V(\alpha )=f(x^*,\alpha )\) for \(x^*\in X^*\).

Let’s assume that there is only one \(x^*\).

\(V(\alpha )=f(x^*,\alpha )\)

What happens to the value function as we relax \(\alpha\)?

\(V_{\alpha_i}(\alpha )=f_{\alpha_i}(x^*(\alpha ),\alpha )\).

\(V_{\alpha_i}(\alpha )=f_x\dfrac{\delta x^*}{\delta \alpha }+f_{\alpha_i}\).

We know that \(f_x=0\) from first order conditions. So:

\(V_{\alpha_i}(\alpha )= f_{\alpha_i}\).

That is, at the optimum, as the constant is relaxed, we can treat the \(x^*\) as fixed, as the first-order movement is \(0\).

