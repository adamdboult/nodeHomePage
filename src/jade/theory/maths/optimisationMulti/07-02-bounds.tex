
\subsection{Identifying upper and lower bounds of linear programming}

In min/max problem, any feasibly solution is an upper/lower bound.

can we get a bound at the other side?
yes, by doing linear combinations of inequalities
eg maximise
\(30x + 100y\)
subject to:
\(4x + 10y <= 40\)
\(x >=3 \)

We can identify a lower bound by inputting something which works, for example \(x=3\) and \(y=0\). This gives us a lower bound of \(90\).

To get an upper bound we can manipulate the constraints:
\(40x + 100y<=400\)
\(10x>=30\)
And then:
\(40x+100y<=370+30\)
\(40x+100y<=370+10x\)
\(30x+100y<=370\)

So we have an upper bound of \(370\).

This lower bound is a result of doing linear combinations of the inequalities. For different combinations, we could have a lower lower bound.

This is the dual problem. How do we choose the linear combination of inequalities such that the resulting lower bound is minimised?

