
\subsection{Single equality constraint}

\subsubsection{Constrained optimisation}

Rather than maximise \(f(x)\), we want to maximise \(f(x)\) subject to \(g(x)=0\).

We write this, the Lagrangian, as:

\(\mathcal{L}(x,\lambda )=f(x)-\sum^m_k\lambda_k [g_k(x)-c_k]\)

We examine the stationary points for both vector \(x\) and \(\lambda \). By including the latter we ensure that these points are consistent with the constraints.

\subsubsection{Solving the Langrangian with one constraint}

Our function is:

\(\mathcal{L}(x, \lambda )=f(x)-\lambda [g(x)-c]\)

The first-order conditions are:

\(\mathcal{L}_{\lambda }= -[g(x)-c]\)

\(\mathcal{L}_{x_i}=\dfrac{\delta f}{\delta x_i}-\lambda \dfrac{\delta g}{\delta x_i}\)

The solution is stationary so:

\(\mathcal{L}_{x_i}=\dfrac{\delta f}{\delta x_i}-\lambda \dfrac{\delta g}{\delta x_i}=0\)

\(\lambda \dfrac{\delta g}{\delta x_i}=\dfrac{\delta f}{\delta x_i}\)

\(\lambda =\dfrac{\dfrac{\delta f}{\delta x_i}}{\dfrac{\delta g}{\delta x_i}}\)

Finally, we can use the following in practical applications.

\(\dfrac{\dfrac{\delta f}{\delta x_i}}{\dfrac{\delta g}{\delta x_i}}=\dfrac{\dfrac{\delta f}{\delta x_j}}{\dfrac{\delta g}{\delta x_j}}\)

