
\subsection{Continous functions}

A function is continuous if:

\(\lim_{x\rightarrow c} f(x)=f(c)\)

For example a function \(\frac{1}{x}\) is not continuous as the limit towards \(0\) is negative infinity. A function like \(y=x\) is continous.

More strictly, for any \(\epsilon >0\) there exists

\(\delta >0 \)

\(c-\delta < x< c +\delta \)

Such that

\(f(c)-\epsilon < f(x) < f(c)+\epsilon \)

This means that our function is continuous at our limit \(c\), if for any tiny range around \(f(c)\), that is \(f(c)-\epsilon\) and \(f(c)+\epsilon\), there is a range around \(c\), that is \(c-\delta \) and \(c+ \delta \) such that all the value of \(f(x)\) at all of these points is within the other range.

\subsubsection{Limits}

Why can't we use rationals for analysis?

If discontinous at not rational number, it can still be continous for all rationals.

Eg \(f(x)=-1\) unless \(x^2>2\), where \(f(x)=1\).

Continous for all rationals, because rationals dense in reals.

But can't be differentiated.

