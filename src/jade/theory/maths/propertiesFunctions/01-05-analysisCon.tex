
\subsection{Concave and convex functions}

\subsubsection{Convex functions}

A convex function is one where:

\(\forall x_1, x_2\in \mathbb{R} \forall t \in [0,1] [f(tx_1+(1-t)x_2 \le tf(x_1)+(1-t)f(x_2)]\)

That is, for any two points of a function, a line between the two points is above the curve.

A function is strictly convex if the line between two points is strictly above the curve:

\(\forall x_1, x_2\in \mathbb{R} \forall t \in (0,1) [f(tx_1+(1-t)x_2 < tf(x_1)+(1-t)f(x_2)]\)

An example is \(y=x^2\).

\subsubsection{Concave functions}

A concave function is an upside down convex function. The line between two points is below the curve.

\(\forall x_1, x_2\in \mathbb{R} \forall t \in [0,1] [f(tx_1+(1-t)x_2 \ge tf(x_1)+(1-t)f(x_2)]\)

A function is strictly concave if the line between two points is strictly below the curve:

\(\forall x_1, x_2\in \mathbb{R} \forall t \in (0,1) [f(tx_1+(1-t)x_2 > tf(x_1)+(1-t)f(x_2)]\)

An example is \(y=-x^2\).

\subsubsection{Affine functions}

If a function is both concave and convex, then the line between two points must be the function itself. This means the function is an affine function.

\(y=cx\)

