
\subsection{Polar co-ordinates}

\subsubsection{All complex numbers can be shown in polar form}

Consider a complex number

\(z=a+bi\)

We can write this as:

\(z=r\cos(\theta ) + ir\sin(\theta )\)

\subsubsection{Polar forms are not unique}

Because the functions loop:

\(ae^{i\theta }=a(\cos(\theta )+i\sin(\theta ))\)

\(ae^{i\theta }=a(\cos(\theta +n\tau )+i\sin(\theta +n\tau ))\)

\(ae^{i\theta }= ae^{i\theta +n\tau}\)

Additionally:

\(ae^{i\theta }=a(\cos(\theta )+i\sin(\theta ))\)

\(ae^{i\theta }=a(\cos(\theta )+i\sin(\theta ))\)

\(ae^{i\theta }=-a(\cos(\theta )-i\sin(\theta ))\)

\(ae^{i\theta }=-a(\cos(\theta +\dfrac{\pi }{2})+i\sin(\theta +\dfrac{\pi }{2}))\)

\subsubsection{Real and imaginary parts of a complex number in polar form}

We can extract the real and imaginary parts of this number.

\(Re(z):=r\cos (\theta )\)

\(Im(z):=r\sin (\theta )\)

Alternatively:

\(Re(z)=r\dfrac{e^{i\theta }+e^{-i\theta }}{2}\)

\(Im(z)=r\dfrac{e^{i\theta }-e^{-i\theta }}{2i}\)

