
\subsection{Moving between polar and cartesian coordinates}

All polar numbers can be shown as Cartesian

\(ae^{i\theta }=a(\cos(\theta )+i\sin(\theta ))\)

\(ae^{i\theta }=a\cos(\theta )+ia\sin(\theta )\)

\(z=a+bi\)

\(e^{i\theta }=\)

\(e^x=\sum^{\infty }_{i=0} \dfrac{x^i}{i!}\)

