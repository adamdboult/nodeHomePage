
\subsection{\(\tau \)}

\subsubsection{Calculating \(\tau \)}

As we note above, \(\sin (\theta )=\cos (\theta )\) at \(\theta =\tau *\frac{1}{8}\)

This is also where \(\tan (\theta )=1\).

$\arctan (k)=\arctan (a)+\int_a^k\frac{1}{1+y^2} \delta y$

We start from \(a=0\).

$\arctan (k)=\arctan (0)+\int_0^k\frac{1}{1+y^2} \delta y$

We know that one of the results for \(\arctan (0)\) is \(0\).

$\arctan (k)=\int_0^k\frac{1}{1+y^2} \delta y$

We want \(k=1\)

$\arctan (1)=\int_0^1\frac{1}{1+y^2} \delta y$

$\frac{\tau }{8}=\int_0^1\frac{1}{1+y^2} \delta y$

$\tau =8\int_0^1\frac{1}{1+y^2} \delta y$

We know that the \(\cos (\theta )\) and \(\sin (\theta )\) functions cycle with period \(\tau \).

Therefore \(cos (n.\tau )=\cos (0)\)

\subsubsection{Calculating \(\tau \)}

As we note above, \(\sin (\theta )=\cos (\theta )\) at \(\theta =\tau *\frac{1}{8}\)

This is also where \(\tan (\theta )=1\).

$\arctan (k)=\arctan (a)+\int_a^k\frac{1}{1+y^2} \delta y$

We start from \(a=0\).

$\arctan (k)=\arctan (0)+\int_0^k\frac{1}{1+y^2} \delta y$

We know that one of the results for \(\arctan (0)\) is \(0\).

$\arctan (k)=\int_0^k\frac{1}{1+y^2} \delta y$

We want \(k=1\)

$\arctan (1)=\int_0^1\frac{1}{1+y^2} \delta y$

$\frac{\tau }{8}=\int_0^1\frac{1}{1+y^2} \delta y$

$\tau =8\int_0^1\frac{1}{1+y^2} \delta y$

We know that the \(\cos (\theta )\) and \(\sin (\theta )\) functions cycle with period \(\tau \).

Therefore \(cos (n.\tau )=\cos (0)\)

