
\subsection{Calculus of sine and cosine}

\subsubsection{Unity}

Note that with imaginary numbers we can reverse all \(i\)s. So:

\(e^{i\theta }=\cos (\theta )+i\sin (\theta )\)

\(e^{-i\theta }=\cos (\theta )-i\sin (\theta )\)

\(e^{i\theta }e^{-i\theta }=(\cos (\theta )+i\sin (\theta ))(\cos (\theta )-i\sin (\theta ))\)

\(e^{i\theta }e^{-i\theta }=\cos (\theta )^2+\sin (\theta )^2\)

\(e^{i\theta }e^{-i\theta }=e^{i\theta -i\theta }=e^0=1\)

So:

\(\cos (\theta )^2+\sin (\theta )^2=1\)

Note that if \(\cos (\theta )^2=0\), then \(\sin (\theta )^2=\pm 1\)

That is, if the real part of \(e^{i\theta }\) is \(0\), the imaginary part is \(\pm 1\). And visa versa.

Similarly if the derivative of the real part of \(e^{i\theta }\) is \(0\), the imaginary part is \(\pm 1\). And visa versa.

\subsubsection{Sine and cosine are linked by their derivatives}

Note that these functions are linked in their derivatives.

\(\frac{\delta }{\delta \theta }\cos (\theta )=\sum_{j=0}^\infty \frac{(\theta )^{(4j+3)}}{(4j+3)!}-\sum_{j=0}^\infty \frac{(\theta )^{4j+1}}{(4j+1)!}\)

\(\frac{\delta }{\delta \theta }\cos (\theta )=-\sin (\theta )\)

Similarly:

\(\frac{\delta }{\delta \theta }\sin (\theta )=cos(\theta )\)

\subsubsection{Both sine and cosine oscillate}

\(\frac{\delta^2 }{\delta \theta^2}\sin (\theta )=-\sin (\theta )\)

\(\frac{\delta^2 }{\delta \theta^2}\cos (\theta )=-\cos (\theta )\)

So for either of:

\(y=\cos (\theta )\)

\(y=\sin (\theta )\)

We know that

\(\frac{\delta^2 }{\delta \theta^2}y(\theta )=-y(\theta )\)

Consider \(\theta =0\).

\(e^{i.0}=\cos (0)+i\sin (0)\)

\(1=\cos (0)+i\sin (0)\)

\(\sin (0)=0\)

\(\cos (0)=1\)

Similarly we know that the derivative:

\(\sin'(0)=\cos(0)=1\)

\(\cos'(0)=-\sin(0)=0\)

Consider \(\cos(\theta )\). 

As \(\cos (0)\) is static at \(\theta =0\), and is positive, it will fall until \(\cos (\theta )=0\).

While this is happening, \(\sin (\theta )\) is increasing. As:

\(\cos (\theta )^2+\sin (\theta )^2=1\)

\(\sin (\theta )\) will equal \(1\) where \(\cos (\theta )=0\).

Due to symmetry this will repeat \(4\) times.

Let's call the length of this period \(\tau \).

Where \(\theta =\tau *0\)

\begin{itemize}
\item \(\cos (\theta )=1\)
\item \(\sin (\theta )=0\)
\end{itemize}

Where \(\theta =\tau *\frac{1}{4}\)

\begin{itemize}
\item \(\cos (\theta )=0\)
\item \(\sin (\theta )=1\)
\end{itemize}

Where \(\theta =\tau *\frac{2}{4}\)

\begin{itemize}
\item \(\cos (\theta )=-1\)
\item \(\sin (\theta )=0\)
\end{itemize}

Where \(\theta =\tau *\frac{3}{4}\)

\begin{itemize}
\item \(\cos (\theta )=0\)
\item \(\sin (\theta )=-1\)
\end{itemize}

\subsubsection{Relationship between\( \cos (\theta )\) and \(\sin(\theta )\)}

Note that \(\sin(\theta + \frac{\tau }{4})=\cos(\theta )\)

Note that \(\sin (\theta )=\cos (\theta )\) at

\begin{itemize}
\item \(\tau *\frac{1}{8}\)
\item \(\tau *\frac{5}{8}\)
\end{itemize}

And that all these answers loop. That is, add any integer multiple of \(\tau \) to \(\theta \) and the results hold.

\(e^{i\theta } = e^{i\theta +n\tau }\)

\(n \in \mathbb{N}\)

\(e^{i\theta } = \cos(\theta )+i\sin(\theta )\)

\(e^{i\theta } = \cos(\theta +n\tau )+i\sin(\theta +n\tau ) \)

\(e^{i\theta } = e^{i(\theta +n\tau )}\)


\subsubsection{Calculus of trig}

Relationship between cos and sine

\(\sin(x+\frac{\pi }{2})=\cos(x)\)

\(\cos(x+\frac{\pi }{2})=-\sin(x)\)

\(\sin(x+\pi )=-\sin(x)\)

\(\cos(x+\pi )=-\cos(x)\)

\(\sin(x+\tau )=\sin(x)\)

\(\cos(x+\tau )=\cos(x)\)

