
\subsection{Set union and intersection}

We discuss functions. Just because we can write a function of sets which exist, does not mean the results of the functions exist. For that we need axioms discussed later.

\subsubsection{Union function}

We define a function on two sets, \(a\lor b\), such that the result contains all elements from either sets.

\(\forall a \forall x \forall y [a\in (x\lor y) \leftrightarrow (a\in x \lor a\in y)]\)

This is commutative: \(a\lor b = b\lor a\)

This is associative: \((a\lor b)\lor c = a\lor (b\lor c)\)

\subsubsection{Intersection function}

We define a function, \(a\land b\), on two sets, such that the result contains all elements which are in both.

\(\forall a \forall x \forall y [a\in (x\land y) \leftrightarrow (a\in x \land a\in y)]\)

This is commutative: \(a\land b = b\land a\)

This is associative: \((a\land b)\land c = a\lor (b\land c)\)

\subsubsection{Distribution of union and intersection}

Union is distributive over intersection: \(a\lor (b\land c)=(a\lor b)\land (a\lor c)\)

Intersection is distributive over union: \(a\land (b\lor c)=(a\land b)\lor (a\land c)\)

