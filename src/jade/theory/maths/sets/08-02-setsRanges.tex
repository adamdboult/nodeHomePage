
\subsection{Domains and ranges}

\subsubsection{Domain}

All values on which the function can be called

\(\forall x(f(x)=y)\rightarrow P(y))\)

\subsubsection{Image}

\(\forall x((\exists y f(x)=y)\rightarrow P(y))\)

Outputs of a function.

AKA: Range

The image of \(x\) is \(f(x)\).

\subsubsection{Preimage}

The preimage of \(y\) is all \(x\) where \(f(x)=y\).

\subsubsection{Codomain}

Sometimes the image is a subset of another set. For example a function may map onto natural numbers above \(0\). Natural numbers above \(0\) would be the image, and the natural numbers would be the codomain.

\subsubsection{Example}

\(f(n)=s(n)\)

Domain is: \(\mathbb{N}\)

Codomain is also: \(\mathbb{N}\)

Image is \(\mathbb{N}\land n\ne 0\)

\subsubsection{Describing  functions}

If function \(f\) maps from set \(X\) to set \(Y\) we can write this as:

\(f:X\rightarrow Y\)

