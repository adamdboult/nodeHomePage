
\subsection{Cardinality}

\subsubsection{Cardinality of finite sets}

The cardinality of a set \(s\) is shown as \(|s|\). It is the number of elements in the set. We define it formally below.

\subsubsection{Injectives, surjectives and bijectives}

Consider \(2\) sets. If there is an injective from \(a\) to \(b\) then for every element in \(a\) there is a unique element in \(b\).

If this injective exists then we say \(|a|\le |b|\).

Similarly, if there is a surjective, that is for every element in \(b\) there is a unique element in \(a\), then \(|a|\ge |b|\).

Therefore, if there is a bijection, \(|a|=|b|\), and if there is only an injective or a surjective then \(|a|<|b|\) or \(|a|>|b|\) respectively.

\subsubsection{Cardinality as a function}

Every set has a cardinality. As a result cardinality cannot be a well-defined function, for the same reason there is no set of all sets.

Cardinality functions can be defined on subsets.

