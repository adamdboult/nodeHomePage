
\subsection{Complements and disjoint sets}

\subsubsection{Disjoint sets}

Sets are disjoint is there is no overlap in their elements. Two sets are \(s_i\) and \(s_j\) are mutually exclusive if:

\(s_i\land s_j=\emptyset\)

A collection of events \(s\) are all mutually exclusive if all pairs are mutually exclusive. That is:

\(\forall s_i \in s\forall s_j\in s[s_i\land s_j\ne \emptyset \rightarrow s_i=s_j]\)

\subsubsection{Complement function}

\(x^C\) is the completement. It is defined such that:

$\forall x [x\land x^C=\varnothing ]$

For a set \(b\), the complement with respect to \(a\) is all elements in \(a\) which are not in \(b\).

\(\forall x \in a \forall y \in b [x \in (a \setminus b) \land y\in (a \setminus b)]\)

\(b\land (a \setminus b)= \varnothing \)

That is, \(b\) and \(a \setminus b\) are disjoint.

\subsubsection{Existence of the complement}

For two sets \(a\) and \(b\) we can write \((a \setminus b)\). This is the set of elements of \(a\) which are not in \(b\).

Consider the axiom of specification:

$\forall x \forall a \exists s[(P(x)\land x\in a )\leftrightarrow (x\in s)]$

We can also write

$\forall x \forall a \forall b\exists s[(x\not\in b\land x\in a )\leftrightarrow (x\in s)]$

Which provides the complement, \(s\).

