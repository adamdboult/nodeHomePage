
\subsection{Cantors theorem}

The cardinality of the powerset is strictly greater than the cardinality of the underlying set.

That is, \(|P(s)|<|s|\).

This applies to finite sets and infinite sets. In particular, this means that the powerset of the natural numbers is bigger than the natural numbers.

\subsubsection{Proof}

If one set is at least as big as another, then then is a surjection from that set to the other.

That is, if we can prove that there is no surjection from a set to its powerset, then we have proved the theorem.

We consider \(f(s)\). If there is a surjection, then for every subset of \(s\) there should be a mapping from \(s\) to that subset.

We take set \(s\) and have the powerset of this, \(P(s)\).

Consider the set:

\(A=\{x\in s|x\nin f(x)\}\)

That is, the set of all elements of \(s\) which do not map to the surjection.

