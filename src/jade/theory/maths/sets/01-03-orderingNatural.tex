
\subsection{Ordering}

\subsubsection{Ordering of the natural numbers}

For natural numbers we can say that number \(n\) preceeds number \(s(n)\). That is:

\(n\le s(n)\)

Similarly:

\(s(n)\le s(s(n))\)

From the transitive property we know that:

\(n\le s(s(n))\)

We can continue this to get:

\(n\le s(s(...s(n)..))\)

What can we say about an arbitary comparison?

\(a\le b\)

We know that either:

\begin{itemize}
\item \(a=b\)
\item \(b=s(s(...s(a)...))\)
\item \(a=s(s(...s(b)...))\)
\end{itemize}

In the first case the relation holds.

In the second case the relation holds.

In the third case the relation does not hold, but antisymmetry holds.

As this is is then defined on any pair, the order on natural numbers is total.

As there is a minimum, \(0\), the relation is also well-ordered.

However if this does not hold then the following instead holds:

