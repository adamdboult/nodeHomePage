
\subsection{Set-builder notation}

\subsubsection{Notation}

NB. This should be later, the way they are described links to the axiom schema of specification.

We can use short-hand to describe sets.

\(\{x\in S: P(x)\}\)

This defines a set by a restriction. For example we will later be able to define natural numbers above \(5\) as:

\(\{x\in \mathbb{N} : x>5\}\)

\subsubsection{Class builder notation}

Emuneration can be done through set builder notation too

Can define sets formally! defintion doesn't just affect sets

\(\forall x (x\in C \leftrightarrow P(x))\)

NB: We're not saying C exists

Can then use examples of equiv class

\(\forall x (x\in C \leftrightarrow x=x)\)

