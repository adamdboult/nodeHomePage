
\subsection{Membership relation}

Say we have a preterite \(P(x)\) which is true for some values of \(x\). Sets allow us to explore the properties of these values.

We may want to talk about a collection of terms for which \(P(x)\) is true, which we call a set.

To do this we need to introduce new axioms, however first we can add (conservative) definitions to help us do this.

We introduce a new relation: membership. If element \(x\) is in set \(s\) then the following relation is true, otherwise it is false:

\(x\in s\)

Sets are also terms. In first-order logic they will be included in quantifiers. Indeed, in set theory, we aim to treat everything as sets.

If a term is not a member of another term, we can write this using the non-member relation as follows:

\(\forall x \forall s [\neg (x\in s)\leftrightarrow x\not\in s]\)

