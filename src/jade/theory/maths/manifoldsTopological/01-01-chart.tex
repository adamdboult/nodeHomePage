
\subsection{Manifolds, charts and atlases}

A manifold is a set of points and associated charts.

A chart is a mapping from each point in a subset of the manifold to a point in a vector space.

These charts are invertible. If we are given coordinates, we can identify the point in the manifold it comes from.

For each point we have a topological neighbourhood. For each point in the neighbourhood, we can map to an element in the tangent space.

\subsubsection{Example: The sphere}

We can map a hemisphere to a subset of \(R^2\). Given a point in \(R^2\) we can identify a specific point on the hemisphere, and given a s specific point on the hemisphere we can identify a point in \(R^2\).

\subsubsection{Universal charts}

If the vector space is flat and non-repeating, then a single chart can be used to map the whole manifold.

\subsubsection{Atlases}

If we have a collection of charts which covers each point needs to be covered at least once, we have an atlas. Each chart needs to be to the same dimensional vector space.


