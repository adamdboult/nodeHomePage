
\subsection{Bezout's identity}

For any two non-zero natural numbers \(a\) and \(b\) we can select natural numbers \(x\) and \(y\) such that

\(ax+by=c\)

The value of \(c\) is always a multiple of the greatest common denominator of \(a\) and \(b\).

In addition, there exist \(x\) and \(y\) such that \(c\) is the greatest common denominator itself. This is the smallest positve value of c..


Let's take two numbers of the form \(ax+by\):

\(d=as+bt\)

\(n=ax+by\)

Where \(n>d\). And \(d\) is the smallest non-zero natural number form.

We know from Euclidian division above that for any numbers \(i\) and \(j\) there is the form \(i=jq+r\).

So there are values for \(q\) and \(r\) for \(n=dq+r\).

If \(r\) is always zero that means that all values of \(ax+by\) are multiples of the smallest value.

\(n=dq+r\) so \(r=n-dq\).

\(r=ax+by-(as+bt)q\)

\(r=a(x-sq)+b(y-tq)\)

This is also of the form \(ax+by\). Recall that \(r\) is the remainder for the division of \(d\) and \(n\), and that \(d=ax+by\) is the smallest positive value.

\(r\) cannot be above or equal to \(d\) due to the rules of euclidian division and so it must be \(0\).

As a result we know that all solutions to \(ax+by\) are multiples of the smallest value.

As every possible \(ax+by\) is a multiple of \(d\), \(d\) must be a common divisor to both numbers. This is because \(a.0+b.1\) and \(a.1+b.0\) are also solutions, and \(d\) is their divisor.

So we know that the smallest positive solution is a common mutliple of both numbers.

We now need to show that that \(d\) is the largest common denominator. Consider a common denominator \(c\).

\(a=pc\)

\(b=qc\)

And as before:

\(d=ax+by\)

So:

\(d=pcx+qcy\)

\(d=c(px+qy)\)

So \(d\ge c\)

