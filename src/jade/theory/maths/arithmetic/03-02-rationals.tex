\subsection{Rational numbers}


\subsubsection{Defining rational numbers}

We previously defined integers in terms of natural numbers. Similarly we can define rational numbers in terms of integers.

\(\forall ab \in \mathbb{I} (\neg (b=0)\rightarrow \exists c (b.c=a))\)

A rational is an ordered pair of integers.

\(\{\{a\},\{a,b\}\}\)

So that:

\(\{\{a\},\{a,b\}\}=\dfrac{a}{b}\)

\subsubsection{Converting integers to rational numbers}

Integers can be shown as rational numbers using:

\((i,1)\)

Integers can then be turned into rational numbers:

\(\mathbb{Q}=\dfrac{a}{1}\)

\(a=\dfrac{a_1}{a_2}\)

\(b=\dfrac{b_1}{b_2}\)

\(c=\dfrac{c_1}{c_2}\)

\subsubsection{Equivalence classes of rationals}

There are an infinite number of ways to write any rational number, as with integers. \(\dfrac{1}{2}\) can be written as \(\dfrac{1}{2}\), \(\dfrac{-2}{-4}\) etc.

The class of these terms form an equivalence class.

We can show these are equal:

\(\dfrac{a}{b}=\{\{a\},\{a,b\}\}\)

\(\dfrac{ca}{cb}=\{\{a\},\{a,b\}\}\)

\(\dfrac{ca}{cb}=\{\{ca\},\{ca,cb\}\}\)

\(\{\{a\},\{a,b\}\}=\{\{ca\},\{ca,cb\}\}\)

