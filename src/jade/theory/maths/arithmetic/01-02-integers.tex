\subsection{Integers}

\subsubsection{Defining integers}

To extend the number line to negative numbers, we define:

\(\forall ab \in \mathbb{N} \exists c (a+c=b)\)

For any pair of numbers there exists a terms which can be added to one to get the other.

For \(1+x=3\) this is another natural number, however for \(3+x=1\) there is no such number.

Integers are defined as the solutions for any pair of natural numbers.

There are an infinite number of ways to write any integer. \(-1\) can be written as \(0-1\), \(1-2\) etc.

The class of these terms form an equivalence class.

\subsubsection{Integers as ordered pairs}

Integers can be defined as an ordered pair of natural numbers, where the integer is valued at: \(a-b\).

For example \(-1\) could be shown as:

\(-1= \{ \{ 0 \},\{0,1\}\}\)

\(-1= \{ \{ 5 \},\{5,6\}\}\)

\((a,b)=a-b\)

\subsubsection{Converting natural numbers to integers}

Natural numbers can be shown as integers by using:

\((n,0)\)

Natural numbers can be converted to integers:

\(\{\{a\},\{a,0\}\}\)
\subsubsection{Cardinality of integers}

