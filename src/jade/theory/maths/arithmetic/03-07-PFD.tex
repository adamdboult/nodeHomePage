\subsection{Partial fraction decomposition}

We have:
\(\frac{1}{A.B}\)

We want this in the form of:

\(\frac{a}{A}+\frac{b}{B}\)

First, lets define \(M\) as the mean of these two numbers, and define \(\delta=M-B\). Then:

\(\frac{1}{AB}=\frac{1}{(M+\delta)(M-\delta)}=\frac{a}{M+\delta}+\frac{b}{M-\delta}\)

We can rearrange the latter two to find:

\(1=a(M-\delta)+b(M+\delta)\)

Now we need to find values of \(a\) and \(b\) to choose.

Let's examine \(a\).

\(a=\frac{1-b(M+\delta)}{M-\delta}\)

\(a=-\frac{bM+b\delta -1}{M-\delta}\)

\(a=-\frac{bM+b\delta -1}{M-\delta}\)

For this to divide neatly we need both the numerator to be a constant multiplier of the denominator. This means the ratio the multiplier for the left hand side of the denominator is equal to the right:

\(\frac{bM}{M}=\frac{b\delta -1}{-\delta}\)

\(b=\frac{b\delta -1}{-\delta}\)

\(b=\frac{1}{2\delta}\)

We can do the same for \(a\).

\(a=-\frac{1}{2\delta}\)

We can plug these back into our original formula:

\(\frac{1}{(M+\delta)(M-\delta)}=\frac{-\frac{1}{2\delta}}{M+\delta}+\frac{\frac{1}{2\delta}}{M-\delta}\)

\(\frac{1}{(M+\delta)(M-\delta)}=\frac{1}{2\delta}[\frac{1}{M-\delta}-\frac{1}{M+\delta}]\)

