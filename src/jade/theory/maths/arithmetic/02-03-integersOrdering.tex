\subsection{Ordering of the integers}

\subsubsection{Ordering integers}

Integers are an ordered pair of naturals.

\(\{\{x\},\{x,y\}\}\)

For example \(-4\) can be:

\(\{\{4\},\{4,8\}\}\)

\(\{\{0\},\{0,8\}\}\)

We extend the ordering to say:

\(\{\{x\},\{x,y\}\}\le \{\{s(x)\},\{s(x),y\}\}\)

\(\{\{x\},\{x,s(y)\}\}\le \{\{x\},\{x,y\}\}\)

So can we define this on an arbitrary pair:

\(\{\{a\},\{a,b\}\}\le \{\{c\},\{c,d\}\}\)

We know that:

\(\{\{a\},\{a,b\}\}=\{\{s(a)\},\{s(a),s(b)\}\}\)

And either of:

\(\{\{a\},\{a,b\}\}=\{\{0\},\{0,A\}\}\)

\(\{\{a\},\{a,b\}\}=\{\{B\},\{B,0\}\}\)

\(\{\{a\},\{a,b\}\}=\{\{0\},\{0,0\}\}\)

As the latter is a case of either of the other \(2\), we consider only the first \(2\).

So we can define:

\(\{\{a\},\{a,b\}\}\le \{\{c\},\{c,d\}\}\)

As any of:

\(1: \{\{0\},\{0,A\}\}\le \{\{0\},\{0,C\}\}\)

\(2: \{\{0\},\{0,A\}\}\le \{\{D\},\{D,0\}\}\)

\(3: \{\{B\},\{B,0\}\}\le \{\{0\},\{0,C\}\}\)

\(4: \{\{B\},\{B,0\}\}\le \{\{D\},\{D,0\}\}\)

Case 1:

\(\{\{0\},\{0,A\}\}\le \{\{0\},\{0,C\}\}\)

Trivial, depends on relative size of \(A\) and \(C\).

Case 2:

\(\{\{0\},\{0,A\}\}\le \{\{D\},\{D,0\}\}\)

We can see that:

\(\{\{D\},\{D,A\}\}\le \{\{D\},\{D,0\}\}\)

And therefore this holds.

Case 3:

\(\{\{B\},\{B,0\}\}\le \{\{0\},\{0,C\}\}\)

We can see that:

\(\{\{B\},\{B,0\}\}\le \{\{B\},\{B,C\}\}\)

And therefore this does not hold.

Case 4:

\(\{\{B\},\{B,0\}\}\le \{\{D\},\{D,0\}\}\)

Trivial, like case 1.

