
\subsection{Boolean algebra}

\subsubsection{Boolean algebra in propositional logic}

We previously discussed properties of normal form, and the results from these properties.

If another structure shares these properties then they will also share the results.

\subsubsection{Sets satisfy the definitions of a boolean algebra}

If a mathematical structure has the following properties, it shares the results from normal form, and is a boolean algebra.

\begin{itemize}
\item Both binary operators are commutitive - \(A\land B=B\land A\) and \(A\lor B=B\lor A\)
\item Both binary operators are associative - \((A\land B)\land C=A\land (B\land C)\) and \((A\lor B)\lor C=A\lor (B\lor C)\)
\item Completements - \(A\land \neg A=\emptyset \) and \(A\lor \neg A=U\)
\item Absorption - \(A\land (A\lor B)=A\) and \(A\lor (A\land B)=A\)
\item Identity - \(A\land U=A\) and \(A\lor \emptyset =A\)
\item Distributivity - \(A\land (B\lor C)=(A\land B)\lor (A\land C)\) and \(A\lor (B\land C)=(A\lor B)\land(A\lor C)\)
\end{itemize}

These hold for sets, and so boolean algebra holds for sets.

