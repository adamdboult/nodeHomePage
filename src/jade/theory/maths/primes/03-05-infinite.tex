
\subsection{Existence of an infinite number of prime numbers}

\subsubsection{Existence of an infinite number of prime numbers}

If there are a finite number of primes, we can call the set of primes \(P\).

We identify a new natural number \(a\) by taking the product of existing primes and adding \(1\).

\(a=1+\prod_{p\in P} p\)

From the fundamental theorem of arithmetic we know all numbers are primes or the products of primes.

If \(a\) is not a prime then it can be divided by one of the existing primes to form number \(n\):

\(\frac{\prod^n p_i +1}{p_j}=n\)

\(\frac{p_j \prod^n_{i\ne j} p_i +1}{p_j}=n\)

\(\prod^n_{i\ne j} p_i +\frac{1}{p_j}=n\)

As this is not a whole number, \(n\) must prime.

We can do this process for any finite number of primes, so there are an infinite number.

