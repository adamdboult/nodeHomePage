
\subsection{Inequality constraints}

linear programming means of the form
max c^Tx
st. Ax<=b
x>=0
this is the canonical form

\subsubsection{Lagrangians with inequality constraints}

We can add constraints to an optimisation problem. These constraints can be equality constraints or inequality constraints. We can write constrained optimisation problem as:

Minimise \(f(x)\) subject to 

\(g_i(x)\le 0\) for \(i=1,…,m\)

\(h_i(x)=0\) for \(i=1,…,p\)

We write the Lagrangian as:

\(\mathcal{L}(x, \lambda, \nu )=f(x)+\sum_{i=1}^m\lambda_i g_i(x)+\sum_{i=1}^p\nu_ih_i(x)\)

If we try and solve this like a standard Lagrangian, then all of the inequality constraints will instead by equality constraints.

\subsubsection{Affinity of the Lagrangian}

The Lagrangian function is affine with respect to \(\lambda \) and \(\nu \).

\(\mathcal{L}(x, \lambda, \nu )=f(x)+\sum_{i=1}^m\lambda_i g_i(x)+\sum_{i=1}^p\nu_ih_i(x)\)

\(\mathcal{L}_{\lambda_i}(x, \lambda, \nu )=g_i(x)\)

\(\mathcal{L}_{\nu_i}(x, \lambda, \nu )=h_i(x)\)

As the partial differential is constant, the partial differential is an affine function.

