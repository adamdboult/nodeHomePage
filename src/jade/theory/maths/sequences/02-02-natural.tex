
\subsection{Summation of natural numbers}


\subsubsection{Goal}

Let's prove that:

\(\sum_{i=0}^n i= \frac{n(n+1)}{2}\)

\subsubsection{Proof by induction}

We use the inference rules Modus Ponens, which says that if \(X\) is true, and \(X\rightarrow Y\) is true, then \(Y\) is true.

\subsubsection{True for \(n=0\)}

We know this is true for \(n=0\):

\(0=\frac{0(0+1)}{2}\)

\(0=0\)

\subsubsection{If it's true for \(n\), it's true for \(n+1\)}

We can also prove that if it true for \(n\), it is true for \(n+1\).

\(\sum_{i=0}^{n+1} i= \frac{(n+1)(n+2)}{2}\)

\((n+1)+\sum_{i=0}^{n} i= \frac{n^2 +3n +2}{2}\)

If it is true for \(n\), then:

\((n+1)+\frac{n(n+1)}{2}= \frac{n^2 +3n +2}{2}\)

\(\frac{n^2+3n+2}{2}= \frac{n^2 +3n +2}{2}\)

\(1=1\)

\subsubsection{Result}

So we know that it is true for \(n=0\), and if it is true for \(n\), then it is true for \(n+1\). As a result it is true for all natural numbers.

