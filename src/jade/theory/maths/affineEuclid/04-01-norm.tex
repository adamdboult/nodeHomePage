
\subsection{Norms}

We can use norms to denote the "length" of a single vector.

\(||v||=\sqrt {\langle v, v\rangle }\)

\(||v||=\sqrt {v^*Mv}\)

\subsubsection{Euclidian norm}

If \(M=I\) we have the Euclidian norm.

\(||v||=\sqrt{v^*v}\)

If we are using the real field this is:

\(||v||=\sqrt{\sum_{i=1}^{n}v^2_i}\)

\subsubsection{Pythagoras' theorem}

If \(n=2\) we have in the real field we have:

\(||v||=\sqrt{v_1^2+v_2^2}\)

We call the two inputs \(x\) and \(y\), and the length \(z\).

\(z=\sqrt {x^2+y^2}\)

\(z^2=x^2+y^2\)

