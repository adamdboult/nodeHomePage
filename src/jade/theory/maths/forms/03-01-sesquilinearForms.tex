
\subsection{Sesquilinear forms}

\subsubsection{Bilinear form recap}

A bilinear form takes two vectors and produces a scalar from the underyling field.

The function is linear in addition in both arguments.

\(\phi (au+x, bv+y)=\phi (au,bv)+\phi (au,y)+\phi (x,bv)+\phi (x,y)\)

The function is also linear in multiplication in both arguments.

\(\phi (au+x, bv+y)=ab\phi (u,v)+a\phi (u,y)+b\phi (x,v)+\phi (x,y)\)

They can be represented as:

\(\phi (u,v)=v^TMu\)

\subsubsection{Sesquilinear forms}

Like bilinear forms, sesquilinear are linear in addition:

\(\phi (au+x, bv+y)=\phi (au,bv)+\phi (au,y)+\phi (x,bv)+\phi (x,y)\)

Sesqulinear forms however are only multiplictively linear in the second argument.

\(\phi (au+x, bv+y)=b\phi (au,v)+\phi (au,y)+b\phi (x,v)+\phi (x,y)\)

In the first argument they are "twisted"

\(\phi (au+x, bv+y)=\bar ab\phi (u,v)+\bar a\phi (u,y)+b\phi (x,v)+\phi (x,y)\)

\subsubsection{The real field}

For the real field, \(\bar b = b\) and so the sesqulinear form is the same as the bilinear form.

\subsubsection{Representing sesquilinear forms}

We can show the sesquilinear form as \(v^*Mu\)

\subsubsection{Stuff}

$f(M)=f([v_1,v_2])$

We introduce \(e_i\), the element vector. This is \(0\) for all entries except for \(i\) where it is \(1\). Any vector can be shown as a sum of these vectors multiplied by a scalar.

$f(M)=f([\sum^m_{i=1}a_{1i}e_i,\sum^m_{i=1}a_{2i}e_i])$

$f(M)=\sum_{k=1}^mf([a_{1k}e_k,\sum^m_{i=1}a_{2i}e_i])$

$f(M)=\sum_{k=1}^m\sum^m_{i=1}f([a_{1k}e_k,a_{2i}e_i])$

Because this in linear in scalars:

$f(M)=\sum_{k=1}^m\sum^m_{i=1}a_{1k}^*a_{2i}f([e_k,e_i])$

$f(M)=\sum_{k=1}^m\sum^m_{i=1}a_{1k}^*a_{2i}e_k^*Me_i$

\subsubsection{Orthonormal basis and \(M=I\)}

$f(M)=\sum_{k=1}^m\sum^m_{i=1}a_{1k}^*a_{2i}e_k^*Me_i$

$f(M)=\sum_{k=1}^m\sum^m_{i=1}a_{1k}^*a_{2i}e_k^*e_i$

$f(M)=\sum_{k=1}^m\sum^m_{i=1}a_{1k}^*a_{2i}\delta_i^k$

$f(M)=\sum^m_{i=1}a_{1i}^*a_{2i}$

