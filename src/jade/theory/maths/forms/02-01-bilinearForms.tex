
\subsection{Bilinear forms}

A bilinear form takes two vectors and produces a scalar from the underyling field.

This is in contrast to a linear form, which only has one input.

In addition, the function is linear in both arguments.

\(\phi (au+x, bv+y)=\phi (au,bv)+\phi (au,y)+\phi (x,bv)+\phi (x,y)\)

\(\phi (au+x, bv+y)=ab\phi (u,v)+a\phi (u,y)+b\phi (x,v)+\phi (x,y)\)

\subsubsection{Representing bilinear forms}

They can be represented as:

\(\phi (u,v)=v^TMu\)

\(f(M)=f([v_1,v_2])\)

We introduce \(e_i\), the element vector. This is \(0\) for all entries except for \(i\) where it is \(1\). Any vector can be shown as a sum of these vectors multiplied by a scalar.

\(f(M)=f([\sum^m_{i=1}a_{1i}e_i,\sum^m_{i=1}a_{2i}e_i])\)

\(f(M)=\sum_{k=1}^mf([a_{1k}e_k,\sum^m_{i=1}a_{2i}e_i])\)

\(f(M)=\sum_{k=1}^m\sum^m_{i=1}f([a_{1k}e_k,a_{2i}e_i])\)

Because this in linear in scalars:

\(f(M)=\sum_{k=1}^m\sum^m_{i=1}a_{1k}a_{2i}f([e_k,e_i])\)

\(f(M)=\sum_{k=1}^m\sum^m_{i=1}a_{1k}a_{2i}e_k^TMe_i\)

\subsubsection{Orthonormal basis and \(M=I\)}

\(f(M)=\sum_{k=1}^m\sum^m_{i=1}a_{1k}a_{2i}e_k^Te_i\)

\(f(M)=\sum_{k=1}^m\sum^m_{i=1}a_{1k}a_{2i}\delta_i^k\)

\(f(M)=\sum^m_{i=1}a_{1i}a_{2i}\)

