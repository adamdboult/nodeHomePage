
\subsection{Determinants}


From invertible matrix section in endo


A matrix can only be inverted if it can be created from a combination of elementary row operations.

How can we identify if a matrix is invertible? We want to create a scalar from the matrix which tells us if this possible. We can this scalar the determinant.

For a matrix \(A\) we label the determinant \(|A|\), or \(\det A\)

We propose \(|A|=0\) when the matrix is not invertible.

So how can we identify the function we need to undertake on the matrix?

\subsubsection{New 1}

We know that linear dependence results in determinants of \(0\).

We can model this as a function on the columns of the matrix.

\(\det M = \det ([M_1, ...,M_n)\)

If there is linear depednence, for example if two columns are the same then:

\(\det ([M_1,...,M_i,...,M_i,...,M_n])=0\)

Similarly, if there is a column of \(0\) then the determinant is \(0\).

\(\det ([M_1,...,0,...,M_n])=0\)
\subsubsection{New 2}

Show linear in addition

How can we identify the determinant of less simple matrices? We can use the multilinear form.

\(\sum c_i\mathbf M_i=\mathbf 0\)

Where \(\mathbf c \ne \mathbf 0\)

Or:

\(M\mathbf c=\mathbf 0\)
\subsubsection{Rule 1: Columns of matrices can be the input to a multilinear form}

A matrix can be shown in terms of its columns.
\(A=[v_1,...,v_n]\)

\(\det A=\det [v_1,...,v_n]\)


\(\det A=\sum_{k_1=1}^m...\sum_{k_n=1}\prod_{i=1}^ma_{ik_i}\det ([e_{k_1},...,e_{k_n}])\)

\subsubsection{Multiplying a matrix by a constant multiplies the determinant by the same amount}

If a whole row or columns is \(0\) then:

\(\det A=\det [v_1,...,v_i,...,v_n]\)

\(\det A'=\det [v_1,...,cv_i,...,v_n]\)


\(\det A=\det [v_1,...,v_i,...,v_n]\)

\(\det A'=\det [v_1,...,cv_i,...,v_n]\)

\(\det A'=c\det [v_1,...,v_i,...,v_n]\)

\(\det A'=c\det A\)

As a result, multiplying a column by \(0\) makes the determinant \(0\).

A matrix with a column of \(0\) therefore has determinant \(0\)

\subsubsection{Rule 2: A matrix with equal columns has a determinant of \(0\).}

\(A=[a_1,...,a_i,...,a_i,...,a_n]\)

\(D(A)=D([a_1,...,a_i,...,a_i,...,a_n])\)

We know from Result 3 that swapping columns reverses the sign. Reversing columns results in the same matrix, so the determinant must be unchanged.

\(D(A)=-D(A)\)

\(D(A)=0\)

\subsubsection{Linear dependence}

If a column is a linear combination of other columns, then the matrix cannot be inverted.

\(A=[a_1,...,\sum_{j\ne i}^{n}c_ja_j,...,a_n]\)

\(\det A=\det ([v_1,...,\sum_{j\ne i}^{n}c_jv_j,...,v_n])\)

\(\det A=\sum_{j\ne i}^{n}c_j\det ([v_1,...,v_j,...,v_n])\)

\(\det A=\sum_{j\ne i}^{n}c_j\det ([v_1,...,v_j,,...,v_j,...,v_n])\)

As there is a repeating vector:

\(\det A=0\)

\subsubsection{Swapping columns multiplies the determinant by \(-1\)}

\(A=[v_1,...,v_i+v_j,...,v_i+v_j,...,v_n]\)

We know.

\(\det A=0\)

\(\det A=\det ([a_1,...,a_i,...,a_i,...,a_n])+\det([a_1,...,a_i,...,a_j,...,a_n])+\det([a_1,...,a_j,...,a_i,...,a_n])+\det([a_1,...,a_j,...,a_j,...,a_n])\)

So:

\(\det ([a_1,...,a_i,...,a_i,...,a_n])+\det ([a_1,...,a_i,...,a_j,...,a_n])+\det([a_1,...,a_j,...,a_i,...,a_n])+\det([a_1,...,a_j,...,a_j,...,a_n])=0\)

As \(2\) of these have equal columns these are equal to \(0\).

\(\det ([a_1,...,a_i,...,a_j,...,a_n])+\det ([a_1,...,a_j,...,a_i,...,a_n])=0\)

\(\det ([a_1,...,a_i,...,a_j,...,a_n])=-\det ([a_1,...,a_j,...,a_i,...,a_n])\)

\subsubsection{Calculating the determinant}

We have

\(\det A=\sum_{k_1=1}^m...\sum_{k_n=1}\prod_{i=1}^ma_{ik_i}\det ([e_{k_1},...,e_{k_n}])\)

So what is the value of the determinant here?

We know that the determinant of the identity matrix is \(1\).

We know that the determinant of a matrix with identical columns is \(0\).

We know that swapping columns multiplies the determinant by \(-1\).

Therefore the determinants where the values of \(k\) are not all unique are \(0\).

The determinants of the others are either \(-1\) or \(1\) depending on how many swaps are required to restore to the identity matrix.

This is also shown as the Leibni formula.

\(\det A = \sum_{\sigma \in S_n}sgn (\sigma )\prod_{i=1}^na_{i,\sigma_i}\)

