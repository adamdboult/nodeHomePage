\subsection{Clauses and horn clauses}

A clause is a disjunction of atomic formulae.

\(A\lor \neg B\lor C\)

This can be written in implicative form.

\((A\lor \neg B)\lor C\)

\(\neg (A\lor \neg B)\rightarrow C\)

\((\neg A\land B)\rightarrow C\)

A horn clause is a clause where there is at most one positive literal. This means the implicative takes the form.

\((A\land B\land C )\rightarrow X\).

\subsection{Inference with horn clauses}

If the horn clause is true, and so is the normal form part, then \(X\) is also true.

As all inference with horn clauses uses Modus Ponens, it is sound.

Inference with horn clauses is also complete.


