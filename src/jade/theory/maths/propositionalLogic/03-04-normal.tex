\subsection{Normal form}

This is where a formula is shown using only:	

\begin{itemize}
\item And / Conjunction- \(\land \)
\item Or / Disjunction   - \(\lor \)
\item Negation - \(\neg \)
\end{itemize}

The conjunctive normal form (CNF) is where a formula is converted to a normal form with the following layout:

$a \land b \land c \land d$

These letters can represent complex sub-formulae, in normal form.

Statements in this form are easier to evaluate, as each subformula can be evaluated separately. The statement is true only if all formulaes within are also true.

The disjunctive normal form (DNF) is similar for \(\lor \).

$a \lor b \lor c \lor d$

\subsection{Properties of the normal form}

The normal binary operators are commutitive - \(A\land B\Leftrightarrow B\land A\) and \(A\lor B\Leftrightarrow B\lor A\)

Both binary operators are associative - \((A\land B)\land C\Leftrightarrow A\land (B\land C)\) and \((A\lor B)\lor C\Leftrightarrow A\lor (B\lor C)\)

Negation is complementary.

\(A\land \neg A\Leftrightarrow F \)

\(A\lor \neg A\Leftrightarrow T\)

Normal binary operators are absorbative.

\(A\land (A\lor B)\Leftrightarrow A\)
\(A\lor (A\land B)\Leftrightarrow A\)

Identity.

\(A\land T\Leftrightarrow A\)

\(A\lor F \Leftrightarrow A\)

Distributivity.

\(A\land (B\lor C)\Leftrightarrow (A\land B)\lor (A\land C)\)

\(A\lor (B\land C)\Leftrightarrow (A\lor B)\land(A\lor C)\)		


