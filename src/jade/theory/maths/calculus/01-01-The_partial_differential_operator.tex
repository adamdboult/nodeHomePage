
\subsection{The partial differential operator}

\subsubsection{Differential}

When we change the value of an input to a function, we also change the output. We can examine these changes.

Consider the value of a function \(f(x)\) at points \(x_1\) and \(x_2\).

$y_1=f(x_1)$

$y_2=f(x_2)$

$y_2-y_1=f(x_2)-f(x_1)$

$\frac{y_2-y_1}{x_2-x_1}=\frac{f(x_2)-f(x_1)}{x_2-x_1}$

Let's define \(x_2\) in terms of its distance from x_1:

$x_2=x_1+\epsilon$

$\frac{y_2-y_1}{\epsilon }=\frac{f(x_1+\epsilon )-f(x_1)}{\epsilon }$

We define the differential of a function as:

$\frac{\delta y}{\delta x}=\lim_{\epsilon \rightarrow 0^+}\frac{f(x+\epsilon )-f(x)}{\epsilon }$

If this is defined, then we say the function is differentiable at that point.

\subsubsection{Differential operator}

