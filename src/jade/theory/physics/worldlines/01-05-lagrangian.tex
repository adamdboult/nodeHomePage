
\subsection{Lagrangians}

We have:

\(S = \int_a^b \sqrt {(\mathbf {\dot q})^T\mathbf M\mathbf {\dot q}}dt\)

We can define:

\(L=\sqrt {(\mathbf {\dot q})^T\mathbf M\mathbf {\dot q}}\)

So we have:

\(S=\int_a^b L dt\)

\subsection{Definiton 1: Lagrangians}

For each state we can calcualte a Lagrangian which is a function of the coordinates and their velocity.

\(L[q, q^i]\)

\subsection{Axiom 2: There exists a correct Lagrangian}

\subsection{Definition 2: Action}

We define the action of a length of time as:

\(A=\int_{t_0}^{t_1} L[q(t), q(t)^.]dt\)

\subsection{Axiom 3: Principle of stationary action}

\(\delta A=0\)

That is, the coordinates and their velocities are such that action is stationary.

\subsection{Theory 1: Euler-Lagrange}

We have \(q(t)\) which makes the action stationary. Consider adding proportion\(\epsilon \) of another function \(f(t)\) to \(q(t)\).

\(A’=\int_{t_0}^{t_1} L[q(t)+\epsilon f(t), q(t)^.+\epsilon f^’(t)]dt\)

\(\dfrac{A’-A}{\epsilon }=\dfrac{1}{\epsilon}\int_{t_0}^{t_1} L[q(t)+\epsilon f(t), q(t)^.+\epsilon f^’(t)]-L[q,q^.]dt\)

We can do a Taylor expansion of \(A’\).

\(A’=\int_{t_0}^{t_1} L[q(t)+\epsilon f(t), q(t)^.+\epsilon f^’(t)]dt\)

\(A’=\int_{t_0}^{t_1} L[q(t),q(t)^.]+\epsilon [f\dfrac{\delta L}{\delta q}+f^.\dfrac{\delta L}{\delta q^.}]+\epsilon^2 [...]dt\)

So:

\(\dfrac{A’-A}{\epsilon }=\dfrac{1}{\epsilon}\int_{t_0}^{t_1}L[q(t),q(t)^.]+\epsilon [f\dfrac{\delta L}{\delta q}+f^.\dfrac{\delta L}{\delta q^.}]+\epsilon^2 [...]-L[q,q^.]dt\)

\(\dfrac{A’-A}{\epsilon }=\int_{t_0}^{t_1}[f\dfrac{\delta L}{\delta q}+f^.\dfrac{\delta L}{\delta q^.}]+\epsilon [...]dt\)

We can now make the left side \(0\), by using the definition of stationary action.

\(\lim_{\epsilon \rightarrow 0} \dfrac{A’-A}{\epsilon }=\int_{t_0}^{t_1}[f\dfrac{\delta L}{\delta q}+f^.\dfrac{\delta L}{\delta q^.}]dt\)

\(\int_{t_0}^{t_1}[f\dfrac{\delta L}{\delta q}+f^.\dfrac{\delta L}{\delta q^.}]dt=0\)

\(\int_{t_0}^{t_1}[f\dfrac{\delta L}{\delta q}]dt +\int_{t_0}^{t_1}[f^.\dfrac{\delta L}{\delta q^.}]dt=0\)

Note that

\(\int_{t_0}^{t_1}[f^.\dfrac{\delta L}{\delta q^.}]dt=[f\dfrac{\delta L}{\delta q^.}]_{t_0}^{t_1}-\int_{t_0}^{t_1}f \dfrac{d}{dt}\dfrac{\delta L}{\delta q^.}dt\)

We assume that \(f(t_0)=f(t_1)=0\) and so:

\(\int_{t_0}^{t_1}[f^.\dfrac{\delta L}{\delta q^.}]dt=-\int_{t_0}^{t_1}f \dfrac{d}{dt}\dfrac{\delta L}{\delta q^.}dt\)

Plugging this back in we get:

\(\int_{t_0}^{t_1}[f\dfrac{\delta L}{\delta q}]-f \dfrac{d}{dt}\dfrac{\delta L}{\delta q^.}dt]=0\)

\(\int_{t_0}^{t_1}f[\dfrac{\delta L}{\delta q}]-\dfrac{d}{dt}\dfrac{\delta L}{\delta q^.}dt]=0\)

Since this applies to all possible functions we get:

\(\dfrac{\delta L}{\delta q}=\dfrac{d}{dt}\dfrac{\delta L}{\delta q^.}\)

\subsection{Definition: Momentum}

\(p=\dfrac{\delta L}{\delta q^.}\)

