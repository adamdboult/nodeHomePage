
\subsection{Defing sine and cosine using Euler's formula}

\subsubsection{Euler's formula}

Previously we showed that:

\(e^x=\sum_{i=0}^\infty \dfrac{x^i}{i!}\)

Consider:

\(e^{i\theta }\)

\(e^{i\theta }=\sum_{j=0}^\infty \dfrac{(i\theta )^j}{j!}\)

\(e^{i\theta }=[\sum_{j=0}^\infty \dfrac{(\theta )^{4j}}{(4j)!}-\sum_{j=0}^\infty \dfrac{(\theta )^{4j+2}}{(4j+2)!}]+i[\sum_{j=0}^\infty \dfrac{(\theta )^{4j+1}}{(4j+1)!}-\sum_{j=0}^\infty \dfrac{(\theta )^{4j+3}}{(4j+3)!}]\)

We then use this to define \(\sin \) and \(\cos \) functions.

\(\cos (\theta ):=\sum_{j=0}^\infty \dfrac{(\theta )^{4j}}{(4j)!}-\sum_{j=0}^\infty \dfrac{(\theta )^{4j+2}}{(4j+2)!}\)

\(\sin (\theta ):=\sum_{j=0}^\infty \dfrac{(\theta )^{4j+1}}{(4j+1)!}-\sum_{j=0}^\infty \dfrac{(\theta )^{4j+3}}{(4j+3)!}\)

So:

\(e^{i\theta }=\cos (\theta )+i\sin (\theta )\)

\subsubsection{Alternative formulae for sine and cosine}

We know

\(e^{i\theta }=\cos (\theta )+i\sin (\theta )\)

\(e^{-i\theta }=\cos (\theta )-i\sin (\theta )\)

So

\(e^{i\theta }+e^{-i\theta }=cos (\theta )+i\sin (\theta )+\cos (\theta )-i\sin (\theta )\)

\(\cos (\theta )=\dfrac{e^{i\theta }+e^{-i\theta }}{2}\)

And

\(e^{i\theta }-e^{-i\theta }=cos (\theta )+i\sin (\theta )-\cos (\theta )+i\sin (\theta )\)

\(\sin (\theta )=\dfrac{e^{i\theta }-e^{-i\theta }}{2i}\)

\subsubsection{Sine and cosine are odd and even functions}

Sine is an odd function.

\(\sin (-\theta )=-\sin (\theta )\)

Cosine is an even function.

\(\cos (-\theta )=\cos (\theta )\)

