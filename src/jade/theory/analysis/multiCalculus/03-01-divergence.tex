
\subsection{Divergence}

This takes a vector field and produces a scalar field.

It is the dot product of the vector field with the del operator.

\(div F = \nabla . F\)

Where \(\nabla =(\sum_{i=1}^n e_i\dfrac{\delta }{\delta x_i})\)

\(div F = \sum_{i=1}^n e_i\dfrac{\delta F_i}{\delta x_i}\)

\subsection{Divergence as net flow}

Divergence can be thought of as the net flow into a point.

For example, if we have a body of water, and a vector field as the velocity at any given point, then the divergence is \(0\) at all points.

This is because water is incompressible, and so there can be no net flows.

Areas which flow out are sources, while areas that flow inwards are sinks.

\subsection{Solenoidal vector fields}

If there is no divergence, then the vector field is called solenoidal.

\subsection{The Laplace operator}

Cross product of divergence with the gradient of the function.

\(\Delta f= \nabla . \nabla f\)

\(\Delta f= \sum_{i=1}^n \dfrac{\delta^2 f}{\delta x^2_i}\)

