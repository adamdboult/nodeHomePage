

\subsection{Defining arrays}

\begin{verbatim}
int vals[] = {1, 2, 3, 4, 5};
\end{verbatim}

Can also define empty array:

\subsection{Accessing values in arrays}

The following are the same. Difference is just syntactic sugar.

\begin{verbatim}
a[i]
*(a+i)
\end{verbatim}

\subsection{Array shifting}

Relevant for insertion in place.

\subsection{sizeof()}

This function gets the length of an array. It is determined at compile time.

\subsection{Buffer overflows}


\subsection{Multi-dimensional arrays}

The following are the same. The difference is just syntactic sugar.

\begin{verbatim}
a[i][j]
??
\end{verbatim}

(Doesn't this need to know dimensions of all but last one? How does this work if size not known in eg a function?)

\subsection{Sort}

Regular arrays. If want to insert or remove can create new array with new size.
Traversal of array is O(1)

array slices etc operations

c increase size of array. automatically creates array twice as big?? how does this work with stack??

size of array unknown at runtime unless provided


arrays if you have array of length 4 and you look at 5 what happens? Does compiler prevent? What about if just have pointer to array?


