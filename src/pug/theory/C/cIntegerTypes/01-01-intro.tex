
\subsection{Integer types in C}

\subsection{Standard (signed) integers}

Following are equivalent.

Signed means includes +ve and -ve numbers.

These are at least 16-bit (ie could be 32, 64, 16, or something else - up to the implementaion)

In 16 bit this goes between -32,767 and 32,767.

\begin{verbatim}
int a = 1;
signed int a = 1;
signed a = 1;
\end{verbatim}

\subsection{Unsigned integers}

If these are 16-bit these are between \(0\) and \(65,535\).
 
\begin{verbatim}
unsigned int a = 1;
unsigned a = 1;
\end{verbatim}

\subsection{Short integers}

These are at least 16 bit, and equal or lesser than standard integers in their bits.

\begin{verbatim}
short int a = 1;
signed short int a = 1;
signed short a = 1;
short a = 1;
\end{verbatim}

Unsigned versions.

\begin{verbatim}
unsigned short int a = 1;
unsigned short a = 1;
\end{verbatim}

\subsection{32-bit}

Guaranteed to be at least as big as int, and at least 32 bit.

\begin{verbatim}
long int a = 1;
signed long int a = 1;
signed long a = 1;
long a = 1;
\end{verbatim}

Unsigned versions:

\begin{verbatim}
unsigned long int a = 1;
unsigned long a = 1;
\end{verbatim}


\subsection{64-bit}

Guaranteed to be at least as big as long, and at least 64 bit.

\begin{verbatim}
long long int a = 1;
signed long long int a = 1;
signed long long a = 1;
long long a = 1;
\end{verbatim}

Unsigned versions:

\begin{verbatim}
unsigned long long int a = 1;
unsigned long long a = 1;
\end{verbatim}

\subsection{char}

At least 8 bit.

char can be unsigned or signed.
\begin{verbatim}
char a = 1;
\end{verbatim}


If signed is between -127 and 128 (if 8 bit)
\begin{verbatim}
signed char a = 1;
\end{verbatim}

If unsigned is between 0 and 255 (if 8 bit)
\begin{verbatim}
unsigned char a = 1;
\end{verbatim}

\subsection{Using American Standard Code for Information Interchange (ASCII)}

\begin{verbatim}
char a = "a";
\end{verbatim}


\subsection{Casting}

\begin{verbatim}
unsigned short int a = 1;
unsigned long long b = (long long) a;
\end{verbatim}


