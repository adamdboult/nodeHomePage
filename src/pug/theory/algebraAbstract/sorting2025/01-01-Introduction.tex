
\subsection{Groups}
Constructing \(d_6\) from \(c_6\) and ??

\(S^1\), \(SO(2)\) and \(U(1)\) all circles


maybe h3 on more on groups after fields for things like general linear, orthogonal uniarty, special linear etc?
+ so these are currently discussed in abstract linear algebra, in part because they are defined on an arbitrary field
+ could define in groups when over specifically R or C

algebra: groups: discrete logarithm (which is defined for prime modular arithmetic, finite groups)
+ if b in group, \(b^k\) defined for all k integers
+ \(a = b^k\)
if we know a and b, finding k is the discrete logarithm problem

Cyclic and alternating groups are the fundamental finite ones of interest

Countable infinite group: addition and z. Uncountable: lie group.
H3 on finite between. Finite should include isomorphism to complex numbers with length 1 and a generator

\subsection{Fields}
in fields
	pag eon field extensions and galois theory

GPT/internet. finite fields. there's a section already but:
+ why does p have to be prime?
+ is this definitely a field? can't divide right? or can I?
finite field stuff:
+ as with all fields:
+ addition and multiplication are commutative and associative.
+ identity eleemnts for addition and multipcation exist.
+ multiuplicaton distributes over addition
+ defining division:
+ for any nonzero element a, there exists another unique element \(a^-1\) such \(a.a^-1 = 1 mod p\)
+ division is multiplication but multiplicative inverse
+ divison by zero not defined, as in all fields.
in all integers mod n (aka Z/nZ, fancy Z), subtraction is uniqueless defined.
but for multiplication we need \(a.a^-1=1\)
\(a^-1\) doesn't always exist unless n is prime. (can sometimes exists, eg \(a=1\), \(a^-1\) is \(1\))
+ why is n being prime sufficient?
+ a and n are coprime: GCD is 1 because n is prime and \(a < n\)
+ bezout identity: exist x and y such \(ax+ny=1\)
+ reduce both sides modulo n:
+ \(ax+ny=1 mod n\)
+ \(ax = 1 mod n\) (because ny is multiple of n)
+ therefore x exists
is x unique?
+ \(ax=1 mod n\)
+ \(ay=1 mod n\)
+ \(ax-ay=1-1 mod n\)
+ \(a(x-y)= 0 mod n\)
+ so n divides \(a(x-y)\), but n prime so
+ \(x-y = 0 mod n\)
+ \(x=y mod n\)
+ because x, \(y < n\), both are the same.
how do we know n = p is necessary?
+ if \(n = a.b\)
+ \(ac=1 mod n\)
+ \(ac=1 mod a.b\)
+ this is a contradictin

group theory
	split out
	page on Z int as group and isomorphism
		iso symbo: $\cong$

	page on modular arithmetic as group
	page on continous group, inc lie, weyl groups

	page on direct product and semi direct product
		direct product:
			\(\times\)
	normal subgroup
		X is subgroup of Y: \(X <= Y\)
			all X in Y, X forms group
		normal subgroup: \(X\unlhd Y\).
			\(yxy^{-1} \in X\)
		every subgroup of abelian group is normal
	page on cosets in groups
		right and left coset
		can be used to decompose groups
		finding cosets
			take a subgroup H of G
			take g in G
			gH forms a left coset
			Hg forms a right coset
			does the coset vary by g?
				left(right) cosets are either disjoint or identical
		if normal, then left and right cosets are same for all g
	quotient group:
		X/Y
		divide a group X by a subgroup of X (Y has to be a (normal?) subgroup of X)
		\(G/N = {aN:a\in G}\)
\subsection{Vector space}

tensors split to "einstein notation and writing homorphisms as tensors" then "third order and higher tensors"
+ 3+ tensor bit to later h3, with exterior product etc too
Polynomial long division on algebra

	spin group when discussing \(SO(n)\)
	group automorphisms
		if do function on each element in group, original group relations still hold
		for abelian, inverse is an automorphisms. Trivial automorphism exists for all
		for \(z_p\) more automorphisms exist
		automorphism group is the group of all automorphisms on a group
			\(Aut(G)\)
		holomorph: a group which contains the group and automorphism group
			\(Hol(G)=G x| Aut(G)\) ["x|" should be the semidirect product symbol]
in vector space stuff, page on clifford algebra


