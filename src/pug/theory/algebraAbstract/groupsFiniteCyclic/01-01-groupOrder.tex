
\subsection{Group order}

For finite groups, each element \(e\) has:

\(e^n=I\)

For some \(n\in \mathbb{N}\)

Where \(I\) is the identity element.

The order of the group is the smallest value of \(n\) such that that holds for all elements.

For example in the multiplicative group \(G=\{-1,1\}\) the order is \(2\).

Or:

\(|G|=2\)

Additionally

\(|-1|=2\)

\(|1|=1\)


\subsection{Finite groups}

Consider the set of natural numbers and addition modulo 4. This forms a group containing:

\(\{0,1,2,3\}\)

This can be written as \(Z_4\) or more generally as \(Z_n\), or \(Z/nZ\).

