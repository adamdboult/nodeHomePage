
\subsection{The orthogonal projection}

in inner product space, orthogonal projection

\(p_uv = \dfrac{<u,v>}{{v,v}}v\)

we then know that \(o=v-p_uv\) is orthogonal to \(u\).


\subsection{Orthogonal set / Orthogonal basis}

if a set of vectors are all orthogonal, they form an orthogonal set
if the set spans the vector space, it is an orthogonal basis.

\subsection{The Gram-Schmidt process}

can we form an orthogonal basis from a non-orthogonal basis? yes, using gram schmidt

we have \(x_1\), \(x_2\) \(x_3\) etc
we want to make \(v_1\), \(v_2\) etc orthognal

\(v_1 = x_1\)
\(v_2 = x_2 - p_{x_2}v_1\)
\(v_3 = x_3 - p_{x_3}v_1 - p_{x_3}v_2v\)

