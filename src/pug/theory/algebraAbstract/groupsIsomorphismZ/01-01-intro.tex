
\subsection{Homomorphism}

Homomorphisms are functions which preserve the relationships between members of a set, and specified functions.

That is, if:

\(a\odot b=c\)

Then \(f(x)\) is morphism if:

\(f(a)\odot f(b)=f(a\odot b)\)

Here we discuss morphisms in the context of groups, but we can define morphisms for sets with more than one function, for example with addition and multiplication.

Morphisms are also known as homomorphisms.

The following are morphisms of the additive group of integers.

Where we refer to \(c\), \(c\ne 0\in \mathbb{I}\).

\begin{itemize}
\item \(f(x)=0\)
\item \(f(x)=x\)
\item \(f(x)=cx\)
\item Converting natural numbers to integers
\end{itemize}

The following are not morphisms

\begin{itemize}
\item \(f(x)=x+1\)
\end{itemize}

\subsection{Isomorphism}

An isomorphism is a morphism which has an inverse.

This means the function is bijective.

The following are isomorphisms:

\begin{itemize}
\item \(f(x)=x\)
\item \(f(x)=cx\)
\item Converting natural numbers to integers
\end{itemize}

The following are not isomorphisms

\begin{itemize}
\item \(f(x)=0\)
\item \(f(x)=x+1\)
\end{itemize}

