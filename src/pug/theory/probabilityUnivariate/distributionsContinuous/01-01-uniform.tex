
\subsection{Uniform distribution}

There is a set \(s\) such that:

\(P(x\in s)=p\)

\(P(x\not\in s)=0\)

\subsubsection{Moments of the uniform distribution}

The mean is the mean of the set \(s\).

If the set is all numbers of the real line between two values, \(a\) and \(b\), then:

The mean is \(\dfrac{1}{2}(a+b)\).

The variance is \(\dfrac{(b-a)^2}{12}\) in the continuous case.

The variance is \(\dfrac{(b-a+1)^2-1}{12}\) in the discrete case.

