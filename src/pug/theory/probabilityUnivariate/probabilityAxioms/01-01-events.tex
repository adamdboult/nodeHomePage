
\subsection{Elementary events}

We have a sample space, \(\Omega \) consisting of elementary events.

All elementary events are disjoint sets.

\subsection{Non-elementary events}

We have a \(\sigma\)-algebra over \(\Omega \) called \(F\). A \(\sigma\)-algebra takes a set a provides another set containing subsets closed under complement. The power set is an example.

All events \(E\) are subsets of \(\Omega\)

\(\forall E\in F E\subseteq \Omega\)

\subsection{Mutually exclusive events}

Events are mutually exclusive if they are disjoint sets.

\subsection{Complements}

For each event \(E\), there is a complementary event \(E^C\) such that:

$E\lor E^C=\Omega$

$E\land E^C=\varnothing$

This exists by construction in the measure space.

\subsection{Union and intersection}

As events are sets, we can define algebra on sets. For example for two events \(E_i\) and \(E_j\) we can define:

\begin{itemize}
\item \(E_i\land E_j\)
\item \(E_i\lor E_j\)
\end{itemize}

