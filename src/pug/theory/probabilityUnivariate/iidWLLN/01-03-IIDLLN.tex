
\subsection{Weak law of large numbers}

The sample mean is:

\(\bar X_n=\dfrac{1}{n}\sum_{i=1}^nX_i\)

The variance of this is:

\(Var[\bar X_n]=Var[\dfrac{1}{n}\sum_{i=1}^nX_i]\)

\(Var[\bar X_n]=\dfrac{1}{n^2}nVar[X]\)

\(Var[\bar X_n]=\dfrac{\sigma^2}{n} \)

We know from Chebyshev’s inequality:

\(P(|X-\mu | \ge k\sigma )\le \dfrac{1}{k^2}\)

Use \(\bar X_n\) as \(X\):

\(P(|\bar X_n-\mu | \ge \dfrac{k\sigma }{\sqrt n})\le \dfrac{1}{k^2}\)

Update \(k\) so \(k:=\dfrac{k\sqrt n}{\sigma}\)

\(P(|\bar X_n-\mu | \ge k)\le \dfrac{\sigma^2}{nk^2}\)

As \(n\) increases, the chance that the sample mean lies outside a given distance from the population mean approaches \(0\).

