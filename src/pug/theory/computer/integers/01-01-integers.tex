
\subsection{Representing integers}

We can expand the natural numbers to the integers.

Consider a byte representing the natural numbers. Previously this would have gone from \(0\) to \(255\), with a series of all \(1\)s representing \(255\).

To introduce integers all numbers with a \(1\) in the leftmost bit are considered to be negative.


\subsection{Signed magnitude}

You can use the first bit as the sign. eg \(-1\) is represented as \(1001\).

The downside to this is that arithmetic operations for integers don't work.

\subsection{Representing using one's complement}

If \(1\) is \(0001\), \(-1\) is \(1110\).

One's complement refers to the NOT operation on the binary number.

The advantage of this approach is that it works well with adders.

Downsides are that negative zero exists (\(0000\) and \(1111\) are both \(0\)).

In addition logic needs to deal with end-around carry.

\subsection{Two's complement}

We can represent the value of negative numbers with two’s complement. 

In one's complement, the inverse of a number is the NOT operation on the binary number.

For two's complement, the inverse is the number which means \(x + inverse(x) = 2^n\) where \(n\) is the number of digits.

So we have \(0011\) as \(3\), the inverse is \(1101\) because \(3+13=2^4=16\).

The use of two’s complement allows us to use the arithmetical logical units for the integers.

