
\subsection{Using ALUs with integers}

We can expand the natural numbers to the integers.

Consider a byte representing the natural numbers. Previously this would have gone from \(0\) to \(255\), with a series of all \(1\)s representing \(255\).

To introduce integers all numbers with a \(1\) in the leftmost bit are considered to be negative.

\subsection{Two's complement}

We represent the value of these negative numbers with two’s complement. With two’s complement the number “after” \(127\) is \(-128\). Note that this does not just use the first bit as a sign. The use of two’s complement allows us to use the arithmetical logical units for the integers.

