
\subsection{Introduction}

Similar to counter machine but:

+ Allows indirect reference of registers. Where \(r\) previously would have referred to the register itself, (eg \(INC(r)\), we can now refer to the value of a pointer using \([r]\). This allows us to write eg \([3]\rightarrow 4\) which means put the value of register \(3\) into register \(4\).

Increments can be rewritten

INC(r) to \([r]+1\rightarrow r\)
DEC(r) to \([r]-1\rightarrow r\)

Also for the changes to the instruction register

normally is \([IR]+1\rightarrow IR\)
with jump is before JZ(r,z). now if \([r]=0\) then \(z\rightarrow IR\). if \([r]!=0\) then \([IR]+1\rightarrow IR\)

The register machine introduces additional operations, taking advantage of indirect operations.
