
\subsection{Assigning literals to variables}

Variables rather than addresses
within a scope memory location is equivalent to variable: in high level and variables


We can assign a value to a variable by eg:

\begin{verbatim}
int a = 10;
\end{verbatim}

This does two different things. First it declares the variable \(a\). This assigns part of the memory for the variable.
Secondly it defines the value of the memory represented by the variable to \(10\).

These can be split out as follows.

\begin{verbatim}
int a;
a = 10;
\end{verbatim}

If the variable is declared before it is defined it is an uninitialised variable, and its value is undefined.

In addition to decimal we can also set values using other literals.

\begin{verbatim}
int a = 10;
int b = 010;
int c = 0x10;
\end{verbatim}

Here \(b\) is octal because of the first \(0\) in the literal. In decimal it is 8.
\(c\) is hexidecimal because of the first \(0x\) in the literal. In decimal it is 16.




\subsection{Assigning variables from other variables}

We can also assign variables from other variables.

\begin{verbatim}
int a = 10;
int b = a;
\end{verbatim}

The following is valid syntax but unlikely to be what was intended.
\begin{verbatim}
int a = 10;
int b;
int c = b = a;
\end{verbatim}


\subsection{Note on lvalues}

lvalue in c
identifiable location in memory. not a constant. not a function. not a literal, not a calculation  eg (a+b)

left has to be lvalue. right can be lvalue or not
int a = 1; // OK. a is an lvalue. doesn't matter what right is.
int b = a; // OK. b is an lvalue. right is also an lvalue, which is ok.
(a+b) = 5; // not OK
5 = a; // Not OK
left hand side has to be an lvalue. has to be an address we can set result of right hand side to.

