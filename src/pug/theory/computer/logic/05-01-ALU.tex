
\subsection{Half adder}

We first introduce the half adder.

Take two inputs \(A\) and \(B\) and put both inputs to two binary operators.

The operators are an XOR gate and an AND gate.

The AND gate only returns \(1\) if both inputs are \(1\). The XOR gate returns \(1\) if only one input is \(1\).

The XOR gate returns the second “digit” while the AND date returns the first.

This means we take two number at either \(0\) or \(1\), and return beteen \(0\) and \(2\) in binary form.

\subsection{Full adder}

If we are adding, say, a 2 bit number, then we need the abilty to carry numbers. The full adder takes two numbers, and also a carry from the previous digit. The full adder then adds these three numbers.

The carry from each full adder (or half adder) is then passed to the next digit.

If the carry is \(1\) for the final digit then there has been an overflow.

\subsection{\(N\)-bit ALU}

An \(8\)-bit ALU processes numbers represented by \(8\) bits. Similar for \(16\)-, \(32\)- and \(64\)-bit ALUs.
