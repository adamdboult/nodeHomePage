
\subsection{The bit}

A single bit can store a binary piece of information. We can use it to distinguish between two states.

These two states could be represented by True \(T\) and False \(F\), but by convention we use \(1\) and \(0\).

We can combine bits to store more complex pieces of information. If we have \(n\) bits, we can distinguish between \(2^n\) states.

Eight bits together constitute a byte. This can represent one of \(2^8=256\) states.

\subsection{Decimal as basis}

We could also represent numbers using a decimal basis, so each element could take \(10\) states, and \(n\) elements could represent \(10^n\) states.

The choice of basis is an abstraction.

\subsection{Representing numbers}

By convention (and in particular in C) we can represent numbers using their basis like: \(0b0100\) for a \(4\)-bit number, representing \(4\) in binary. The \(0b\) at the start indicates that what will follow is a number written in binary.

Could also write \(0d5322\) for \(5322\), in decimal.

Could write \(0x53A4\) for \(21412\) in hexidecimal.
