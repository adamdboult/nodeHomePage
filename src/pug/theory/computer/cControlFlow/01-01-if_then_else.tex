
\subsection{If-then-else}

\begin{verbatim}
if () x else y
\end{verbatim}
can group multiple with block

\begin{verbatim}
if () {} else {}
\end{verbatim}
block curly counts as single command



can pass varibales to logical evaluation
\begin{verbatim}
int x = 1;
if (x)
\end{verbatim}

evaluating ints and longs. seems to be if >= 1 then true otherwise false?
does something on char too?

\subsection{Conditional operator}

If we want to set the value of something based on a conditional, eg:

\begin{verbatim}
if (a > b) {
    result = x;
}
else {
    result = y;
}
\end{verbatim}

We can instead do the following

\begin{verbatim}
result = a > b ? x : y;
\end{verbatim}

If we have something of the form:

\begin{verbatim}
if (a) {
    result = a;
}
else {
    result = y;
}
\end{verbatim}


We can write the following in some implementations of C.

\begin{verbatim}
result = a ? : y;
\end{verbatim}

Known as the Elvis operator (because of "?:").



