
\subsection{pacman -S: Installing and updating packages}


pacman -S (sync) is family of commands

to refresh packages and upgrade (u upgrade, y, download database from remote)

\begin{verbatim}
pacman -Syu
\end{verbatim}


install it
\begin{verbatim}
pacman -Syu my_package
\end{verbatim}

This syncs as installing, which is safer. The following is less safe but can also be done:

\begin{verbatim}
pacman -S my_package
\end{verbatim}

You can search for packages:

\begin{verbatim}
pacman -Ss string_in_package
\end{verbatim}


\subsection{pacman -S: Managing the cache}


cache is stored in pacman in /var/cache/pacman/pkg/


/etc/pacman.conf


can clear cache of uninstalled packages with
\begin{verbatim}
pacman -Sc
\end{verbatim}

double clean to be more aggressive (remove cache of installed packages)
\begin{verbatim}
pacman -Scc
\end{verbatim}


\subsection{pacman -R}
remove it
\begin{verbatim}
pacman -Rns my_package
\end{verbatim}

The -s flag removes dependencies which are no longer needed.

The -n flag removes config files.

Can just remove the package and not dependencies:

\begin{verbatim}
pacman -R my_package
\end{verbatim}



\subsection{pacman -Q}

Query

Info on a package, including what depends on it.
\begin{verbatim}
pacman -Qi <package>
\end{verbatim}


list of explicitly installed:
\begin{verbatim}
pacman -Qe
\end{verbatim}
To see packages installed without the official repository (eg AUR) use
\begin{verbatim}
pacman -Qm
\end{verbatim}

Packages which were installed as dependencies
\begin{verbatim}
pacman -Qd
\end{verbatim}

Packages which are dependencies and orphans
\begin{verbatim}
pacman -Qdt
\end{verbatim}

list of local files associated with package:
\begin{verbatim}
pacman -Ql <package_name>
\end{verbatim}

