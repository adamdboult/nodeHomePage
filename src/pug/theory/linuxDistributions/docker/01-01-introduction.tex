
\subsection{Pulling images}

\begin{verbatim}
docker pull alpine:latest
\end{verbatim}


List images:
\begin{verbatim}
docker image ls
\end{verbatim}

Or:

\begin{verbatim}
docker images
\end{verbatim}


To remove an image:
\begin{verbatim}
docker image remove alpine:latest
\end{verbatim}

To remove all images (without an associated container):

\begin{verbatim}
docker image prune --all
\end{verbatim}


\subsection{Running images as containers}

If the image is not already pulled, it will automatically be pulled, and so there is generally no need to manually pull images.

\begin{verbatim}
docker container create --name container_name alpine:latest
docker create --name container_name alpine:latest
\end{verbatim}

If no name is provided, a random one will be created.

Once a container has been created, it can be started.

List containers. The "a" flag makes it show all containers, not just those running.
\begin{verbatim}
docker ps -a
\end{verbatim}

\begin{verbatim}
docker container start container_name
docker start container_name
\end{verbatim}

We can run it interactively and with a TTY.

\begin{verbatim}
docker container start --interactive --tty container_name
docker start -it container_name
\end{verbatim}

Run can be used instead of create and start.
\begin{verbatim}
docker container run -it --name container_name alpine:latest
docker run -it --name container_name alphine:latest
\end{verbatim}

Stopping containers.
\begin{verbatim}
sudo docker kill $(sudo docker ps -q)
\end{verbatim}

Removing containers.
\begin{verbatim}
sudo docker rm $(sudo docker ps -a -q)
sudo docker system prune -af (this does much more than other stuff, saved lots of space. what does it do?)
\end{verbatim}


\subsection{Working without root}

