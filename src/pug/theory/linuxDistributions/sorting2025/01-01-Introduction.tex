
\subsection{Introduction}

arch notes:
+ h3 on mkinitcpio

pacman -Q. note that doing Qq means the version numbers aren't returned, which is probably what you want


title include: Os level virtualisation, system virtual machines
native:
+ kvm (Kernel-based Virtual Machine)
  * vfio-pci
  * host pass through
  * runs on top of linux kernel
  * requires intel vt or amd v
to something on running software inc OS for same hardware
+ vmware
+ virtualbox
"vagrant" with virtualbox?
hosted:
+ qemu
  * emulates hardware
  * jit/interpreter? is dynamic recompilation something else?
proxmox: kvm + qemu
to uninstall orphans:
pacman -Qtdq | sudo pacman -Rns -


guix
+ guix install emacs
+ guix remove emacs
don't need sudo, instals locally.
+ "guix pull" to update package information
+ "guix upgrade" to upgrade installed packages
+ can run eg "guix shell emacs --" to open guix shell which includes emacs. can then run "emacs"
+ alt can do "guix shell emacs -- emacs"
+ maybe just "guix shell emacs"
+ can view DAG with "guix graph --type=package emacs > emacs-graph.dot"
+ running eg "guix shell emacs" gives packages required to run emacs, ie runtime dependencies. running "guix shell --development" emacs gives build time dependencies.
+


nix-shell from non-nix os
eg can run nix-shell -p lolcow cowsay to get shell with both
can run "nix-shell -p lolcow --run lolcow" to just run it
nix-collect-garbage
nix-env exists, but can clash with existing packages
pacakges stored in /nix/store/
run nix-channel update to update packages
can maintain shell.nix file to save exact config. running nix-shell looks for this file in same folder.

docker save (creates tar?)

docker: how to use GPU/CUDA

docker rm --rum (removes after finished running)
docker rmi is alias for docker image rm
docker rm ias alias for remove a container
docker: section on tags
docker compose to separate page
podman page tookrebuild and restart docker-compose after pulling:
+ docker-compose up -d --build
  + actually don't need to do down first?
  + can do this at the start too. replace readme with this? or just remove this stuff from readme given it's in my home page



pacman.
+ --asdeps
+ can use that flag when installing a package, and it will be treated as a dependency on other thing
+ if install something which has that as optional, will be uninstalled.


XEN as alternative to KVM. XEN runs bare metal, no need for host OS?
docker update or something?

managing qemu installs with virt-manager, libvert

arch linux: concept of groups as well as packages. mega packages (packages which just depend on others, means now backages which get added are added too).
makepkg -r flag to uninstall anything just installed for the creation procesd
arch: makepkg makes tarball
using pacman -U after makepkg installs from tarbvall?
using -i flag with makepkg just automates this

both porgage and emerge on gentoo page name

thing on unattended upgrades in page on each distro

alpine linux. very lightweight. has own package manager. alpine pacakge keeper. uses musl libc and busybox. lightweight

buildroot. distro for embedded systems


