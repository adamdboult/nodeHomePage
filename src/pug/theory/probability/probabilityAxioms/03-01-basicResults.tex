
\subsection{Probability of null}

\(P(\Omega )=1\)

\(P(\Omega \lor \varnothing )=1\)

\(P(\Omega )+P(\varnothing )=1\)

\(P(\varnothing )=0\)

\subsection{Monotonicity}

Consider \(E_i\subseteq E_j\):

\(E_j=E_i\lor E_k\)

\(P(E_j)=P(E_i\lor E_k)\)

Disjoint so:

\(P(E_j)=P(E_i)+P(E_k)\)

We know that \(P(E_k)\ge 0\) from axiom \(1\) so:

\(P(E_j)\ge P(E_i)\)

\subsection{Bounds of probabilities}

As all events are subsets of the sample space:

\(P(\Omega )\ge P(E)\)

\(1\ge P(E)\)

From axiom \(1\) then know:

\(\forall E\in F [0\le P(E)\le 1]\)

\subsection{Union and intersection for null and universal}

\(P(E\land \varnothing )=P(\varnothing )=0\)

\(P(E\lor \Omega )=P(\Omega )=1\)

\(P(E\lor \varnothing)=P(E)\)

\(P(E\land \Omega )=P(E)\)

\subsection{Separation rule}

Firstly:

\(P(E_i)=P(E_i\land \Omega)\)

\(P(E_i)=P(E_i\land (E_j\lor E_j^C))\)

\(P(E_i)=P((E_i\land E_j)\lor (E_i\land E_j^C))\)

As the latter are disjoint:

\(P(E_i)=P((E_i\land E_j)+(E_i\land E_j^C))\)

\subsection{Addition rule}

We know that:

\(P(E_i\lor E_j)=P((E_i\lor E_j)\land (E_j\lor E_j^C))\)

By the distributive law of sets:

\(P(E_i\lor E_j)=P((E_i\land E_j^C)\lor E_j)\)

\(P(E_i\lor E_j)=P((E_i\land E_j^C)\lor (E_j\land (E_i\lor E_i^C))\)

By the distributive law of sets:

\(P(E_i\lor E_j)=P((E_i\land E_j^C)\lor (E_j\land E_i)\lor (E_j\land E_i^C))\)

As these are disjoint:

\(P(E_i\lor E_j)=P(E_i\land E_j^C)+ P(E_j\land E_i)+P(E_j\land E_i^C)\)

From the separation rule:

\(P(E_i\lor E_j)=P(E_i)-P(E_i\land E_j)+ P(E_j\land E_i)+P(E_j)-P(E_j\land E_i)\)

\(P(E_i\lor E_j)=P(E_i)+P(E_j)-P(E_i\land E_j)\)

\subsection{Probability of complements}

From the addition rule:

\(P(E_i\lor E_j)=P(E_i)+P(E_j)-P(E_i\land E_j)\)

Consider \(E\) and \(E^C\):

\(P(E\lor E^C)=P(E)+P(E^C)-P(E\land E^C)\)

We know that \(E\) and \(E^C\) are disjoint, that is:

\(E\land E^C=\varnothing\)

Similarly by construction:

\(E\lor E^C=\Omega \)

So:

\(P(\Omega )=P(E)+P(E^C)-P(\varnothing)\)

\(1=P(E)+P(E^C)\)

