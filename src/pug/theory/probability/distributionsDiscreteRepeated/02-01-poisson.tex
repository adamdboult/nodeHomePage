
\subsection{Poisson distribution}

\subsection{Definition}

We can use the Poisson distribution to model the number of indepedent events that occur in an a time period.

For a very short time period the chance of us observing an event is a Bernoulli trial.

\(P(1)=p\)

\(P(0)=1-p\)

\subsection{Chance of no observations}

Let's consider the chance of repeatedly getting \(0\): \(P(0;t)\).

We can see that: \(P(0;t+\delta t)=P(0;t)(1-p)\).

And therefore:

\(P(0;t+\delta t)-P(0;t)=-pP(0;t))\)

By setting \(p=\lambda \delta t\):

\(\dfrac{P(0;t+\delta t)-P(0;t)}{\delta t}=-\lambda P(0;t))\)

\(\dfrac{\delta P(0;t)}{\delta t}=-\lambda P(0;t)\)

\(P(0;t)=Ce^{-\lambda t}\)

If \(t=0\) then \(P(0;t)=0\) and so \(C=1\).

\(P(0;t)=e^{-\lambda t}\)

\subsection{Deriving the Poisson distribution}


