
\subsection{Introduction}

pricing:
Practical pricing
The problem

Firms dont have access to elasticity information.
Base on existing competitors
Gross Profit Margin Target
What The Market Will Bare
intertemporal strategies: sales max, market share max

\subsection{Management accounts}
Not audited
\subsection{Regulatory accounts}
\subsection{corp gov}
CEO, CTO, ops, other positions?

governance processes?
\subsection{Corporate data}
\subsubsection{Benchmarking}
\subsubsection{Internal benchmarking}
Departments to benchmark
Data to use
\subsubsection{External benchmarking}
Selection of peers
\subsubsection{Metrics}
Internal dashboard with key metrics. Key to get employee buy in
Measure outputs not inputs
Pair conflicting outputs like Customer satisfaction and fraud detection and false positives
Look for outliers eg eBay users, people looking at themselves on LinkedIn
Measure what you want to target. Identify key metrics and track them
\subsubsection{Sales Analysis}
Funnel chart
\subsection{Business Strategy}
\subsubsection{Choosing a Product}
\subsubsection{Identifying Ideas}
What is the mission?
One line summary of product
Building blocks
Does the product copy or combine old blocks, or add new blocks?
What is valuable?
What is the problem I am solving?
Can I be the best product to a specific set of users?

\subsubsection{Can we do it?}
Core competencies
What areas must I be an expert in?
What skills are unique?
What beliefs are unique?

\subsubsection{Market Research}
Product lifecycle
Product Market fit
Which consumer segment are you targeting? Expand later
Why couldn't it be done now?
Why will it be too late soon?
Is this unique?
What scale needed to achieve profitability?
Is your product ripe for competition? Network effects good, external tech improvements bad
Vertical integration can be very valuable. Nature, benefits, barriers, costs
- Spacex and tesla can be examined through this lens
Industry timing
Early entry has unique advantages and disadvantages
What scale of the market is relevant when looking at competition?
Costs

\subsubsection{Capturing value}
Can I ensure longetivity?
Product creates x value, y captured, reasons, implications?
Scientists tend not to capture value, Wright, Einstein, implications
Huge boost in cars, airlines, but most investments bad
\subsubsection{Pricing}
Rounded prices, eg 20vs 74, feel better. Called fluency

However overly rounded feels artificially inflated. 100, 5000

Left digit fluctuations. 1.99 vs 2. Charm


pricing: if previously free can be dangerous. reference price or category? eg red cross donuts, airline baggage vs calls, christmas dinner

\subsubsection{Enterprise Software}
\subsubsection{Differences from other software}
Longer lead times to integrate with software.
Deal with intermediaries
Established competitors, needs a new product to enter
\subsubsection{Context}
Much bigger market
Cloud computing allows easier deployment
Mobile access needs to be managed

\subsubsection{Startups}
Innovative companies more likely to use startups
Most industries undergo large changes resulting from tech market drifts. How does your product react to this?

\subsubsection{Growth}
\subsubsection{Themes}
Monopoly in small market and expand
Launch gradually not suddenly, reflects nature of client growth
Partnerships with big users overrated as a method of growth

\subsubsection{International}
International scaling. Write website in strings and crowdsource translations to English strings

\subsubsection{Marketing}
Referral program? Eg x pounds to referer and referee
Paid growth? Customer acquisition cost CAC vs clv
Which users to physically target? Those like yourself facing the same problem
Shotgun approach and target the most involved
Get small users manually and listen to feedback
Viral K factor

\subsubsection{Segment targeting}
Customer life time value? Clv. Increase stickiness, and increase viral for sustainable growth
Clv timing and discount rate?

\subsubsection{Tracking}
What are the metrics for active users, intensity?

\subsubsection{Retention}
Emails
Do not get on spam list! Notifications better than newsletters, consider user types, what do they value for retention?

\subsubsection{Product Development}
\subsubsection{Automation}
Processes which should be automated, like client registration, may better be done manually first so allow for better designed automation

\subsubsection{Segments}
Optimise for power users, and marginal users, notifications?

\subsubsection{Scaling}
\subsubsection{Management}
Company wide memos of meetings
Maintain transparency
\subsubsection{Web Design}
\subsubsection{Usability}
Consider the knowledge gap between zero and what is required to use the app. How can you help people move, how can you move the end, what is the impact of new features?
Faqs, tooltips, page specific help
Customer support from high levels crucial and can drive improvements

\subsubsection{User updates}
Notifications of new features on login

\subsubsection{Storyboards}
What are the magic moments of the site for each segment, how to get them to reach them?
What are the emotions your users should have while on the site, signing up? How can design facilitate this? More of a relationship with the site, eg commentary during sign up
\subsection{Governance}
\subsubsection{Boards}
\subsubsection{C Suite}
\subsubsection{Shareholders}
\subsubsection{Laws}
\subsubsection{Types of businesses}
Limited liability
Public limited liability
Partnership
Sole trader
Franchise
\subsubsection{Family}
Can think longer term
Owner, manager principal agent lessened
Deaths, family issues can have large impacts
\subsection{start ups}
start up big: how to assess profitability of existing products? how to assess barriers to entry?

how to do market research? page on minimum viable product
\subsection{Financial modelling}
Model sales in each period, eg month, quarter, year

sales have attrition rates, new customers

maybe different in not substription or regular

revenue per sale

total revenue

expense modelling: staff

salary projection for each type of staff. annual increase assumpotions. inc tax, bonus, other costs, benefits

get total operating expenses

then net income = rev-op expense

cash balance:
+ start with cash value
+ each period gain net income and investment
interest?
taxation?
depreciation?
amortisation?
EBIT?
EBITDA?


