
\subsection{Introduction}

Lattice-based cryptography

classical encryption
	split out monoalphabetic substitutions, and cracking them
	page on transposition ciphers, and cracking them
	codebooks and book ciphers, and cracking them
	one time pads
h3 polyalphabetic substitution and stream ciphers and cracking them

modern symmetric h3:
	AES
asymmetric h3
	split out integer factorisation and elliptic-curve crpytography to their own page
	diffie hellman
	RSA

ECC backdoor: can choose parameters of curve to use, if p=nq then can reverse if know n?


probabilistic data strucutres.
bloom filter.
rather than test for in or out and give yes no, give yes and possibly. less space. good if can deal with possibly. eg for caching type stuff.

Cryptography: cipher. Block cipher and character cipher. Stream cipher.

probabilistic algorithm:
+ page on gradient descent with momentum
+ page on Adaptive Gradient Algorimth (adagrad) and Root Mean Square Propagation (RMSP) (improved version of Adagrad)
+ page on Adam
+ h3 on types of method when have datasets: batch gradient descent (calculate error for each example then update model), stochastic gradient descent(update model after each calculation); mini batch gradient descent (update after set number of entries)
+ las vegas algorithm: guaranteed to be correct, not guaranteed to take a given time/space



Crypto: Substitution–permutation network (sp network). Related to block ciphers

arithmetic coding. lossless encoding.


probabilistic algorithms: simulated annealing

h3 on improving types of algorithms with randomness
1. optimisation problems using stochasitic optimisation, including stochastic gradient descent and simulated annealing

pseudo random numbers
thing on passwords. john the ripper. password cracking


