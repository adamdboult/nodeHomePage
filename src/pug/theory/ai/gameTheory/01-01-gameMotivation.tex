
\subsection{Interaction between agents}

\subsubsection{Introduction}

Previously actions map to states. Not now.

\subsubsection{More intro}

Previously we modelled utility as a function of variables in control of the agent, or constants. We now add another type of term: variables controlled by other agents.

Consider a simple pair of agents:

\(u_a=f_a(x_a,y_a)\)

\({actions_a}=\{x_a,y_a\}\)

\(u_b=f_b(x_b,y_b)\)

\({actions_b}=\{x_b,y_b\}\)

Each agent’s decision does not affect the other agent. Consider now a utility function:

\(u_a=f_a(x_a,y_a)\)

\({actions_a}=\{{x offer}; {y offer}\}\)

\(u_b=f_b(x_b,y_b)\)

\({actions_b}=\{accept; reject\}\)

Where \(a\) offers a trade to \(b\) and \(b\) accepts or rejects. This is an example of a sequential game. There are many types of game, with differing implications.

