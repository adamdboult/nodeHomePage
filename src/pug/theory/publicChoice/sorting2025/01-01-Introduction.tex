
\subsection{Introduction}

italian elitist school, inc james burnham

power actions:
+ take assets
+ enslave: require work

What is the difference between mosca and Michel's version of elite theory?

may's theorem
arrow's impossibility theorem
mckelvey schofield chaos theorem
median voter theorem
condorcet paradox
condorcet jury  theorem
gibbard satterthwaite theorem


Politics game theory
+ Enforcement of rules
+ Setting of rules
+ voting on rules


international relations
+ realism
+ liberalism

\subsection{Big h3: Public choice}
identity politics on econ. form coalitions based on demographics, other factors. reinforcing. if others do it, you do it as defence.
policies. official discrimination. types of identity. different groups have different policy preferences. eg religious groups. need narratives. oppressed, goals?

\subsection{Institutions}
\subsubsection{2 player games h3}
One period model, want to steal from other

multi period game, can punish if steal

application of prisoner dilemma
\subsubsection{Multi-party game h3}
Can punish stealers collectively.

Can punish those who don't punish.

Other equilibria:

+ Subset agree to coordinate, but allow stealing of outsiders

+ Coalition depends on heterogenous punishment costs

Stability of coalitions. Changes will be opposed by those in power. Stability of ruling coalition can be set in
\subsubsection{Voting}
condorcet

Members in ruling coaliton can vote
\subsubsection{Multi-party game with production h3}
Want incentive to produce, so something to steal

Trade-off between growth  and expropriation

Production can lead to changes in punishment costs. Coalition fragility

What are policy levers:
+ Taxes
+ Institutions

What determines equilibrium for both
\subsubsection{Multi-state games}
Need to spend on defence/offence

Different equilibrium on taxes/institutions

Trade-off between internal and external security

