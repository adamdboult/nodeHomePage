
\subsection{Own-price elasticity of demand}

We have our Marshallian demand function:

\(x_i=x_{di}(I, \mathbf p)\)

The derivative of this with respect to price is the additional amount consumed after prices increase.

\(\dfrac{\delta }{p_i}x_{di}(I, \mathbf p)\)

For the Cobb-Douglas utility function, this is:

\(\dfrac{\delta }{p_i}x_{di}(I, \mathbf p)\)

In addition to the derivative, we may be interested in the elasticity. That is, the proportional change in output after a change in price.

\(\xi_i =\dfrac{\dfrac{\Delta x_i}{x_i}}{\dfrac{\Delta p_i}{p_i}}\)

\(\xi_i =\dfrac{\Delta x_i}{\Delta p_i}\dfrac{p_i}{x_i}\)

For the point-price elasticity of demand we evaluate infintesimal movements.

\(\xi_i =\dfrac{\delta x_i}{\delta p_i}\dfrac{p_i}{x_i}\)

\subsection{Constant price elasticity of demand}

If the point-price elasticity of demand is constant we have:

\(\xi_i =\dfrac{\delta x_i}{\delta p_i}\dfrac{p_i}{x_i}=c\)

This means that small changes in the price at low level cause large changes in quantity.

\subsection{Arc-price elasticity of demand}

We may have price changes which are non-infintesimal.

\(E_d=\dfrac{\Delta Q/\bar Q}{\Delta P/\bar P}\)

Where \(\bar Q\) and \(\bar P\) are the mid-points between the start and end.

