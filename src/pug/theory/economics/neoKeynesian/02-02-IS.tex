
\subsection{The Keynesian cross and the Investment Saving (IS) curve}

\subsubsection{The Keynesian cross}

We have:

\(Y=C(Y-T(Y))+I(r)+G+NX(Y)\)

Where:

\begin{itemize}
\item \(Y\) is output
\item \(C\) is consumption
\item \(T\) is taxes
\item \(I\) is investment
\item \(r\) is the real interest rate
\item \(G\) is government spending
\item \(NX\) is net exports
\end{itemize}

The Keynesian cross plots:

\(Y\)

Against:

\(C(Y-T(Y))+I(r)+G+NX(Y)\)

This identifies an equilibrium level of output.

\subsubsection{The IS curve}

The IS curve plots the equilibrium level of output from the Keynesian cross against the real interest rate.

As the real interest rate rises, investment and therefore output falls.

\subsubsection{The slope of the IS curve}

The slope of the IS curve depends on taxes and net exports.

