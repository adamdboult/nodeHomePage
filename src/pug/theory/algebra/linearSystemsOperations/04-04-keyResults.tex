
\subsection{Transposition and conjugation}

\subsubsection{Transposition}

A matrix of dimensions \(m*n\) can be transformed into a matrix \(n*m\) by transposition.

\(B=A^T\)

\(b_{ij}=a{ji}\)

\subsubsection{Transpose rules}

\((M^T)^T=M\)

\((AB)^T=B^TA^T\)

\((A+B)^T=A^T+B^T\)

\((zM)^T=zM^T\)

\subsubsection{Conjugation}

With conjugation we take the complex conjugate of each element.

\(B=\overline A\)

\(b_{ij}=\overline a_{ij}\)

\subsubsection{Conjugation rules}

\(\overline {(\overline A)}=A\)

\(\overline {(AB)}=(\overline A)( \overline B)\)

\(\overline {(A+B)}=\overline A+\overline B\)

\(\overline {(zM)}=\overline z \overline M\)

\subsubsection{Conjugate transposition}

Like transposition, but with conjucate.

\(B=A^*\)

\(b_{ij}=\bar{a_{ji}}\)

Alternatively, and particularly in physics, the following symbol is often used instead.

\((A^*)^T=A^\dagger\)

