
\subsection{The fixed effects estimator}
\subsubsection{Recap on the model}

Our model is:

\(y_{ij}=\alpha + X_{ij}\theta + \xi_j + \epsilon_{ij}\)

\subsubsection{The fixed effects estimator}

With fixed effects we assume that \(U_{ij}\) is a constant for each group. That is:

\(U_{ij}=\delta_{ij}U_j\)

\(y_{ij}=\alpha + X_{ij}\theta +\epsilon_{ij}+\delta_{ij}U_{j}\)

We can use this in a regression if the standard assumptions of OLS are met. In particular, that group membership is uncorrelated with the error term.

We add these dummies to \(X_{ij}\) and regress:

\(y_{ij}=\alpha + X_{ij}\theta +\epsilon_{ij}\)

The parameter for the dummy is the fixed effect of group membership.

As we are including membership in the dependent variables, there is no problem if group membership correlates with other independent variables.

\subsubsection{Using the within and between transformations}

\((y_{ij}-\bar y_{i})=(X_{ij}-\bar X_{i})\theta +(U_{ij} -\bar U_{i}) +(\epsilon_{ij}-\bar \epsilon_{i})\)

Or:

\(\ddot y_{ij}=\ddot X_{ij}\theta +\ddot \epsilon_{ij}\)

This this get the same outcome, but is a different computational process.

