
\subsection{Extreme IID multinomial}

\subsubsection{IID}

The probability of \(j\) being chosen is:

\(P_{ij}=P(\epsilon_{ik} <v_{ij} -v_{ik} +\epsilon_{ij}\forall k\ne j)\)

If these are independent then we have:

\(P_{ij}=\prod_{k\ne j} P(\epsilon_{ik} <v_{ij} -v_{ik} +\epsilon_{ij})\)

\(P_{ij}=\prod_{k\ne j} F_\epsilon (v_{ij} -v_{ik} +\epsilon_{ij})\)

We do not know \(\epsilon_{ij}\) so we have to integrate over possibilities.

\(P_{ij}=\int [\prod_{k\ne j} F_\epsilon (v_{ij} -v_{ik} +\epsilon_{ij})]f_\epsilon(\epsilon_{ij})d\epsilon_{ij}\)

\subsubsection{Extreme values}

We have:

\(P_{ij}=\int [\prod_{k\ne j} F_\epsilon (v_{ij} -v_{ik} +\epsilon_{ij})]f_\epsilon(\epsilon_{ij})d\epsilon_{ij}\)

If \(\epsilon \) is extreme value type-I this gives us:

\(P_{ij}=\dfrac{e^{v_{ij}}}{\sum_k e^{v_{ik}}}\)

\subsubsection{Independence of irrelevant alternatives}

Consider the ratio two probabilities:

\(\dfrac{P_{ij}}{P_{im}}=\dfrac{e^{v_{ij}}}{e^{v_{im}}}\)

This means that changes to any other products do not affect relative odds.

This can be undesirable. For example removing one option may cause unbalanced substitution.

For example raising the price of buses may cause more substitution to trains than helicopter, for a commute.

