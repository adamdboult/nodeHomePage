
\subsection{Taylor series}


\(f(x)\) can be estimated at point \(c\) by identifying its repeated differentials at point \(c\).

The coefficients of an infinate number of polynomials at point \(c\) allow this.

\(f(x)=\sum_{i=0}^{\infty }a_i(x-c)^i\)

\(f'(x)=\sum_{i=1}^{\infty }a_i(x-c)^{i-1}i\)

\(f''(x)=\sum_{i=2}^{\infty }a_i(x-c)^{i-2}i(i-1)\)

\(f^j(x)=\sum_{i=j}^{\infty }a_i(x-c)^{i-j}\dfrac{i!}{(i-j)!}\)

For \(x=c\) only the first term in the series is non-zero.

\(f^j(c)=\sum_{i=j}^{\infty }a_i(c-c)^{i-j}\dfrac{i!}{(i-j)!}\)

\(f^j(c)=a_ij!\)

So:

\(a_j=\dfrac{f^j(c)}{j!}\)

So:

\(f(x)=\sum_{i=0}^\infty (x-c)^i \dfrac{f^i(c)}{i!}\)

