
\subsection{Vector potential}

Given a vector field \(\mathbf F\) we may be able to identify another vector field \(A\) such that:

\(\mathbf F =\nabla \times \mathbf A\)

Existence:

We know that the divergence of the curl for any vector field is \(0\), so this applies to \(A\):

\(\nabla . (\nabla \times \mathbf A)=0\)

Therefore:

\(\nabla . \mathbf F= 0\)

This means that if there is a vector potential of \(\mathbf F\), then \(\mathbf F\) has no divergence.

\subsection{Non-uniqueness of vector potentials}

Vector potentials are not unique.

If \(\mathbf A\) is a vector potential of \(\mathbf F\), then so is \(\mathbf A + \nabla c\), where \(c\) is a scalar field and \(\nabla c\) is its gradient.

\subsection{Conservative vector fields}

Not all vector fields have scalar potentials. Those that do are conservative.

For example if a vector field is the gradient of a scalar height function, then the height is a scalar potential.

If a vector field is the rotation of water, there will not be a scalar potential.

