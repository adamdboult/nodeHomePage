
\subsection{Curl}

The curl of a vector field is defined as:

\(curl \mathbf F=\nabla \times \mathbf F\)

Where: \(\nabla =(\sum_{i=1}^n e_i\dfrac{\delta }{\delta x_i})\)

And: \(\mathbf x\times \mathbf y=\||\mathbf x|| ||\mathbf y|| \sin(\theta )\mathbf n\)

The curl of a vector field is another vector field.

The curl measures the rotation about a given point. For example if a vector field is the gradient of a height map, the curl is \(0\) at all points, however for a rotating body of water the curl reflects the rotation at a given point.

\subsection{Divergence of the curl}

If we have a vector field \(\mathbf F\), the divergence of its curl is \(0\):

\(\nabla . (\nabla \times \mathbf F)=0\)

