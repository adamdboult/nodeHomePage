
\subsection{Scalar potential}

Given a vector field \(\mathbf F\) we may be able to identify a scalar field \(P\) such that:

\(\mathbf F=-\nabla P\)

\subsection{Non-uniqueness of scalar potentials}

Scalar potentials are not unique.

If \(P\) is a scalar potential of \(\mathbf F\), then so is \(P+c\), where \(c\) is a constant.

\subsection{Conservative vector fields}

Not all vector fields have scalar potentials. Those that do are conservative.

For example if a vector field is the gradient of a scalar height function, then the height is a scalar potential.

If a vector field is the rotation of water, there will not be a scalar potential.

