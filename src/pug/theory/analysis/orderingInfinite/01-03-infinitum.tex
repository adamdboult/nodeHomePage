
\subsection{Infinitum and supremum}

\subsubsection{Infinitum}

Consider a subset \(S\) of a partially ordered set \(T\).

The infinitum of \(S\) is the greatest element in \(T\) that is less than or equal to all elements in \(S\).

For example:

\(\inf [0,1]=0\)

\(\inf (0,1)=0\)

\subsubsection{Supremum}

The supremum is the opposite: the smallest element in \(T\) which is greater than or equal to all elements in \(S\).

\(\sup [0,1]=1\)

\(\sup (0,1)=1\)

\subsubsection{Max and min}

If the infinitum of a set \(S\) is in \(S\), then the infinimum is the minimum of set \(S\). Otherwise, the minimum is not defined.

\(\min [0,1]=0\)

\(\min (0,1)\) isn't defined.

Similarly:

\(\max [0,1]=1\)

\(\max (0,1)\) isn't defined.

