
\subsection{\(\sigma \)-algebra}

\subsubsection{Review of algebra on a set}

An algebra, \(\Sigma \), on set \(s\) is a set of subsets of \(s\) such that:

\begin{itemize}
\item Closed under intersection: If \(a\) and \(b\) are in \(\Sigma \) then \(a\land b\) must also be in \(\Sigma \)
\item \(\forall ab [(a \in \Sigma \land b \in \Sigma )\rightarrow (a\land b \in \Sigma)]\)
\item Closed under union: If \(a\) and \(b\) are in \(\Sigma \) then \(a\lor b\) must also be in \(\Sigma \).
\item \(\forall ab [(a \in \Sigma \land b \in \Sigma )\rightarrow (a\lor b \in \Sigma)]\)
\end{itemize}

If both of these are true, then the following is also true:

\begin{itemize}
\item Closed under complement: If \(a\) is in \(\Sigma \) then \(s \backslash a\) must also be in \(\Sigma \)	
\end{itemize}

We also require that the null set (and therefore the original set, null's complement) is part of the algebra.

\subsubsection{\(\sigma \)-algebra}

A \(\sigma \)-algebra is an algebra with an additional condition:

All countable unions of sets in \(A\) are also in \(A\).

This adds a constraint. Consider the real numbers with an algebra of all finite sets.

This contains all finite subsets, and their complements. It does not contain \(\mathbb{N}\).

However a \(\sigma \)-algebra requires all countable unions to be including, and the natural numbers are a countable union.

The power set is a \(\sigma \)-algebra. All other \(\sigma \)-algebras are subsets of the power set.

