
\subsection{Association rules}

\subsubsection{The data}

We have a transaction dataset, \(D\).

This includes transactions of items in \(I\).

Any subset of \(I\) is an itemset.

A subset of size \(k\) is a \(k\)-itemset.

Transactions are a \(k\)-itemset with a unique id, tid.

The set of all transactions is \(T\).

A tidset is a subset of \(T\).

\subsubsection{Forming a lattice}

We have a total order on the items.

An itemset \(ab\) is greater than \(a\), for examplle.

The two points of the lattice are the nullset, and \(I\).

\subsubsection{Mappings}

We have a mapping from \(I\) to \(T\) called \(t\).

We have another mapping from \(T\) to \(I\) called \(i\).

\subsubsection{Frequency}

We define the frequency of an itemset as the number of transactions it appears in.

We can write the frequency of \(A\) as \(\sum A\).

