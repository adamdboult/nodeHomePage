
\subsection{The projection and annihilation matrices}

\subsubsection{The projection matrix}

We have \(X\).

The projection matrix is \(X(X^TX)^{-1}X^T\)

The projection matrix maps from actual y to predicted y

\(\hat y = Py\)

Each entry refers to the covariance between actual and fitted

\(p_{ij}=\dfrac{Cov (\hat y_i, y_j}{Var (y_j)}\)

\subsubsection{The annihilation matrix}

We can get residuals too:

\(u=y-\hat y=y-py=(1-P)y\)

\(1-P\) is called the annihilator matrix

We can now use the propagation of uncertainty

\(\Sigma^f = A\Sigma^x A^T\)

To get:

\(\Sigma^u = (I-P)\Sigma^y (I-P)\)

Annihilator matrix is:

\(M_X=I-X(X^TX)^{-1}X^T\)

Called this because:

\(M_XX=X-X(X^TX)^{-1}X^TX\)

\(M_XX=0\)

Is called residual maker

