
\subsection{Policy goals}

\subsubsection{Targets}

Central banks can use monetary policy to achieve goals. Monetary policy directly affects inflation and exchange rates, but also losses in output which result from the stickiness of prices.

Due to the interaction of each of these targets, there are trade-offs in pursuing multiple goals. Reducing output volatility could require volatile inflation, and stabilising the exchange rate could destabilise inflation and output.

\subsubsection{Export promotion}

The theory behind currency manipulation is that by devaluating a currency exports are more competitive and so GDP will rise.

If a currency is devalued by printing large amounts of money, this will make exports cheaper temporarily if they take time to adjust their nominal prices upwards.

If a currency is devalued by imposing capital controls and intervening in capital markets, then the relative price of exports can be maintained below the level it would otherwise be at.

