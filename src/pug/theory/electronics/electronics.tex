\documentclass[oneside]{book}

\usepackage{amsmath, amssymb}
\usepackage{pgfplots}
\pgfplotsset{compat=1.18}
\usepackage{parskip}

\begin{document}

\author{Adam Boult (www.bou.lt)}
\title{Electronics}
\maketitle

\setcounter{tocdepth}{0}
\tableofcontents

\chapter*{Preface}
\addcontentsline{toc}{chapter}{Preface}

This is a live document, and is full of gaps, mistakes, typos etc.



\part{Electric circuits}
\include{circuits}
\include{circuitDiagrams}
\include{switchesVariable}
\include{diodesCapacitors}

\part{Inputs and outputs}
\include{lights}
\include{motors}
\documentclass[oneside]{book}

\usepackage{amsmath, amssymb}
\usepackage{pgfplots}
\usepackage{parskip}

\begin{document}

\author{Adam Boult (www.bou.lt)}
\title{Audio}
\maketitle

\setcounter{tocdepth}{0}
\tableofcontents

\chapter*{Preface}
\addcontentsline{toc}{chapter}{Preface}

This is a live document, and is full of gaps, mistakes, typos etc.



\part{SORT}
\include{sorting2025}

\end{document}


\include{radio}
\include{temperatureMoisture}
\include{clocks}

\part{Intermediate}
\include{mains}
\include{circuitBoards}

\part{Pre-computer electronics for finite state machines including making a calculator: boot ROMs/firmware (mask ROM/diode matrix), punch cards, memory mapped IO for LEDs and keyboards and volatile RAM (SRAM and DRAM) (eg rom for e and pi in calculator, (could have big page on sequential/combinatorial logic before, and could introduce some of these here)}

\part{RAM and storage}

\part{Motherboards}

\part{Keyboards, printers and terminals}

\part{SORT}
\include{sorting2025}

\end{document}

