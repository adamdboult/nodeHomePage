
\subsection{Introduction}

permenides in natural languge?

Concept of common knowledge

Proper names
Particulars and universals (hume bundle theory, platonic ontology, aristotle 3rd man, parmenides, relational universals (is darker than), infinite regress, wittgenstein and ontology of facts "the world is all that is the case", "the world is the totality of facts, not of things")
Syllogisms and Aristotle
Truths of fact and necessity
Correspondence theory of truth
Coherence theory of truth
Coherence theory of justification
Foundationalism

modal logic?

\subsection{Rationalism}
Basic beliefs: Incorrigibility (cognito ergo sum)
\subsection{Empiricism}
\subsection{h3 Linguistics}
ideal language philosophy
ordinary language philosophy

logical positivism.
+ problem: how to verify?
+ problem: not verifiable uninteresting?

lingustics. nominalism vs platonism.

ordinary language philosophy

formal vs natural language

+ we can construct statements
+ we know some statements to be a-priori true, a-priori false

types of universal:
+ types
+ properties
+ relations
\subsubsection{communication between intelligences}
modes of persuasion
+ logos - reason
+ ethos - character
+ pathos - appeal to emotion

types of talk
+ dialectic
+ didactic
+ debate
+ eristic
\subsubsection{Universals? something more like lingustics?}
universals:
+ things we talk about have properties. what is the nature of those properties?
+ these are universals
+ plato: theory of forms
+ platonic realism: universals, abstract objects "exist"

universal vs particular
abstract vs concrete
\subsection{Sort}
what is meant by fact. "the case"

foundationalism.
what is root of belief
basic beliefs
+ empiricism (senses)
+ self evident belifs


dualism vs monism

types of monism:
+ physicalism
+ idealism
+ neutral monism

cartesian dualism

mind:
+ functionalism? conciousness?

do we have free will?


aritotle question: are the heaven's teleological or material? implication

vitalism. non-physical element explains living vs non-living.

divine providence and telos. god the sustainer. natural order.
general providence, special providence

free will and compatibalism

"substance" monads etc


causation
final cause

absolute skepticism

heidegger, phenomenology

consciousness vs self-consciousness

various proofs for god?

rationalists: can rationally get at existance of god (decarte, others?)

heidegger method as different?
divine revelation as method?

how we interact with world? leaps of faith. kiirkegaard





Strauds arguement about validity even if claim of mental state is correct
Eg someone who believes in absolute good evil can genuinely believe that, but belief could be incorrect. Other person can hold different belief


theory of forms
+ exemplar vs thing itself
+ meter bar is exactly 1 meter, but the form of 1 meter doesn't necessary have length.
length(meter bar in paris) = 1 exactly.
length(length()) not defined?

but here we have at least a definition.
can there be a definition of the good, the just
we have this for some words. the large etc

is of identify
is of predication

what is there a form of? every type of thing?



logical positivism

semiotics
structuralism


Epistemology. We can't really verify atomic facts. Even our senses are in a sense in the past. Motivates move to probability, Bayes.


Berkeley and empiricism. Statements are like I perceive chair. How do move to chair is real. Can chair be real without perception


Transedental argument
Statement about mental state is true.
This implies other true statements about noumenal world
Generally considered debunked?
Role of skepticism here
Skeptical of mental beliefs


What statements on our mental state can we make?
Heidegger?
Dasein, being in the world




bacon new science. materialism.
decarte. solopsism.
kant unification. material world is projection of min. we experience phenomena world, not thing in it self. noumenal world



falsifiability

plato:
+ justified true beliefs



decarte: Discourse on the Method

francis bacon method. hume method. how they differ from each other, decarte

aristotle method



pre phenomenology. what can we say
cognito ergo sum
locke conception of mind
hume empericism

after descartes, spinoza? more rationalism.
plato before?

empericist history
+ aristotle
+ francis bacon
+ locke
+ hume

possible states of worlds
+ monism
+ neutral monism
+ materialism
+ mind body dualism
+ subjective idealism
+ dualism

subjective idealism:
+ bishop berkeley


quine: word and object


ontology definition
metaphysics definition

epistemology definition
scepticism definition

a priori, synthetic a-priori
phenomonon and noumenal world

gettier problems

francis bacon: novum organum

\subsection{big thing on phenomenology}
ego id, super ego

being and time?

conscious
subconsuious
\subsubsection{existance of other intelligences}
defined by negation? we are not the other?

the mind:
+ the problem of other minds

\subsection{empiricism}

problem of induction somewhere. just because we observe, and thing fits, doesn't mean we have law

description of empirical knowledge
reductionism, substance, holism, emergentism

aristotelian science:
+ gather data
+ 4 causes
+ describe forms? if wrong, change form? eg zoology?

bacon: can do experiemnts to test induction or form? eg aristotle and anvil/feather gravity
newton: simplify to allow for mathematical models. test those
bacon: do science ith goal of technology. knowledge is power
aristotle: classification can be seen as modern "hypothesis"

\subsubsection{observation stuff}
senses
+ through sensory input we can say more about a-postori, but complex how. falsifiability, positivity etc
+ we can create modedls in our heards to understand what we sense

empiricism:
+ a priori is maths/logic
+ a postiroi is after knowledge

direct realism vs indirect realism. do we experiencec world, or internal representation of it
\subsubsection{models h3}
page on:
+ moedlling observations, general concept
+ Epistemological rationalism
+ Epistemological skepticism and empiricism
+ assessing models
+ page on dualism. is mind part of model?
+ materialism. everything built up mechanically, computationally. physics
naive realism? scientific realism means no such thing as color per se

metaphysical realism. world exists independent of our perception of it. empiricism: some facts can be known through sensory inputs

problem of induction:
+ popper kuhn: falsifiability instead

idealism. mind is primary? mind builds simulcra of material world to interact with others?

logical positivism: only statements which are tautologies or verifiable

