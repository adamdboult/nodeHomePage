
\subsection{Introduction}


pyenv (and pyenv-virtualenv and pyenv-virtualenvwrapper)

pyenv-virtualenv allows use of virtualenv and pyenv

eg can run 
\begin{verbatim}
pyenv install 2.7.15
\end{verbatim}

\begin{verbatim}
pyenv uninstall 2.7.15
\end{verbatim}

to list installed versions
\begin{verbatim}
pyenv versions
\end{verbatim}

switch to version:

\begin{verbatim}
pyenv global 2.7.15
\end{verbatim}

\begin{verbatim}
pyenv local 2.7.15
\end{verbatim}

\begin{verbatim}
pyenv shell 2.7.15
\end{verbatim}

restore:

\begin{verbatim}
pyenv global system
\end{verbatim}

now python/pip will use the python version in question


\subsection{pipenv}


basically integrates adn replaces pip and venv

uses Pipfile

automatically creates virtual environments

\begin{verbatim}
pipenv install pandas
\end{verbatim}

automatically works with Pipfile and creates if needed

\begin{verbatim}
pipenv uninstall pandas
\end{verbatim}

\begin{verbatim}
pipenv run python main.py
\end{verbatim}

update lock file
\begin{verbatim}
pipenv lock
\end{verbatim}

install from lock
\begin{verbatim}
pipenv sync
\end{verbatim}

does lock and sync
\begin{verbatim}
pipenv update
\end{verbatim}

spawns shell in environment. exit with exit()
\begin{verbatim}
pipenv shell
\end{verbatim}

see dependency graph
\begin{verbatim}
pipenv graph
\end{verbatim}

