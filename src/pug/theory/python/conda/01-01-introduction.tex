
\subsection{Introduction and installing miniconda}

miniconda allows the creation of Python environments, including Python versions. Pip can be used inside these environments, and in addition other non-Python packages can be installed inside these environment.

Miniconda is not available on official repos.

\subsection{The "base" environment and creating conda environments}



list environments
\begin{verbatim}
conda info --envs
\end{verbatim}
environment.yml
conda env create -f environment.yml
+ creates from environment file

conda install packages
conda virtual environemnts

\begin{verbatim}
conda create --name <env_name>
conda create -n myenv python=3.9
conda install -n myenv scipy
conda install -n myenv scipy=0.17.3
conda create -n myenv python=3.9 scipy=0.17.3 astroid babel
conda create --prefix ./envs jupyterlab=3.2 matplotlib=3.5 numy=1.21
conda activate ./envs
conda deactivate
\end{verbatim}

\subsection{Installing packages in conda environments}
pip in conda

if we install pip when making environment, can install using pip inside conda
within environemtns
\begin{verbatim}
conda list (list packages)
conda install scipy
\end{verbatim}
update environment
\begin{verbatim}
conda env update --prefix ./env --file environment.yml --prune
\end{verbatim}

\subsection{Deleting conda environments}

\subsection{.condarc}
.condarc (has default packages for new environemnts?)

The condarc file contains info on which environment should
~/.condarc

\subsection{Setting up a mirror for conda}


page on local mirror using conda-mirror

\subsection{anaconda}
anaconda is miniconda but with additional default packages.
