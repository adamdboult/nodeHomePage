
\subsection{Introduction}

ints and floats

complex


type checking. na checking

define as specific data type in python. long int etc

null/na etc in python. nan. inf

overflows of int etc size in python
what happens if number gets too big?


\subsection{Arithmetic}

// in python is integer point division. / is floating point division

\subsection{References and copying on write}

\begin{verbatim}

a = 1000
b = a

\end{verbatim}

This gives \(b\) the same address as \(a\).

If we instead do the following then \(b\) will have a different reference after it is changed.

\begin{verbatim}
a = 1000
b = a
b = b + 1
\end{verbatim}


\subsection{Small integer caching}

Everything in Python is an object.

Normally, when a number is referenced an object for it is created.

If the number is a small integer (between \(-5\) and \(256\) inclusive) instead a reference to these objects are used.

\subsection{Casting}

\subsection{Booleans}

\subsection{Dynamic typing and lack of generics}

Don't need generics because of dynamic typing.

Uses ducktyping.

\subsection{Type hints}

\begin{verbatim}
x: int = 1
y: float = 2
\end{verbatim}

