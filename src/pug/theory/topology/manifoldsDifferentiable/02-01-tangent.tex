
\subsection{Tangent space and tangent vectors}

Take a topological space: can all subsets in the toplogy be mapped to \(n\) dimensional space? if so, manifold

For this we need openness: a graph for example isn't open and so isn't a manifold

We also need the same number of dimensions at each point

Isn't always the case. eg two circles conneceted by a line is not a manifold. it's 2d in circles, 1d on line (and 3d at connections)

We have a homeomorphism from each point in the toplogy to an n dimensional coordinate system

We also have homeomorphisms of transformation maps, between different points on the topology

The vector space from the homeomorphism is tangent to the manifold at that point. the set of all tangents forms a tangent space

Interior: \(M\); boundry \(\delta M\)
Tangent on a manifold:

The tangent space of manifold \(M\) at point \(p\) is denoted \(TM_p\).

If we have a normal field

\(v=v^ie_i\)

Then we can differentiate wrt a direction \(x\).

\(\dfrac{\delta }{\delta x}v=\dfrac{\delta }{\delta x}v^ie_i\)

\(\dfrac{\delta }{\delta x}v=e_i\dfrac{\delta v^i}{\delta x}\)

Because the basis does not change.

If the basis does change we instead have:

\(\dfrac{\delta }{\delta x}v=\dfrac{\delta }{\delta x}v^ie_i\)

\(\dfrac{\delta }{\delta x}v=e_i\dfrac{\delta v^i}{\delta x}+v^i\dfrac{\delta e_i }{\delta x}\)

General point. basis can vary across manifold

After this basis diff

\subsubsection{Tangent space as vector bundle}

\subsubsection{Christoffel symbols (page)}

Christoffel symbols are connections.

\subsubsection{The torsion tensor (own page)}

Torsion tensor is

\(T_{jk}^i=\Gamma^i_{jk}-\Gamma^i_{kj}\)


If torsion is \(0\), then the connection is symmetric.

\subsubsection{Basis of tangent space}

We can use as the basis for tangent space:

\(\{\dfrac{(\delta }{\delta x^1})_p,(\dfrac{\delta }{\delta x^2})_p,...\}\)


This means we can write a tangent vector as:

\(u=u^i(\dfrac{\delta }{\delta x^i})_p\)

\subsubsection{Basis of contant space}

We can use as the basis for the contangent space:

\(\{dx^1,dx^2,...\}\)

\subsubsection{Metric on the tangent space (to Riemann)}

\subsubsection{Basis of metric (to Riemann)}

The metric depends on the basis too:

\(g_{ij}(p)=g((\dfrac{\delta }{\delta x^i})_p,(\dfrac{\delta }{\delta x^j})_p)\)

The metric on two tangent vectors is defined on the components.

\(g=g_{ij}(p)u^iv^j\)

