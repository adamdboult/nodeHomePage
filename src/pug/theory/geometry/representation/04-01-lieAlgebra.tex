
\subsection{Lie algebra}

Lie groups have symmetries. We can consider only the infintesimal symmetries.

For example the unit circle has many symmetries, but we can consider only those which rotate infintesimally.

\subsubsection{Example}

Take a continous group, such as \(U(1)\). Its Lie algebra is all matrices such that their exponential is in the Lie group.

\(\mathfrak{u}(1)=\{X\in \mathbb {C}^{1\times 1}|e^{tX}\in U(1) \forall t\in \mathbb{R}\}\)

This is satisfied by the matrices where \(M=-M^*\). Note that this means the diagonals are all \(0\).

\subsubsection{Scale of specific Lie algebra matrices doesn't matter}

Because of \(t\).

\subsubsection{Commutation of Lie group algebra}

Consider two members of the Lie algebra: \(A\) and \(B\). The commutator is:

\(A\).

The corresponding Lie group member is:

\(e^{t(A+B)}=e^{tA}e^{tB}\)

While the Lie group multiplication may not commute, the corresponding addition of the Lie algebra does.

