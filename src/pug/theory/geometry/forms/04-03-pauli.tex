
\subsection{Pauli matrices}

Pauli matrices are \(2\times 2\) matrices which are unitary and hermitian.

That is, \(P^*=P^{-1}\).

And \(P^*=P\).

\subsubsection{The Pauli matrices}

The matrices are:

\(\sigma_1 =\begin{bmatrix} 0&1  \\ 1&0  \end{bmatrix}\)

\(\sigma_2 =\begin{bmatrix} 0&-i \\ i&0  \end{bmatrix}\)

\(\sigma_3 =\begin{bmatrix} 1&0  \\ 0&-1 \end{bmatrix}\)

The identity matrix is often considered alongside these as:

\(\sigma_0 =\begin{bmatrix} 1&0  \\ 0&1  \end{bmatrix}\)

\subsubsection{Pauli matrices are their own inverse}

\(\sigma_i^2 =\sigma_i\sigma_i\)

\(\sigma_i^2 =\sigma_i\sigma_i^*\)

\(\sigma_i^2 =\sigma_i\sigma_i^{-1}\)

\(\sigma_i^2 =I\)

\subsubsection{Determinants and trace of Pauli matrices}

\(\det \sigma_i =-1\)

\(Tr (\sigma_i) =0\)

As the sum of eigenvalues is the trace, and the product is the determinant, the eigenvalues are \(1\) and \(-1\).

