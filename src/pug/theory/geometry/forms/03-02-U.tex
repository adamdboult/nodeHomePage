\subsection{Unitary groups \(U(n, F)\)}

\subsubsection{Metric preserving transformations for sesquilinear forms}

For bilinear forms, the transformations which preserved metrics were:

\(P^T=P^{-1}\)

For sesquilinear they are different:

\(u^*Mv\)

\((Pu)^*M(Pv)\)

\(u^*P^*MPv\)

So we want the matrices where:

\(P^*MP=M\)

\subsubsection{The unitary group}

The unitary group is where \(M=I\)

\(P^*P=I\)

\(P^*=P^{-1}\)

We refer to these using \(U\) instead of \(P\).

\(U^*=U^{-1}\)

\subsubsection{Parameters of the unitary group}

The unitary group depends on the dimension of the vector space, and the underlying field. So we can have:

\begin{itemize}
\item \(U(n, R)\); and
\item \(U(n, C)\).
\end{itemize}

\subsubsection{We generally refer only to the complex}

For the \(U(n, R)\) we have:

\(U^*=U^{-1}\)

\(U^T=U^{-1}\)

This is the condition for the orthogonal group, and so we would instead write \(O(n)\).

As a result, \(U(n)\) refers to \(U(n,C)\).
\subsubsection{\(U(1)\): The circle group}

