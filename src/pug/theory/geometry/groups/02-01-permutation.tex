
\subsection{Permutations and the symmetric group}

A permutation is defined as a bijection from a set to itself.

For a set of size \(n\), the number of permutations is \(n!\). This is because there are \(n\) possibilities for the first item, \(n-1\) for the second and so on.

\subsubsection{The symmetric group}

The set of all permutations forms a group, the symmetric group. This forms a group because:

\begin{itemize}
\item There is an identity element
\item Each combination of permutations is also in the group.
\item Each permutation has an inverse in the group.
\end{itemize}

\subsubsection{Permutation groups}

A subgroup of the symmetric group is called a permutation group.


