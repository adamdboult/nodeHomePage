
\subsection{Einstein summation convention}

A vector can be written as a sum of its components.

\(v=\sum_i e_i v^i\)

The Einstein summation convention is to remove the \(\sum_i \) symbols where they are implicit.

For example we would instead write the vector as:

\(v=e_iv^i\)

\subsubsection{Adding vectors}

\(v+w=(\sum_i e_i v^i)+(\sum_i f_iw^i)\)

\(v+w=\sum_i (e_iv^i+f_iw^i)\)

\(v+w=e_iv^i+f_iw^i\)

If the bases are the same then:

\(v+w=e_i (v^i+w^i)\)

\subsubsection{Scalar multiplication}

\(cv=c\sum_ie_iv^i\)

\(cv=\sum_i ce_iv^i\)

\(cv=ce_iv^i\)

\subsubsection{Matrix multiplication}

\(AB_{ik}=\sum_jA_{ij}B_{jk}\)

\(AB_{ik}=A_{ij}B_{jk}\)

\subsubsection{Inner products}

\(\langle v, w\rangle =\langle \sum_i e_iv^i, \sum_j f_jw^j\rangle \)

\(\langle v, w\rangle =\sum_iv^i\langle  e_i, \sum_j f_iw^j\rangle \)

\(\langle v, w\rangle =\sum_i \sum_jv^i\overline {w^j}\langle e_i, f_j\rangle\)

If the two bases are the same then:

\(\langle v, w\rangle =\sum_i \sum_jv^i\overline {w^j}\langle e_i, e_j\rangle\)

We can define the metric as:

\(g_{ij}:=\langle e_i,e_j\rangle \)

\(\langle v, w\rangle =v^i\overline {w^j}g_{ij}\)

