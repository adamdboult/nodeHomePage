
\subsection{Market definition}

\subsubsection{Defining the relevant market}

We want to see what market the monopolist can exert a profitable increase in price. This my not be all of their offerings.

The relevant market includes the good offered by the monopolist, along with relevant competitors in supply and demand

\subsubsection{What is the product?}

If we are considering a commodity it is easy to see that, say, steel supplied by one firm is comparable to that supplied by another. For other goods this is more complex.

For example, does Google provide search services, making it highly dominant? Or does it in fact provide advertising services for a small number of searches aimed at purchases? In the latter case it is a closer good to Amazon or Ebay.

\subsubsection{Supply side substitution}

If a price rise from a firm caused other firms to increase supply, this is relevant.

Would others be able to raise output?

Would others be able to enter the market?

\subsubsection{Demand side substitution}

If a price rise from a firm causes buyers to react, price increases will be less rewarding for the firm.

Assessment: Price elasticity of demand

\subsubsection{Geographic market}

May be many players but fewer locally. Threat of entry may still be a key motivation for price setting. Supply side substitution.

\subsubsection{Vertical integration}

Benefits: no double mark up.

2 monopolists both exert monopoly power, more deadweight loss. Integration solves this.

Contract theory argument. Don’t want to be held if need change

Costs: can keep out downstream competitors

Facilitating collusion? Vertical integration allows upstream to monitor downstream price from their customer.

Restoring monopoly power?

Problem: to what extent can monopoly upstream abuse their power? One option is high price, but they could also do 2 part tariffs

But 2 part tariffs are unstable, as there is an incentive for the provider to offer the last downstream one a lower marginal cost.

Vertical integration then restores this power.

