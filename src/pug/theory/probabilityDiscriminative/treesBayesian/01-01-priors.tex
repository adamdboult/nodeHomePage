
\subsection{Priors of trees}

\subsubsection{Priors for simple trees}

We can define a tree as a set of nodes: \(T\).

For each node we define a splitting variable \(k\) and a splitting threshold \(r\).

Our prior is \(P(T, k, r)\).

We split this up to:

\(P(T, k, r)=P(T)P(k, r|T)\)

\(P(T, k, r)=P(T)P(k|T)P(r|T, k)\)

So we want to estimate:

\begin{itemize}
\item \(P(T)\) - The number of nodes.
\item \(P(k|T)\) - Which variables we split by, given the tree size.
\item \(P(r|T, k)\) - The cutoff, given the tree size and the variables we are splitting by.
\end{itemize}

\subsubsection{Priors for mixed trees}

If at the leaf we have a parametric model, our prior is instead:

\(P(T, k, r, \theta )=P(T)P(k|T)P(r|T, k)P(\theta | T, k, r)\)

We then need to additionally estimate \(P(\theta | T, k, r)\).

