
\subsection{The Naive Bayes posterior}

\subsubsection{Bayes theorem}

Consider Bayes' theorem

\(P(y|x_1,x_2,...,x_n)=\dfrac{P(x_1,x_2,...,x_n|y)P(y)}{P(x_1,x_2,...,x_n)}\)

Here, \(y\) is the label, and \(x_1,x_2,...,x_n\) is the evidence. We want to know the probability of each label given evidence.

The denominator, \(P(x_1,x_2,...,x_n)\), is the same for all, so we only need to identify:

\(P(y|x_1,x_2,...,x_n)\propto P(x_1,x_2,...,x_n|y)P(y)\)

\subsubsection{The assumption of Naive Bayes}

We assume each \(x\) is independent. Therefore:

\(P(x_1,x_2,...,x_n|y)=P(x_1|y)P(x_2|y)...P(x_n|y)\)

\(P(y|x_1,x_2,...,x_n)\propto P(x_1|y)P(x_2|y)...P(x_n|y)P(y)\)

