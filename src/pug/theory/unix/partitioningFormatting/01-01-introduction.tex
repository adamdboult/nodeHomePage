
\subsection{Introduction}

three options for partitioning are fdisk, gdisk and parted. parted generally seems the better option.

\subsection{fdisk}

fdisk is designed with MBR in mind, but later versions have some GPT support:

\begin{itemize}
  \item fdisk -l (list things in /dev/) (or can use lsblk)
  \item fdisk /dev/sda (or whatever correct device is)
  \item this opens dialogue:
  \begin{itemize}
    \item "d" to delete partitions
    \item create a new table, using MBR or GPT
    \item create partitions (can press "n" for new)
    \item make one bootable
    \item "w" to write"
  \end{itemize}
\end{itemize}

\subsection{parted}

partitioning using parted:

\begin{itemize}
  \item supports MBR and GPT
  \item different to fdisk? needed if drives over 2TB?
  \item parted -l (list things in /dev/) (or can use lsblk)
  \item parted /dev/sda (or whatever correct device is)
  \item this opens dialogue:
  \begin{itemize}
    \item see status with "print"
    \item type "quit" when done
    \item make gpt using "mklabel gpt"
    \item make mbr using "mklabel msdos"
    \item make partitions: "mkpart". is interactive
    \item make one bootable? "set <partition> boot on"
  \end{itemize}
\end{itemize}
  
\subsection{gdisk}

gdisk is similar to fdisk but aimed at GPT.

\subsection{What partitions can be used?}

root partition
if uefi, efi system partition (boot partition). also need boot partition if doing LVM or encryption on BIOS
swap, though this is discussed later


