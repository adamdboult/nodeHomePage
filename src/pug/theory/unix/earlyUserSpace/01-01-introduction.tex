
\subsection{Introduction}

Doing this encrypts the root partition.

The boot partition must remain unencrypted for this.

unified kernel image (UKI)
+ can be loaded by UEFI
+ contains all of: uefi stub loader (eg systemd-stub); kernel command line; microcode; initramfs image; kernel image; splash screen


/etc/mkinitcpio.conf
/etc/mkinitcpio.d/


/boot/initramfs-%

dracut (context of initrd)

initramfs
+ can be used to make logical volumes root. also used for encrpyted file systems as discussed later.

logical volume manager (LVM)


dm-crypt is kernel thing
+ cryptsetup and cryptmount are frontends for user space
+ dm-crypt encrypts a loop file? different to full disk encryption?

page on early user space/initramfs
+ after kernel is loaded, before init process is started
+ loads a temporary root file system
+ initrd and initramfs are alternative implementation
  * initrd
    1. mounts /dev/ram
    2. runs /linuxrc
    3. when done prep is done so runs /sbin/init
  * initramfs
    1. newer
+ used for logical partitions and encrypted drives


mkinitcpio on the early user space

done after loading kernel, before init process
+ used for eg encrpying swap disk and allowing reload on boot.
+ but not for encrpyting root drive right? if encrypted file system, how can we load the kernel? main root partition must be unencrpted during boot right?



  * encrypting boot partition using GRUB
  * plain dm-crypt
  * dm-crypt + LUKS
    1. LUKS2 encrpyed root partition (/), unencrypted (/boot) partition
  * approaches using LVM:
    1. LUKS on LVM
    2. LVM on LUKS
