
\subsection{Introduction}

three options for partitioning are fdisk, gdisk and parted. parted generally seems the better option.


root partition
if uefi, efi system partition (boot partition). also need boot partition if doing LVM or encryption on BIOS
swap, though this is discussed later

\subsection{fdisk}

Same name as DOS fdisk.

fdisk is designed with MBR in mind, but later versions have some GPT support:

\begin{itemize}
  \item fdisk -l (list things in /dev/) (or can use lsblk)
  \item fdisk /dev/sda (or whatever correct device is)
  \item this opens dialogue:
  \begin{itemize}
    \item "d" to delete partitions
    \item create a new table, using MBR or GPT
    \item create partitions (can press "n" for new)
    \item make one bootable
    \item "w" to write"
  \end{itemize}
\end{itemize}

\subsection{cfdisk}

Curses ndisk

\subsection{lsblk}

See devices in /dev/
\subsection{fsck}

Fix file system.

\subsection{gdisk}

gdisk is similar to fdisk but aimed at GPT (is it part of util-linux though?)

\subsection{wipefs}

