\documentclass[oneside]{book}

\usepackage{amsmath, amssymb}
\usepackage{pgfplots}
\usepackage{parskip}

\begin{document}

\author{Adam Boult (www.bou.lt)}
\title{Unix}
\maketitle

\setcounter{tocdepth}{0}
\tableofcontents

\chapter*{Preface}
\addcontentsline{toc}{chapter}{Preface}

This is a live document, and is full of gaps, mistakes, typos etc.



%\part{Comparison of machine languages}
%Reduced Instruction Set Computer (RISC)
%Complex Instruction Set Computer (CISC)

%Main ones
%IBM System/360
%Motorola 6800
%Motorola 68000
%Zilog Z80 (gameboy based on this, ppu separate). based on intel 8080
%Intel 8086 (x86)
%ARM
%PowerPC
%MIPS

\part{OS basics}

\include{batchProcessing}
\include{interrupts}
\include{concurrencyControl}
\include{bios}


%printk (print in kernel world, different to printf in user world)
%lsmod to list kernel modules
%insmod to insert kernel module
% dmesg to see kernel messages
% module_init() and module_exit() to specify functions to run when module loaded and unloaded
% #include <linux/module.h> to use module_init() and module_exit()
% #include <linux/init.h> to use __init and __exit
% #include <linux/fs.h> to use file_operations struct

% /boot ?
% /boot/vmlinuz-... (kernel)

\part{Linux file systems}
%doing before devices because file systems can be just in RAM - no devices. eg linux distros which just load from flash drive

% trailing & and nohup somewhere?
% comm and sort somewhere

\include{fileSystems}
\include{fileSystemsLinux}
\include{fileSystemsNavigation}
\include{fileSystemsChanging}
\include{shellReading}
\include{fileSystemsCommands}
\include{sedTr}
\include{thompsonShellXargs}
\include{edVi}
\include{checkSums}
\include{compression}
\include{jails}

\part{Devices}
% /dev folder. /dev/null, /dev/zero, /dev/random.  % inc /dev/random, /dev/urandom, security risks (because pseudo)

% character devices
% block devices
\include{basicShell}
\include{keyboardLocales}

\part{Users and permissions}
% passwd(give root a password?)
%useradd, userdel[what happens to files with user as owner?],logout. use passwd to set passwords. whoami. /etc/passwd is file with info on users. also /etc/shadow (which contains hash of password). pinky
%chown, su[run as different user], sudo command, pinky/finger
% /root is root home directory. /home/[user] folders. 
% /sbin for binaries only root can run
% /usr/bin and /usr/sbin    for binaries for all users. not aimed for use by system admin?.
%\include{users}
%groups. usermod to add user to group. users have primary group associated with just them, usually same name. can change using usermod. groupadd, groupdel, groupmod, 777 etc. what happens to file when group deleted? chgrp. command groups shows what groups a user is in
%cont groups gpasswd to set passwords for groups. /etc/groups, /etc/gshadow
%masks for permissions. umask
%run elevated temp using setuid setgid
%\section{Introduction}


\section{Introduction}



\subsection{Endomorphism}

An endomorphism is one where the domain and codomain are the same.

The following are endomorphisms:

\begin{itemize}
\item \(f(x)=0\)
\item \(f(x)=x\)
\item \(f(x)=cx\)
\end{itemize}

The following are not endomorphisms

\begin{itemize}
\item Converting natural numbers to integers
\item \(f(x)=x+1\)
\end{itemize}

\subsection{Automorphism}

An endomorphism which is also an isomorphism

The following are automorphisms:
\begin{itemize}
\item \(f(x)=x\)
\item \(f(x)=cx\)
\end{itemize}

The following are not automorphisms

\begin{itemize}
\item \(f(x)=0\)
\item \(f(x)=x+1\)
\item Converting natural numbers to integers
\end{itemize}




\subsection{Endomorphism}

An endomorphism is one where the domain and codomain are the same.

The following are endomorphisms:

\begin{itemize}
\item \(f(x)=0\)
\item \(f(x)=x\)
\item \(f(x)=cx\)
\end{itemize}

The following are not endomorphisms

\begin{itemize}
\item Converting natural numbers to integers
\item \(f(x)=x+1\)
\end{itemize}

\subsection{Automorphism}

An endomorphism which is also an isomorphism

The following are automorphisms:
\begin{itemize}
\item \(f(x)=x\)
\item \(f(x)=cx\)
\end{itemize}

The following are not automorphisms

\begin{itemize}
\item \(f(x)=0\)
\item \(f(x)=x+1\)
\item Converting natural numbers to integers
\end{itemize}




\subsection{Defining groups}

\subsubsection{Magma}

A magma, or groupoid, is a set with a single binary operation.

These can be defined as an ordered pair \((s,\odot )\) where \(s\) is the set, and \(\odot \) is the binary operation.

If \(a\) and \(b\) are in \(s\), then \(a\odot b\) is also in \(s\).

The following are magmas:

\begin{itemize}
\item Natural numbers and addition
\item \(n\times n\) matrices with determinants other than \(0\)
\item Natural numbers above \(0\) and addition
\item Integers and addition
\item Rational numbers and division
\item \(\{-1, 1\}\) and multiplication
\end{itemize}

The following are not magmas:

\begin{itemize}
\item Natural numbers up to \(10\) and addition
\end{itemize}

\subsubsection{Semigroup}

A semigroup is a magma whose binary operation is associative.

The following are semigroups:

\begin{itemize}
\item Natural numbers and addition
\item \(n\times n\) matrices with determinants other than \(0\)
\item Natural numbers above \(0\) and addition
\item Integers and addition
\end{itemize}

The following are not semigroups:

\begin{itemize}
\item \(\{-1, 1\}\) and multiplication
\item Rational numbers and division
\item Natural numbers up to \(10\) and addition
\end{itemize}

\subsubsection{Monoid}

A monoid is a semigroup with an identity element

The following are monoids:

\begin{itemize}
\item Natural numbers and addition
\item \(n\times n\) matrices with determinants other than \(0\)
\item Integers and addition
\item \(\{-1, 1\}\) and multiplication
\end{itemize}

The following are not monoids:

\begin{itemize}
\item Natural numbers above \(0\) and addition
\item Rational numbers and division
\item Natural numbers up to \(10\) and addition
\end{itemize}

\subsubsection{Group}

A group is a monoid where there is an inverse operation for the binary operation.

The following are groups:

\begin{itemize}
\item Integers and addition
\item \(n\times n\) matrices with determinants other than \(0\)
\item \(\{-1, 1\}\) and multiplication
\end{itemize}

The following are not groups:

\begin{itemize}
\item Natural numbers above \(0\) and addition
\item Rational numbers and division
\item Natural numbers and addition
\item Natural numbers up to \(10\) and addition
\end{itemize}



\subsection{Defining groups}

\subsubsection{Magma}

A magma, or groupoid, is a set with a single binary operation.

These can be defined as an ordered pair \((s,\odot )\) where \(s\) is the set, and \(\odot \) is the binary operation.

If \(a\) and \(b\) are in \(s\), then \(a\odot b\) is also in \(s\).

The following are magmas:

\begin{itemize}
\item Natural numbers and addition
\item \(n\times n\) matrices with determinants other than \(0\)
\item Natural numbers above \(0\) and addition
\item Integers and addition
\item Rational numbers and division
\item \(\{-1, 1\}\) and multiplication
\end{itemize}

The following are not magmas:

\begin{itemize}
\item Natural numbers up to \(10\) and addition
\end{itemize}

\subsubsection{Semigroup}

A semigroup is a magma whose binary operation is associative.

The following are semigroups:

\begin{itemize}
\item Natural numbers and addition
\item \(n\times n\) matrices with determinants other than \(0\)
\item Natural numbers above \(0\) and addition
\item Integers and addition
\end{itemize}

The following are not semigroups:

\begin{itemize}
\item \(\{-1, 1\}\) and multiplication
\item Rational numbers and division
\item Natural numbers up to \(10\) and addition
\end{itemize}

\subsubsection{Monoid}

A monoid is a semigroup with an identity element

The following are monoids:

\begin{itemize}
\item Natural numbers and addition
\item \(n\times n\) matrices with determinants other than \(0\)
\item Integers and addition
\item \(\{-1, 1\}\) and multiplication
\end{itemize}

The following are not monoids:

\begin{itemize}
\item Natural numbers above \(0\) and addition
\item Rational numbers and division
\item Natural numbers up to \(10\) and addition
\end{itemize}

\subsubsection{Group}

A group is a monoid where there is an inverse operation for the binary operation.

The following are groups:

\begin{itemize}
\item Integers and addition
\item \(n\times n\) matrices with determinants other than \(0\)
\item \(\{-1, 1\}\) and multiplication
\end{itemize}

The following are not groups:

\begin{itemize}
\item Natural numbers above \(0\) and addition
\item Rational numbers and division
\item Natural numbers and addition
\item Natural numbers up to \(10\) and addition
\end{itemize}



\subsection{Subgroups}

A subgroup of a group is a subset of a group, which also forms a group with the same element.

For example all even numbers are a subgroup of the addition group of integers.



\subsection{Subgroups}

A subgroup of a group is a subset of a group, which also forms a group with the same element.

For example all even numbers are a subgroup of the addition group of integers.



\subsection{Abelian groups}

A commutative group, that is where \(a\odot b=b\odot a\).

The following are abelian groups:

\begin{itemize}
\item Integers and addition
\item \(\{-1, 1\}\) and multiplication
\end{itemize}

The following are not abelian groups:

\begin{itemize}
\item Natural numbers above \(0\) and addition
\item Rational numbers and division
\item Natural numbers and addition
\item Natural numbers up to \(10\) and addition
\item \(n\times n\) matrices with determinants other than \(0\)
\end{itemize}



\subsection{Abelian groups}

A commutative group, that is where \(a\odot b=b\odot a\).

The following are abelian groups:

\begin{itemize}
\item Integers and addition
\item \(\{-1, 1\}\) and multiplication
\end{itemize}

The following are not abelian groups:

\begin{itemize}
\item Natural numbers above \(0\) and addition
\item Rational numbers and division
\item Natural numbers and addition
\item Natural numbers up to \(10\) and addition
\item \(n\times n\) matrices with determinants other than \(0\)
\end{itemize}



\subsection{Group order}

For finite groups, each element \(e\) has:

\(e^n=I\)

For some \(n\in \mathbb{N}\)

Where \(I\) is the identity element.

The order of the group is the smallest value of \(n\) such that that holds for all elements.

For example in the multiplicative group \(G=\{-1,1\}\) the order is \(2\).

Or:

\(|G|=2\)

Additionally

\(|-1|=2\)

\(|1|=1\)



\subsection{Group order}

For finite groups, each element \(e\) has:

\(e^n=I\)

For some \(n\in \mathbb{N}\)

Where \(I\) is the identity element.

The order of the group is the smallest value of \(n\) such that that holds for all elements.

For example in the multiplicative group \(G=\{-1,1\}\) the order is \(2\).

Or:

\(|G|=2\)

Additionally

\(|-1|=2\)

\(|1|=1\)



\section{Creating groups}



\subsection{Permutations and the symmetric group}

A permutation is defined as a bijection from a set to itself.

For a set of size \(n\), the number of permutations is \(n!\). This is because there are \(n\) possibilities for the first item, \(n-1\) for the second and so on.

\subsubsection{The symmetric group}

The set of all permutations forms a group, the symmetric group. This forms a group because:

\begin{itemize}
\item There is an identity element
\item Each combination of permutations is also in the group.
\item Each permutation has an inverse in the group.
\end{itemize}

\subsubsection{Permutation groups}

A subgroup of the symmetric group is called a permutation group.




\subsection{Permutations and the symmetric group}

A permutation is defined as a bijection from a set to itself.

For a set of size \(n\), the number of permutations is \(n!\). This is because there are \(n\) possibilities for the first item, \(n-1\) for the second and so on.

\subsubsection{The symmetric group}

The set of all permutations forms a group, the symmetric group. This forms a group because:

\begin{itemize}
\item There is an identity element
\item Each combination of permutations is also in the group.
\item Each permutation has an inverse in the group.
\end{itemize}

\subsubsection{Permutation groups}

A subgroup of the symmetric group is called a permutation group.




\subsection{Morphism}

Morphisms are functions which preserve the relationships between members of a set, and specified functions.

That is, if:

\(a\odot b=c\)

Then \(f(x)\) is morphism if:

\(f(a)\odot f(b)=f(a\odot b)\)

Here we discuss morphisms in the context of groups, but we can define morphisms for sets with more than one function, for example with addition and multiplication.

Morphisms are also known as homomorphisms.

The following are morphisms of the additive group of integers.

Where we refer to \(c\), \(c\ne 0\in \mathbb{I}\).

\begin{itemize}
\item \(f(x)=0\)
\item \(f(x)=x\)
\item \(f(x)=cx\)
\item Converting natural numbers to integers
\end{itemize}

The following are not morphisms

\begin{itemize}
\item \(f(x)=x+1\)
\end{itemize}


\subsubsection{Isomorphism}

An isomorphism is a morphism which has an inverse.

This means the function is bijective.

The following are isomorphisms:

\begin{itemize}
\item \(f(x)=x\)
\item \(f(x)=cx\)
\item Converting natural numbers to integers
\end{itemize}

The following are not isomorphisms

\begin{itemize}
\item \(f(x)=0\)
\item \(f(x)=x+1\)
\end{itemize}

\subsubsection{Endomorphism}

An endomorphism is one where the domain and codomain are the same.

The following are endomorphisms:

\begin{itemize}
\item \(f(x)=0\)
\item \(f(x)=x\)
\item \(f(x)=cx\)
\end{itemize}

The following are not endomorphisms

\begin{itemize}
\item Converting natural numbers to integers
\item \(f(x)=x+1\)
\end{itemize}

\subsubsection{Automorphism}

An endomorphism which is also an isomorphism

The following are automorphisms:
\begin{itemize}
\item \(f(x)=x\)
\item \(f(x)=cx\)
\end{itemize}

The following are not automorphisms

\begin{itemize}
\item \(f(x)=0\)
\item \(f(x)=x+1\)
\item Converting natural numbers to integers
\end{itemize}

\subsubsection{Monomorphism}

A morphism which is injective. That is:

\(f(a)=f(b)\rightarrow a=b\)

The following are monomorphisms:

\begin{itemize}
\item \(f(x)=x\)
\item \(f(x)=cx\)
\item Converting natural numbers to integers
\end{itemize}

The following are not monomorphisms:

\begin{itemize}
\item \(f(x)=0\)
\item \(f(x)=x+1\)
\end{itemize}



\subsection{Morphism}

Morphisms are functions which preserve the relationships between members of a set, and specified functions.

That is, if:

\(a\odot b=c\)

Then \(f(x)\) is morphism if:

\(f(a)\odot f(b)=f(a\odot b)\)

Here we discuss morphisms in the context of groups, but we can define morphisms for sets with more than one function, for example with addition and multiplication.

Morphisms are also known as homomorphisms.

The following are morphisms of the additive group of integers.

Where we refer to \(c\), \(c\ne 0\in \mathbb{I}\).

\begin{itemize}
\item \(f(x)=0\)
\item \(f(x)=x\)
\item \(f(x)=cx\)
\item Converting natural numbers to integers
\end{itemize}

The following are not morphisms

\begin{itemize}
\item \(f(x)=x+1\)
\end{itemize}


\subsubsection{Isomorphism}

An isomorphism is a morphism which has an inverse.

This means the function is bijective.

The following are isomorphisms:

\begin{itemize}
\item \(f(x)=x\)
\item \(f(x)=cx\)
\item Converting natural numbers to integers
\end{itemize}

The following are not isomorphisms

\begin{itemize}
\item \(f(x)=0\)
\item \(f(x)=x+1\)
\end{itemize}

\subsubsection{Endomorphism}

An endomorphism is one where the domain and codomain are the same.

The following are endomorphisms:

\begin{itemize}
\item \(f(x)=0\)
\item \(f(x)=x\)
\item \(f(x)=cx\)
\end{itemize}

The following are not endomorphisms

\begin{itemize}
\item Converting natural numbers to integers
\item \(f(x)=x+1\)
\end{itemize}

\subsubsection{Automorphism}

An endomorphism which is also an isomorphism

The following are automorphisms:
\begin{itemize}
\item \(f(x)=x\)
\item \(f(x)=cx\)
\end{itemize}

The following are not automorphisms

\begin{itemize}
\item \(f(x)=0\)
\item \(f(x)=x+1\)
\item Converting natural numbers to integers
\end{itemize}

\subsubsection{Monomorphism}

A morphism which is injective. That is:

\(f(a)=f(b)\rightarrow a=b\)

The following are monomorphisms:

\begin{itemize}
\item \(f(x)=x\)
\item \(f(x)=cx\)
\item Converting natural numbers to integers
\end{itemize}

The following are not monomorphisms:

\begin{itemize}
\item \(f(x)=0\)
\item \(f(x)=x+1\)
\end{itemize}



\subsection{Generating sets}

We can define a group through a generating set and an operation.

And define the group as \(G=< S >\)

\subsection{Finite groups}

Consider the set of natural numbers and addition modulo 4. This forms a group containing:

\(\{0,1,2,3\}\)

This can be written as \(Z_4\) or more generally as \(Z_n\), or \(Z/nZ\).



\subsection{Generating sets}

We can define a group through a generating set and an operation.

And define the group as \(G=< S >\)

\subsection{Finite groups}

Consider the set of natural numbers and addition modulo 4. This forms a group containing:

\(\{0,1,2,3\}\)

This can be written as \(Z_4\) or more generally as \(Z_n\), or \(Z/nZ\).



\section{Group operations}



\subsection{The group commutator}

The group commutator is:

\([a,b]=a^{-1}b^{-1}ab\)

If the group is abelian then \([a,b]=0\). The group commutator is a measure of how non-abelian the group is.

This has the following properties:

\begin{itemize}
\item Alternativity: \([A,A]=I\)
\end{itemize}



\subsection{The group commutator}

The group commutator is:

\([a,b]=a^{-1}b^{-1}ab\)

If the group is abelian then \([a,b]=0\). The group commutator is a measure of how non-abelian the group is.

This has the following properties:

\begin{itemize}
\item Alternativity: \([A,A]=I\)
\end{itemize}



\subsection{The direct product of groups}

If we have two groups \(G\) and \(H\) we can form new group \(G\times H\).

For every \(g\in G\) and \(h\in H\) there is \((g,h)\in G\times H\).

The binary operation we have is:

\((g_1, h_1)(g_2,h_2)=(g_1g_2,h_1,h_2)\)



\subsection{The direct product of groups}

If we have two groups \(G\) and \(H\) we can form new group \(G\times H\).

For every \(g\in G\) and \(h\in H\) there is \((g,h)\in G\times H\).

The binary operation we have is:

\((g_1, h_1)(g_2,h_2)=(g_1g_2,h_1,h_2)\)



\section{Specifc groups}



\subsection{The trivial group}

The trivial group is the group with just the identity member \(I\).



\subsection{The trivial group}

The trivial group is the group with just the identity member \(I\).



\subsection{The infinite cyclic group (\(Z\))}

\subsubsection{The additive group of integers}

\subsubsection{Generating cyclic groups}

We can generate a group with a single element, it is a cyclic group.

For example, we can define a group \(G=<1>\) which gives us the additive group of integers.

\subsubsection{Infinite cyclic groups are isomorphic to the additive group of integers}

More generally, any infinite cyclic group is isomorphic to the additive group of integers.

Consider the multiplicative group of \(< i >\).

This contains \(\{1,-1,i,-i \} \).

This is also automorphic to the natural number and modulo addition group above.

We can define finite cyclic groups of size \(n\) using the generating element \(z^{\dfrac{1}{n}}\). This is isomorphic to the general cyclic group \(C_n\), and to \(Z/nZ\).

\subsubsection{Abelian cyclic groups}

Cyclic groups are abelian.



\subsection{The infinite cyclic group (\(Z\))}

\subsubsection{The additive group of integers}

\subsubsection{Generating cyclic groups}

We can generate a group with a single element, it is a cyclic group.

For example, we can define a group \(G=<1>\) which gives us the additive group of integers.

\subsubsection{Infinite cyclic groups are isomorphic to the additive group of integers}

More generally, any infinite cyclic group is isomorphic to the additive group of integers.

Consider the multiplicative group of \(< i >\).

This contains \(\{1,-1,i,-i \} \).

This is also automorphic to the natural number and modulo addition group above.

We can define finite cyclic groups of size \(n\) using the generating element \(z^{\dfrac{1}{n}}\). This is isomorphic to the general cyclic group \(C_n\), and to \(Z/nZ\).

\subsubsection{Abelian cyclic groups}

Cyclic groups are abelian.



\subsection{The finite cyclic groups (\(C_n\) or \(Z_n\))}




\subsection{The finite cyclic groups (\(C_n\) or \(Z_n\))}




\subsection{The circle group \(T\)}

The circle group, \(T\), includes all complex numbers of magnitude \(1\).



\subsection{The circle group \(T\)}

The circle group, \(T\), includes all complex numbers of magnitude \(1\).



\section{Normal subgroups}



\subsection{Cosets and normal subgroups}


A coset is defined between a group and a subgroup of the group.

For a group \(G\), and its subgroup \(H\):

\begin{itemize}
\item The left coset is \(\{gH\}\)
\item The right coset is \(\{Hg\}\)
\end{itemize}

For \(\forall g\in G\).

For abellian groups, the left and right cosets are the same.

The left and right cosets can also be the same, even if the group \(G\) is not abelian.

\subsubsection{Normal subgroups}

If the left and right cosets are the same then \(H\) is a normal subgroup.

\subsubsection{Cosets divide a group.}

Consider two left cosets, \(aH\) and \(bH\), with a common element.

This means that \(ah_i=bh_j\).

We can use this to get:

\(a=bh_jh_i^{-1}\)

\(b=ah_ih_j^{-1}\)

We know that:

\(ah\in aH\)

\(bh\in bH\)

So:

\(bh_jh_i^{-1}h\in aH\)

\(ah_ih_j^{-1}h\in bH\)

And so:

\(bH\subset aH\)

\(aH\subset bH\)

Therefore:

\(aH=bH\)

\subsubsection{Example 1}

Consider the group \(\{-1,1\},\times \)

For the subgroup \(\{1\},\times \), the left coset is \(\{gH\}=\{1,-1\}\).

The right coset is the same.

\subsubsection{Example 2}

Consider the group of integers and addition: \((Z,+)\)

For subgroup \((mZ,+)\), the left and right cosets are the same because the group is abelian.

The coset of the subgroup is the subgroup multiplied by each element in \(G\).

This is \(mZ\), \(mZ+1\), \(mZ+2\) and so on.

Once we reach \(mZ+m\) this has looped, and is already a coset, so we only need the sets upto \(mZ+m-1\).



\subsection{Cosets and normal subgroups}


A coset is defined between a group and a subgroup of the group.

For a group \(G\), and its subgroup \(H\):

\begin{itemize}
\item The left coset is \(\{gH\}\)
\item The right coset is \(\{Hg\}\)
\end{itemize}

For \(\forall g\in G\).

For abellian groups, the left and right cosets are the same.

The left and right cosets can also be the same, even if the group \(G\) is not abelian.

\subsubsection{Normal subgroups}

If the left and right cosets are the same then \(H\) is a normal subgroup.

\subsubsection{Cosets divide a group.}

Consider two left cosets, \(aH\) and \(bH\), with a common element.

This means that \(ah_i=bh_j\).

We can use this to get:

\(a=bh_jh_i^{-1}\)

\(b=ah_ih_j^{-1}\)

We know that:

\(ah\in aH\)

\(bh\in bH\)

So:

\(bh_jh_i^{-1}h\in aH\)

\(ah_ih_j^{-1}h\in bH\)

And so:

\(bH\subset aH\)

\(aH\subset bH\)

Therefore:

\(aH=bH\)

\subsubsection{Example 1}

Consider the group \(\{-1,1\},\times \)

For the subgroup \(\{1\},\times \), the left coset is \(\{gH\}=\{1,-1\}\).

The right coset is the same.

\subsubsection{Example 2}

Consider the group of integers and addition: \((Z,+)\)

For subgroup \((mZ,+)\), the left and right cosets are the same because the group is abelian.

The coset of the subgroup is the subgroup multiplied by each element in \(G\).

This is \(mZ\), \(mZ+1\), \(mZ+2\) and so on.

Once we reach \(mZ+m\) this has looped, and is already a coset, so we only need the sets upto \(mZ+m-1\).



\subsection{Quotient groups}

We have a group \(G\) and a normal subgroup \(N\).

We define a quotient group from this as \(G/N\). This is the set of cosets from \(N\).



\subsection{Quotient groups}

We have a group \(G\) and a normal subgroup \(N\).

We define a quotient group from this as \(G/N\). This is the set of cosets from \(N\).



\subsection{Group extension}

This defines a group \(G\) from a normal subgroup \(N\) and a quotient group \(Q\).



\subsection{Group extension}

This defines a group \(G\) from a normal subgroup \(N\) and a quotient group \(Q\).



\section{Theorems}



\section{Theorems}



\subsection{Cayley's theorem}

Cayley's theorem states that every group \(G\) is isomorphic to a subgroup of the symmetric group acting on \(G\).

Multiplication by a member of \(G\) is a bijective function, as for each \(g\) there is also a \(g^{-1}\).

This means that multiplication of each member of \(G\) is a permutation, and so is a subset of the symmetric group on \(G\).



\subsection{Cayley's theorem}

Cayley's theorem states that every group \(G\) is isomorphic to a subgroup of the symmetric group acting on \(G\).

Multiplication by a member of \(G\) is a bijective function, as for each \(g\) there is also a \(g^{-1}\).

This means that multiplication of each member of \(G\) is a permutation, and so is a subset of the symmetric group on \(G\).



\subsection{Lagrange's theorem}

Lagrange's theorem states that for any finite group \(G\), the order of every subgroup is a divisor of the order of \(G\).

Consider subset \(H\). We know that all cosets are disjoint, and that the union of all cosets is \(G\).

As cosets are the same size, we know that:

\(|G|=m|H|\), where \(m\) is the number of cosets.

This means that if a group has order \(10\), a subgroup must have order \(1\), \(2\) \(5\) or \(10\).



\subsection{Lagrange's theorem}

Lagrange's theorem states that for any finite group \(G\), the order of every subgroup is a divisor of the order of \(G\).

Consider subset \(H\). We know that all cosets are disjoint, and that the union of all cosets is \(G\).

As cosets are the same size, we know that:

\(|G|=m|H|\), where \(m\) is the number of cosets.

This means that if a group has order \(10\), a subgroup must have order \(1\), \(2\) \(5\) or \(10\).



\section{Group action}



\subsection{Group action}

We have a group \(G\) and a set \(S\).

We have a function \(g.s\) which maps onto \(S\) such that:

\begin{itemize}
\item \(I.s=s\)
\item \((gh).s=g(h.s)\)
\end{itemize}



\subsection{Group action}

We have a group \(G\) and a set \(S\).

We have a function \(g.s\) which maps onto \(S\) such that:

\begin{itemize}
\item \(I.s=s\)
\item \((gh).s=g(h.s)\)
\end{itemize}




%prevening attacks: ulimit to prevent fork bomb[ie setting up many processes to do a denial of service attack]. ulimit also prevents excessive use of memory/cpu. stored in /etc/security/limits.conf
% Linux Security Modules and SElinux page
%/usr generally read only stuff
% /var more write only?






\part{Systemd}
% new includes init, service
%\include{cron} %old

\part{Compiling}

% diff and patch
% diff -u oldFile newFile > mods.diff  # -u tells diff to output unified diff format
% patch < mods.diff
% xdelta is similar, but aimed at binary files, whereas diff and patch are aimed at text files

\include{gcc}
\include{cHeaders}
\include{gdb}
\include{make}
\include{cLibrariesNamespaces}
\include{cMacros}

%\include{stackTrace} % using backtrace() function from glibc to do stack trace

\part{Programming techniques}
\include{staticAnalysis}
\include{dynamicAnalysis}

\part{Package management}
% uname[to get info on kernel etc]

% debian package updates
% apt-get update to get list of packages possible updates
% apt-get upgrade to upgrade packages

% debian install package:
% apt-get install <package>
% apt-get remove <package>
% apt-get purge <package> # removes conf files too

% debian rebuild package from source:
% apt-get build-dep <package> # gets dependencies of building pacakge
% apt-get source <package>

% apt exists as alternative to apt-get and apt-cache. front end for it, somewhat more user friendly

\include{debian}
\include{redHat}
\include{arch}
\include{docker}
%\include{snapFlatpakAppImage}

\part{Scripting}
\include{sh}
\include{bash}
\include{awk}
\include{perl}

\part{Parallel programming}
% map reduce
% race conditions
% multi threading
% thread safety. mutex. semaphore. locks.
% futures and promises
% cuda and gpgpu
\include{parallel}

\part{Databases}
\include{csv}
\include{XML}
\include{JSON}
\include{YAML}
\documentclass[oneside]{book}

\usepackage{amsmath, amssymb}
\usepackage{pgfplots}
\usepackage{parskip}

\begin{document}

\author{Adam Boult (www.bou.lt)}
\title{Single-node databases and Structured Query Language (SQL)}
\maketitle

\setcounter{tocdepth}{0}
\tableofcontents

\chapter*{Preface}
\addcontentsline{toc}{chapter}{Preface}

This is a live document, and is full of gaps, mistakes, typos etc.



\part{Databases}
\include{basics}
\include{SQL}
\include{SQLite}

\part{NoSQL}
\include{mongoDB}

\part{Improving knowledge}
\include{knowledge}
\include{expert}

\end{document}


\include{SQL}
\include{SQLite}
%\include{noSQL} % mongo

\part{Device storage}
%commands:
%+ fdisk
%+ mount
%+ umount
%+ swapon
%+ mkswap
%+ df
%+ mkfs
%+ fsck[fix file system]
%swapfile (separate to partition)
%fstab
%/mnt
%/media

\part{Text editors}
\include{vim}
\include{emacs}

\part{Encryption}
\include{encryptionHardDrive}
\include{encryptionRAM}

\part{Decompiling}
\include{hexEditor}
\include{ghidra}
%\include{trojan} % concept of time bomb/logic bomb? payload
%\include{virus} % To networks?and worms? ceoncept of spread and replication not in trojan

\part{Improving knowledge}
\include{knowledge}
\include{expert}

\end{document}

