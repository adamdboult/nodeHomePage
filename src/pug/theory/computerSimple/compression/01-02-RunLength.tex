
\subsection{Run-length encoding: The ND model}

eg 12W6RABC4D is WWWWWWWWWWWWRRRRRRABCDDD

or 4444444aaaaaa123 to 447aa6123

ND model. N is number of repeats, D is what to repeat. if bigger than N can take, then split up

eg 111111111111: 9131

\subsection{RLE with binary/bitstream}

thing next on how that works with binary/bitsteam (eg could do 3 bits at a time for 85)

\subsection{Run-length encoding: The data packet model}

If there is something which repeats a lot (eg 0) then can split that out and then do data packets for the rest

eg if we have 00003640000000000006305: 04364090363015

this is RND model?

The strength of RLE with data packets depends on frequency of special character.

\subsection{Run-length encoding with delta encoding}

we can use delta encoding to make repeated characters more likely to be 0 and non zero is present.

do 2 digits to show going to be a run

what about cases like 1111122222

becomes 115225, but how do we know it's not 52 1s, a 2 then a 5? encoding tricks?

\subsection{LZW compression}

A. Lempel and J. Ziv, with later modifications by Terry A. Welch

code table. eg \(2^12 = 4096\) codes. first \(256 (0-255)\) are the literal bytes


256-4095 are blocks of bytes

algorithm is how to determine code table

\subsection{zip, deflate and lzma2}

zip

deflate

lzma2

