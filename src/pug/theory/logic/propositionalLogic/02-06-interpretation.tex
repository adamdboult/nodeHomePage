\subsection{Interpretations}

An interpretation assigns meaning to propositional variables in a formula.

For example an interpretation of the formula \(\theta \lor \gamma \) assigns values to each of \(\theta \) and \(\gamma \).

\subsection{Satifisable}

A formula is satisfisable if there is some interpretation where it is true.

For example \(\theta \) is satisfisable but \(\theta \land \neg \theta \) is not.

\subsection{Tautology}

A formula is a tautology if it is true in all interpretations.

Examples of tautologies include:
\begin{itemize}
\item \(\theta \lor \neg \theta \)
\end{itemize}

