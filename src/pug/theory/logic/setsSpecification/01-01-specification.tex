
\subsection{Axiom schema of specification}

\subsubsection{The axiom schema of unrestricted comprehension}

We want to formalise the relationship between the preterite and the set. An obvious way of doing this is to add an axiom for each preterite in our structure that:

\(\forall x \exists s[P(x)\leftrightarrow (x\in s)]\)

This is known as "unrestricted comprehension" and there are problems with this approach.

Consider a predicate for all terms which are not members of themselves. That is:

\(\neg (x\in x)\)

This implies the following is true:

\(\forall x\exists s[\neg (x\in x) \leftrightarrow (x\in s)]\)

As this is true for all \(x\), it is true for \(x=s\). So:

\(\exists s[\neg (s\in s) \leftrightarrow (s  \in s)]\)

This statement is false. As we have inferred a false formula, the axiom of unrestricted comprehension does not work. This result is known as Russel's Paradox.

This is an axiom schema rather than an axiom. That is, there is a new axiom for each preterite.

\subsubsection{Axiom schema of specification}

To resolve Russel's paradox, we amend the axiom schema to:

\(\forall x \forall a \exists s[(P(x)\land x\in a )\leftrightarrow (x\in s)]\)

That is, for every set \(a\), we can define a subset \(s\) for each predicate.

This resolves Russel's Paradox. Let's take the same steps on the above formula as in unrestricted comprehension;

\(\forall x \forall a \exists s[(\neg (x\in x)\land x\in a )\leftrightarrow (x\in s)]\)

\(\exists s[(\neg (s\in s)\land s\in s )\leftrightarrow (s\in s)]\)

So long as the subsets \(s\) are not members of themselves, this holds.

