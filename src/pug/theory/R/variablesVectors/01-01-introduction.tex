
\subsection{Introduction}


\subsection{Filter}
Filter()


subsetting vectors eg x[!is.na(x)].

slices

\subsection{Reduce and length}
Reduce()

length()
\subsection{Map}
Map()

\subsection{Type checking}
is.[x]

typeof

\subsection{Random numbers}
random numbers

set.seed




\subsection{Sorting vectors}

\subsection{Factors on vectors}
factors in r: ordered, unordered

\subsection{Casting between types}
as.name

as.[x]

\subsection{Defining variables}

c function is concatenate

\begin{verbatim}
a =  c(1, 2, 3)
a <- c(1, 2, 3)
\end{verbatim}
defining using eg 1L to get variable types

nested vector just a vector

\begin{verbatim}
y <- c(  1, 2 , 3)
y <- c(c(1, 2), 3)
\end{verbatim}
y <- c(x, 1). 



complex numbers, other number types in r


getting sequences, 1:30 etc

seq() function
rep() function


\subsection{Using variables on the right hand side}


is R copy on write? what happens if you set y = x? copies or just two pointers?

\begin{verbatim}
a <- c(1, 2, 3)
b <- c(a, 1)
\end{verbatim}

\begin{verbatim}
a <- c(1, 2, 3)
b <- a + 1
\end{verbatim}


\subsection{Substituting by conditions}
substituting by conditions. eg

\begin{verbatim}
y[y<0]<-0
\end{verbatim}

\subsection{Arithmetic}

\subsection{Logic}

and or in R

\subsection{SORT}


what happens when make anything? all on heap with variable just being pointer?


infinity in R


inf eg 1/0 (-inf?)

+ most simple is a vector
+ arithmetic page:
  * does stuff for each entry in vector eg x + 2
  * summary stuf for vector. sum, min, max, len etc
+ null, na etc in R
+ nan


+ see what name points to:
  * eg if x <- c(1,2,3)
  * y <- x
  * addresses are same
  * if modify, addresses are different
  * only creates copy when needs to
  * aka copy on modify
  * r objects generally immutable
  * tra


overflows of numbers in r. eg what happens if int gets too big

dates and times in base R (other data types, int log etc)


