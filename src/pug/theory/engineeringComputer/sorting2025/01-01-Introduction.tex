
\subsection{Introduction}

are diff and sdiff on linux?

"drivers" somewhere. in big title?
in context of no OS

instructions can take multiple cycles to complete

integrated circuits
+ Very Large Scale Integration
+ Metal Oxide Semiconductor (MOS) chips

manufacturing CPUs:
+ overclocking/cpu scaling
+ ram hz?

pipeline model for CPU
bubbles in cpu? related to pipeline model.

in title: pipeline, instruction level parallel, superscalar, microcode/firmware

Super scaler processor: can do multiple operations per cycle by having multiple execution units. Separate concept to multiple cores






\subsection{Introduction}

+ something on cache
  * SRAM (stochastic random-access memory) as opposed to DRAM (dynamic access memory) which is regular ram
  * part of the CPU in practice
  * in between register and DRAM

\subsection{Branch prediction}
Will a conditional jump be taken?

\subsection{Branch target prediction}
Predict where to go next

cpu out of order execution

cache
branch prediction


\subsection{Introduction}

Data (SIMD) (RV64IMFDQV\_Ziscr\_Zifencei) and Basic Linear Algebra Subprograms (BLAS) (computerInstructions)
int ALU: multiplier-accumulator (MAC)
+ does this belong in mult section or not?
+ i think this is part of P, not part of standard multiply anyway

rv32im
with multiplication and division

rv32imf
with float

floating ALU: fused multiply-add (FMA)
+ system on a chip
+ SSE/AVX
+ Microcode


on floating, include RV64IMF



\subsection{Introduction}

cisc:
+ When discussing cisc, discuss "register memory" as alternative to "Load store". "CISC and the register-memory architecture"?
+ register memory architecture allows operations on memory as well as registers

\subsection{CISC}
op codes can take many clocks built from simpler instructions
+ actually this applies to risc too. point is other things right? different instruction lengths, direct RAM access in operations

















FOR BELOW. ARE ARCHITECTURES FEATURES OF A CPU INSTRUCTION SET, OR FEATURES OF ITS IMPLEMENTATION? impacts where this should go.
+ these are about implementation, so later









concepts of different architecutes:
+ harvard architecture: separate buses for data and code (and possibly different memory too) (these are data bus and address bus)
+ von neumann: single bus for both, same memory
+ data bus and address bus, but on same memory






psu notes
atx 20+4 main power cable. 20 on older/lower power motherboards
+ aka p1, pc main, atx cable
+ powers motherboard, ram, notably not CPU. 20+4 cable means can go with either 24 or 20 motherboard
pci e power cables are not symettric. only one end will fit in PSU.
+ 6 or 8 pin. 6+2 cables can be used.
+ GPUs use pci e.
+ not all pci e devices need extra power, eg wireless card. GPUs generally do.
modular vs non-modular PSU
cpu connector on motherboard. either 4 or 8 pin. 4+4 cable allows for flexibility.
+ aka p4, eps connector. 12V
other power: sata cable. required for sata, but can also be used for eg rgb lights. 6pin

dram: row hammer attack

h3 on validating CPUs in big page on manufacturing CPUs
in perihperals: concept of memory mapped IO.
parts of CPU: branch unit; dispatch unit
microarchitecture: the way  an instruction set architecture is implemented. could ave differnet microarchitecture for the same ISA
Load store unit as specialised execution unit

manufuacturing cpus and perihperals: thign on io bus


