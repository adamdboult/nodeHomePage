
\subsection{Introduction}


iterator stuff:
+ "yield" in functions
+ generators
+ "next"

\subsection{Object literals}

can define object literals

\begin{verbatim}
const person = {firstName:"John", lastName:"Doe", age:50, eyeColor:"blue"};

const person = {};
person.firstName = "John";
person.lastName = "Doe";
person.age = 50;
person.eyeColor = "blue";
\end{verbatim}

\subsection{Access}

\begin{verbatim}
objectName.property
objectName["property"]
\end{verbatim}


can iterate over properties in object
for (let variable in object) {
  // code to be executed
}

Object.getOwnPropertyNames()

Object.values(myObj); gets values

\subsection{Deleting parts of object}

can delete
\begin{verbatim}
delete person.age;
delete person["age"];
\end{verbatim}

\subsection{Nested objects}

can nest objects
\begin{verbatim}
myObj = {
  name:"John",
  age:30,
  cars: {
    car1:"Ford",
    car2:"BMW",
    car3:"Fiat"
  }
}
\end{verbatim}

Can access nested parts of objects.
\begin{verbatim}

myObj.cars.car2;
myObj["cars"]["car2"];
\end{verbatim}


\subsection{Printing objects}

\begin{verbatim}
JSON.stringify(myObj); can print
\end{verbatim}

\subsection{Getters and settings}

getters and setters for object. can access and update values in a more flexible way.

\begin{verbatim}
const person = {
  firstName: "John",
  lastName: "Doe",
  language: "en",
  get lang() {
    return this.language.toUpperCase();
  }
};
person.lang; will retrun En
\end{verbatim}

const person = {
  firstName: "John",
  lastName: "Doe",
  language: "",
  set lang(lang) {
    this.language = lang.toUpperCase();
  }
};
\begin{verbatim}
person.lang = "en"; will make language be "En"
\end{verbatim}

methods in object literals
\begin{verbatim}
const person = {
  firstName: "John",
  lastName: "Doe",
  id: 5566,
  fullName: function() {
    return this.firstName + " " + this.lastName;
  }
};
\end{verbatim}


constructors to make objects

\begin{verbatim}
function Person(first, last, age, eye) {
  this.firstName = first;
  this.lastName = last;
  this.age = age;
  this.eyeColor = eye;
}
\end{verbatim}

primatives can be done as objects
\begin{verbatim}
new String()    // A new String object
new Number()    // A new Number object
new Boolean()   // A new Boolean object
new Object()    // A new Object object
new Array()     // A new Array object
new RegExp()    // A new RegExp object
new Function()  // A new Function object
new Date()      // A new Date object 
\end{verbatim}

but primatives faster than doing this.

can add things to constructors using prototype

\begin{verbatim}
function Person(first, last, age, eyecolor) {
  this.firstName = first;
  this.lastName = last;
  this.age = age;
  this.eyeColor = eyecolor;
}
Person.prototype.nationality = "English";

const myFather = new Person("John", "Doe", 50, "blue");
const myMother = new Person("Sally", "Rally", 48, "green");
\end{verbatim}

making an object iterable: @@iterator method
use of 

\subsection{thing?}
is generator stuff?
\begin{verbatim}

function* my_function(){
\end{verbatim}

arrays and sets


+ JavaScript this
