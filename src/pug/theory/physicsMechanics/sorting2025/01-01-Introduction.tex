
\subsection{Introduction}

Conformal symmetry in physics?

Are there Newtonian systems which can't be described as Lagrangian/Hamiltonian? Yes. Multiple force functions, just 1 Lagrangian/Hamiltonian?
All lagrangians can be formulated as Newtonian
Can derive this (so long as Hessian of Lagrangian is not zero,which is required for unique systems
All Lagrangian systems are Hamiltonian, can derive this


Euler on the principle of least action: depends on conservation on energy throughout a path, and across all paths

Principle of least action

\(delta int [mvs ds]=0\)
==
\(delta int [mvv dt] =0\)
==
\(delta int [1/2 mv^2 dt]=0\)
\(delta int [T dt] =0\)
Where \(T\) is kinetic energy
==
\(delta int [(E-V) dt] =0\)

Can combine
\(delta int [2T dt]=0\)
\(delta int [T+E-V]dt=0\)
\(delta int [T-V] dt + delta int [E] dt=0\)
Conservation of energy so
\(delta int [T-V] dt + delta (Et) =0\)
\(delta int [T-V] dt + E delta t + t delta E =0\)

t delta E is zero by assumption. Energy on all paths the same
\(delta int [T-V] dt + E delta t =0\)
Further only consider paths where no change in travel time, so delta t =0
\(delta int [T-V] dt =0\)
Is modern version of least action. Hamiltonian principle
Earlier one was maurpurtouis principle

When defining force as delta potential, can then derive f=ma

physics
	after h3 on gravity
		h3 on electrostatics
			introduce forces here
			introduce inertial mass and gravitational mass
			lagrangian for both
			symmetries of them

			charge density fields
		h3 on magnetostatics
			B, H and J fields
		h3 on magnetism
			page on symmetries, can't do v shifts!
			page on electromagnetism?
			introduce lagrangians here?

Symmetries in electrostatics: potential can be scaled. Ditto for gravitational field?
Magnet symmetry: vector potential field can be changed


discrete state transition: atomic states

atomic. all we say is a differnt to b. \(a->b\) (can represent transition as matrix if discrete, integer with mod + 1 etc. or \(e^-H\) if continuous, or real )
 actually is this true? don't we need to represent the state to figure out where to go next, and this gives isomorphism? think about it...
 if finite states then transition matrix can be used, deterministic
 might be possible to simplify the transition matrix though, combination of other transition matrices? can represent the state as a vector of attributes in that case, or as having objects with relations to each other?



